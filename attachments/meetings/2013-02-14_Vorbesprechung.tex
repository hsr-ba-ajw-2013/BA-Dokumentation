\subsection*{14.02.2013 Vorbesprechung}

\paragraph*{Teilnehmer}

\begin{itemize}
	\item Hans Rudin, HRU (HSR)
	\item Kevin Gaunt, KGA (HSR)
	\item Daniel Hiltebrand, DHI (Crealogix)
	\item Manuel Alabor, MAL (Team)
	\item Alexandre Joly, AJO (Team)
	\item Michael Weibel, MWE (Team, Protokoll)
\end{itemize}

\paragraph*{Traktanden}
\begin{itemize}
	\item StoryboardBuilder -- was ist der aktuelle Stand
	\item Ideen/Diskussion andere Themen
\end{itemize}

\paragraph*{StoryboardBuilder}
\begin{itemize}
	\item Ende Januar Entscheid: nicht weiterentwickeln
	\item Nicht weil Produkt/Markt nicht interessant ist
	\item Ziel war: damit die Crealogix UX-Services zu unterstützen
	\item Crealogix wird keine UX-Services gegen aussen mehr anbieten
		\begin{itemize}
			\item mehr interne Projekte betreuen
		\end{itemize}
	\item Zwei Schwerpunkte: Education \& Financial Services
		\begin{itemize}
			\item StoryboardBuilder gehört nicht zu einem solchen Schwerpunkt
		\end{itemize}
	\item Dies obwohl das Potential für das Produkt gesehen wird
	\item Anforderungsspezifikation verfeinert bis Ende Januar
	\item Technischer Prototyp gestartet, aber wieder gestoppt aufgrund der Neuorientierung
	\item Idee wäre: StoryboardBuilder an externe Firma weitergeben
	\item Bestehende Mitbewerber bauen ihre Angebote aus
\end{itemize}

\paragraph*{Diskussion BA}
\begin{itemize}
	\item Frage an die Runde: ist es interessant für euch, die Entwicklung des StoryboardBuilders in der BA weiterzuführen?
	\begin{itemize}
		\item MAL: Wie würde das aussehen?
		\begin{itemize}
			\item DHI: Beispiel aufgrund gewählter Technologie zu entwickeln
			\item DHI: Auf Basis der fachlichen Spezifikation der Crealogix
			\item HRU: BA sollte nicht zu einer Fleissarbeit werden
			\item HRU: Was wären denn die Herausforderungen wenn das in der BA weiterentwickelt werden würde?
			\item HRU: Die momentane Ausgangslage ist anders, da kein wirklicher Kunde existiert
			\item DHI: Es sind sicher einige Ideen da, die technisch Herausfordernd sind
			\item DHI: Grafische Repräsentation auf Screen, Objektmodell umsetzen
			\item DHI: Wie gesagt, hat auch nichts dagegen, wenn eine neue Arbeit gemacht werden würde
			\item DHI: Laut Kenntnisstand von DHI sollte auch von Seiten der Industriepartner weniger Betreuung beinhalten, mehr vom Team
		\end{itemize}
	\end{itemize}
	\item KGA: Was ist nun der Anspruch an das Meeting? Müssen wir am Ende des Meeting schon wissen was gemacht werden soll?
	\begin{itemize}
		\item DHI: Ist offen, hat keinen Anspruch auf Entscheid jetzt - sollte aber bald geschehen, da Bachelorarbeit bald startet
	\end{itemize}
	\item DHI: Hat div. Alternativen die man anschauen könnte
	\item MAL: für ihn ist das Durchführen der BA mit dem StoryboardBuilder nicht mehr besonders interessant, aufgrund Änderungen seitens Crealogix
	\begin{itemize}
		\item DHI: Ja das stimmt, BA würde nicht mehr zu einem realen Projekt führen
	\end{itemize}
	\item MWE: gehts ähnlich wie MAL
	\item DHI: Thema 1
	\begin{itemize}
		\item {\bf Kino-Reservationssystem}
		\item Kennt Kinobesitzer in Rapperswil
		\item Buchungssystem
		\item Mit Anbindung an Kassensystem
		\item nicht nur für einzelne Plätze
		\item sondern auch Firmenanlässe, Frauenkino, Catering etc.
		\item Konzeptionell entwerfen
		\item soweit wie möglich implementieren
		\item könnte sehr interessant sein, DHI's Meinung nach
	\end{itemize}
	\item DHI: Thema 2
	\begin{itemize}
		\item {\bf Personal Financing Management}
		\item Zusatz zu E-Banking
		\item Eigene Zahlungen analyisieren
		\item Verschiedene Banken
		\item Soviel für Versicherung, Einkauf, etc.
		\item Basiert auf einer isländisch-schwedischen Firma
		\item Integrationsarbeit (in ein Bankensystem einbauen)
		\item MAL: Geht es darum, das zu integrieren und anschaulich darzustellen, Data-Mining ist aber bereits gemacht?
		\begin{itemize}
			\item DHI: Ja
			\begin{itemize}
				\item Schwierigkeit: Sicherheit
				\item Zertifizierung, Authentifizierung
				\item nur Teilaspekt möglich zum lösen
				\item Versch. Komponenten
				\item Aspekt wichtig auf welcher sich die BA konzentrieren soll
			\end{itemize}
			\item DHI: müsste das genauer anschauen wie das machbar wäre
			\begin{itemize}
				\item da es ein sehr grosses System wäre
				\item müsste verifizieren ob das gehen soll
			\end{itemize}
		\end{itemize}
		\item HRU: wäre das so kurzfristig machbar?
		\begin{itemize}
			\item DHI: müsste angeschaut werden
			\item DHI: Verträge bestehen
			\item DHI: Anders ist E-Banking in Java mit Oracle
			\begin{itemize}
				\item DHI: Verfügbar machen möglich
			\end{itemize}
		\end{itemize}
		\item HRU: Zwei Komponenten, was ist Webfrontend?
		\begin{itemize}
			\item DHI: Netty server von airlock für Authentifizierung
			\item DHI: möglichst HTML das übers web geht
			\item DHI: J2EE - JSP vorne
		\end{itemize}
		\item HRU: Isländische Software
		\begin{itemize}
			\item DHI: .NET basiert
			\item DHI: hat aber eine WCF/REST/AJAX Schnittstelle welche relativ gut ins Frontend integrierbar wäre
			\item DHI: von einem System ins andere transferieren (Oracle zu MSSQL DB)
			\item Statistisch aufwerten und wieder anzeigen
			\item machbarkeit unklar, muss verifiziert werden
		\end{itemize}
	\end{itemize}
	\item MAL: was hat HRU für Projekte
	\begin{itemize}
		\item MAL: Realtime-Themen wären interessant (Mobile, Chat, Messaging?)
		\item HRU: hat keine Projekte
	\end{itemize}
	\item DHI: hat evtl. noch andere Projekte, müsste das aber noch anschauen
	\item MAL: Bis Testumgebung steht würden wohl Wochen vergehen
	\begin{itemize}
		\item DHI: stimmt wohl
	\end{itemize}
	\item MWE: Persönlich interessiert vorallem an WebRTC
	\item HRU: Was ist mit WebRTC gemeint? 
	\begin{itemize}
		\item MWE: Near-Realtime Communication (Daten, Video, Audio) zw. Browser
	\end{itemize}
	\item MWE: Was wäre denn für Crealogix interessant -- Zentrale Frage
	\begin{itemize}
		\item DHI: Crealogix muss nicht unbedingt dabei sein, wenn nicht nötig
	\end{itemize}
	\item update Maus-Scanner
	\begin{itemize}
		\item MAL: wäre denn Mobile auch ein Thema?
		\item DHI: Mobile ist sehr zentral
	\end{itemize}
	\item HRU: Gibt es denn Fraktionen (web/mobile) im Team?
	\begin{itemize}
		\item MWE: Web-Mensch, aber Mobile wäre auch sehr interessant
	\end{itemize}
	\item DHI: Fragt bei Crealogix CEO/Entwicklungsleiter nach bzgl. Mobile
	\item MAL: E-Learning wäre auch interessant bzgl. Mobile
	\begin{itemize}
		\item DHI: könnte nachfragen obs da auch was geben würde
	\end{itemize}
	\item AJO: Auch vorallem an Mobile interessiert
	\begin{itemize}
		\item arbeitet auch vorallem in Mobile
		\item HRU: Auch sie MAL?
		\item \emph{MAL bejaht}
	\end{itemize}
	\item müsste nachfragen, wäre aber sicher interessant
	\item MAL: wäre sicher interessant auf DHI's Themen zu warten, andererseits müsste auch Teamintern bzw. mit KGA/HRU geschaut werden
	\item DHI: Wann beginnt die Arbeit? -- Montag 18.02.2013
	\item DHI: 3 Bereiche Education
	\begin{enumerate}
		\item Campusmanagement
		\item Time2Learn
		\begin{itemize}
			\item 30'000 die mit T2L arbeiten
			\item Mit Center For Young Professionals zusammenarbeit (Evtl. da was interessantes)
			\item in Bubikon
		\end{itemize}
	\end{enumerate}
	\item DHI: was machen wenn nichts herauskommt?
	\begin{itemize}
		\item DHI: würde gerne wieder mit Team zusammenarbeiten, wenn möglich
	\end{itemize}
	\item HRU: Wäre schon der Weg zum gehen
	\begin{itemize}
		\item parallel müssten Überlegungen angestellt werden, ob es andere Themen geben würde 
		\item gibt den 3 Studierenden möglichst freie Hand
	\end{itemize}
	\item KGA: Termin bis wann die Entscheidung fällen müsste
	\begin{itemize}
		\item HRU: allerspätistens erste Woche
		\item DHI: wird noch heute mit den 3 Crealogix Leuten Kontakt aufnehmen
		\item DHI: bis Morgen, 15.0.2 Antwort wenn möglich
		\item MAL: Mittwoch zu spät oder zu früh?
		\item HRU: Wann sind Sie (Studierende) an der HSR? -- MO/DI/MI
		\item HRU: Mittwoch wäre nicht unbedingt zu spät
		\item MAL: Mittwoch wäre Zusammenkunft um definitiv zu entscheiden. Aber mit Kommunikation bis dann
		\item MAL: DHI kann sicher schnell entscheiden, je nach dem pers. Anwesenheit nicht nötig
		\item DHI: würde gerne pers. dabei sein
	\end{itemize}
	\item DHI: gibt morgen Feedback
	\item HRU: das ist gut -- in der Runde Team/HSR diskutieren was gemacht werden kann
\end{itemize}

\paragraph*{Diskussion mit Team/HSR (ohne DHI)}
\begin{itemize}
\item KGA: die beiden Themen tönen also nicht besonders interessant für das Team?
\item Team: Ja -- Thema 2 wäre interessant aber wohl zu gross für die kurze Zeit
\item HRU: Reservationssystem vorallem Businessanalyse
\item HRU: Sicher gut zu schauen, was DHI einbringt
\item HRU: aber auch schauen was Team/HSR für Ideen hat
\item KGA: was wäre ursprünglich SA-Idee gewesen?
\begin{itemize}
	\item MWE: XMPP Server in node.js -- modular, flexibel (u.a. auch bzgl. Datenbankanbindung)
	\item MAL: und auch Skalierbarkeit von node.js interessant
\end{itemize}
\item MAL: interessant wäre vielleicht auch Frontend Framework
\item HRU: Alle miteinander auf dem Laufenden halten bzgl. Ideen
\item \ldots weitere eher informelle Diskussion mit Team/HSR \ldots
\end{itemize}

\paragraph*{Nächstes Meeting}

Mittwoch, 20. Februar 2013, 10:10 Uhr