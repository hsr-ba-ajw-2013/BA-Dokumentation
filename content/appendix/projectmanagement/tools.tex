\section{Tools}
\label{sec:tools}

Ergänzend zu diesem Abschnitt ist der Anhang \ref{sec:urls} ``\nameref{sec:urls}'' zu erwähnen. Er enthält alle wichtigen Internetadressen zu den spezifischen Tools und Code Repositories.

\subsection{Projektverwaltung}
Für die komplette Projektplanung, die Zeitrapportierung sowie das Issue-Management wird Redmine eingesetzt.

Der aktuelle Stand der Arbeiten am Projekt kann hier jederzeit eingesehen werden und wird vom Projektteam aktiv aktualisiert.


\subsection{Entwicklungsumgebung}
Zur Entwicklung von Quellcode-Artefakten steht eine mit Vagrant \cite{Vagrant} paketierte Virtual Machine bereit. Sie enthält alle notwendigen Abhängigkeiten und Einstellungen:

% FIXME bessere dokumentation der vagrant box

\begin{itemize}
	\item node.js 0.10.0
	\item PostgreSQL 9.1
	\item Ruby 2.0.0 (installiert via rvm)
	\item ZSH (inkl. oh-my-zsh)
\end{itemize}

Das Code Repository enthält ein \emph{Vagrantfile} welches durch den Befehl \emph{vagrant up} in der Kommandozeile automatisch das Image der vorbereiteten VM lokal verfügbar macht und startet.

\subsection{Git Repositories}
Sowohl Quellcodeartefakte als auch die in LaTeX formulierte Thesis (dieses Dokument) wird in auf GitHub abgelegten Git Repositories versioniert bzw. zentral gespeichert.

\subsection{Continuous Integration}
Für das Projekt wird Travis CI als Continuous Integration Lösung verwendet. Nähere Informationen sind Abschnitt ``\nameref{sec:qualitymanagement}'' unter ``\nameref{sec:continuousintegration}'' zu finden.