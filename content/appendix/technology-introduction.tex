\chapter{Technologie Einführung}

Der folgende Anhang beschreibt einige Ressourcen und Tutoriale um sich selber
auf den aktuellen Stand in Sachen \nameref{sec:ti-javascript},
\nameref{sec:ti-nodejs} und \nameref{sec:ti-web-entwicklung} zu bringen.

\section{JavaScript}
\label{sec:ti-javascript}

\subsection*{Einführung}
\begin{itemize}
	\item
\end{itemize}

\subsection*{Fortgeschrittenes}
\begin{itemize}
	\item JavaScript: The Good Parts \cite{Crockford:2008:JGP:1386753} \\
		Sobald ein Basiswissen über JavaScript erreicht wurde, ist dieses Buch
		quasi unumgänglich. Es klärt über die Unzulänglichkeiten von JavaScript
		auf und gibt wertvolle Tipps, die JavaScript angenehm machen.
	\item Javascript Module Pattern \cite{JSModulePattern} \\
		Damit grössere Applikationen strukturiert werden können benötigt es
		momentan noch (bis ECMAScript 6 Standard ist) klar definierte Patterns.
		Das Module Pattern ist nach Ansicht der Autoren eines der besten davon.
\end{itemize}

\subsection*{Weitere Links}
\begin{itemize}
	\item Mozilla Developer Network Wiki \cite{MDN} \\
		Ausführliches Nachschlagewerk für JavaScript, CSS und
		andere Web--APIs und --Technologien.
\end{itemize}

\section{Node.js}
\label{sec:ti-nodejs}

\subsection*{Einführung}
\begin{itemize}
	\item Felix's Node.js Guide \cite{Nodeguide} % FIXME
	\item Node Tutorials \cite{NodeTuts}
	\item Node.js API Docs \cite{NodejsAPIDocs}
\end{itemize}

\subsection*{Fortgeschrittenes}
\begin{itemize}
	\item
\end{itemize}

\subsection*{Weitere Links}
\begin{itemize}
	\item
\end{itemize}

\section{Web Entwicklung}
\label{sec:ti-web-entwicklung}

\subsection*{Einführung}
\begin{itemize}
	\item
\end{itemize}

\subsection*{Fortgeschrittenes}
\begin{itemize}
	\item
\end{itemize}

\subsection*{Weitere Links}
\begin{itemize}
	\item
\end{itemize}