\chapter{Technologie Einführung}

Der folgende Anhang beschreibt einige Ressourcen und Tutoriale um sich selbst
auf den aktuellen Stand in Sachen \nameref{sec:ti-javascript} und \nameref{sec:ti-nodejs} zu bringen.
Weiter zeigt \ref{sec:ti-used-libraries} einige Websites für Bibliotheken die in der
Beispielapplikation benutzt wurden, auf.

\section{JavaScript}
\label{sec:ti-javascript}

\subsection*{Links für Fortgeschrittene}
\begin{itemize}
	\item JavaScript: The Good Parts \cite{Crockford:2008:JGP:1386753} \\
		Sobald ein Basiswissen über JavaScript erreicht wurde, ist dieses Buch
		quasi unumgänglich. Es klärt über die Unzulänglichkeiten von JavaScript
		auf und gibt wertvolle Tipps, die JavaScript angenehm machen.
	\item Javascript Module Pattern \cite{JSModulePattern} \\
		Damit grössere Applikationen strukturiert werden sind momentan noch (bis ECMAScript 6 Standard ist) klar definierte Patterns nötig.
		Das Module Pattern ist nach Ansicht der Autoren eines der besten davon.
	\item Secrets of the JavaScript Ninja \cite{resig2012secrets} \\
		Der jQuery-Author beschreibt in diesem Buch Techniken um zum
		JavaScript-Guru aufzusteigen.
\end{itemize}

\subsection*{Weitere Links}
\begin{itemize}
	\item Mozilla Developer Network Wiki \cite{MDN} \\
		Ausführliches Nachschlagewerk für JavaScript, CSS und
		andere Web--APIs und --Technologien.
\end{itemize}

\section{Node.js}
\label{sec:ti-nodejs}

\subsection*{Einführung}
\begin{itemize}
	\item Felix's Node.js Guide \cite{Nodeguide} \\
		Felix Geisendörfer ist ein Kern-Entwickler von node.js und hat verschiedene
		Tutoriale über Node.js geschrieben.
	\item Node Tutorials \cite{NodeTuts}
	\item Mastering Node.js \cite{MasteringNode} \\
		Kostenloses Buch vom Express.js \cite{Expressjs} etc. Entwickler TJ
		Holowaychuck \cite{TJH}. Beinhaltet sowohl Informationen für Anfänger wie
		auch für Fortgeschrittene.
	\item Node.js API Docs \cite{NodejsAPIDocs}
\end{itemize}


\section{Einführung für benutzte Libraries}
\label{sec:ti-used-libraries}
\begin{itemize}
	\item Express.js Guide \cite{ExpressGuide} \\
		Erste Schritte mit Express.js können getrost mithilfe des Guides von
		Express.js gemacht werden. Sobald man die grundsätzliche Struktur einer
		Applikation mithilfe des Frameworks versteht, ist es hilfreich, einige
		Beispiele \cite{ExpressExamples} anzuschauen. Die ``Roomies''
		Applikation kann natürlich auch als Beispiel fungieren.
	\item Handlebars.js \cite{Handlebars} \\
		Die Handlebars.js Website bietet direkt eine Einführung in das Templating.
	\item Developing Backbone.js Applications \cite{BackboneFundamentals} \\
		Dieses Buch beinhaltet eine Einführung in die relevanten Konzepte von Backbone.js,
		zeigt Beispielapplikationen und auch weiterführende Libraries/Konzepte.
\end{itemize}