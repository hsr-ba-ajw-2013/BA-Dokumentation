\chapter{Installationsanleitung}

\section{Mit Vagrant}

\subsection*{Grundvoraussetzungen}
Folgende Software muss im Voraus installiert sein. Es wird hier nicht genauer
darauf eingegangen wie diese installiert wird.

\begin{itemize}
	\item \textbf{git} \cite{git}
	\item \textbf{Vagrant} \cite{Vagrant}
	\item \textbf{Virtualbox} \cite{Virtualbox}
\end{itemize}

\subsection*{Installation der Roomies-Virtualbox}
Für ``<TAG>'' muss der richtige ``git tag'' eingesetzt werden. Falls das nicht gewünscht ist, kann auch auf dem ``master''-Branch gearbeitet werden.

\begin{lstlisting}[language=Bash, caption=Installationsanleitung Vagrant}
~ $> git clone git://github.com/hsr-ba-ajw-2013/BA.git && cd BA

# Virtualbox starten
~BA/ $> vagrant up

# Zur Virtualbox per SSH verbinden
~BA/ $> vagrant ssh

# Initiales setup
~BA/ $> ./install.sh

# Applikation starten
~BA/ $> npm start
\end{lstlisting}

\section{Ohne Vagrant}

\subsection*{Grundvoraussetzungen}
Folgende Software muss im Voraus installiert sein. Es wird hier nicht genauer
darauf eingegangen wie diese installiert wird.

\begin{itemize}
	\item \textbf{Node.js} Version 0.8.x \cite{nodejs}\\
		Eventuell mittels \textbf{NVM} \cite{NVM} falls nicht anders möglich.
	\item \textbf{PostgreSQL} \cite{PostgreSQL} oder SQLite \cite{SQLite}
	\item \textbf{GNU Make} \cite{GNUmake}
	\item \textbf{git} \cite{git}
\end{itemize}

\subsection*{Installation der Roomies-Applikation}

Für ``<TAG>'' muss der richtige ``git tag'' eingesetzt werden. Falls das nicht gewünscht ist, kann auch auf dem ``master''-Branch gearbeitet werden.

\begin{lstlisting}[language=Bash, caption=Installationsanleitung Vagrant}
~ $> git clone git://github.com/hsr-ba-ajw-2013/BA.git && cd BA

# Initiales setup
~BA/ $> ./install.sh

# Applikation starten
~BA/ $> npm start
\end{lstlisting}

\subsubsection*{Konfiguration}
Für die Konfiguration müssen die Datenbank-Informationen angepasst werden.