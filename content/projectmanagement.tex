\chapter{Projektmanagement}

\section{Plattformen}
\subsection{Thesis}
Die Bachelor-Thesis ist zwecks vereinfachter Zusammenarbeit im LaTeX-Format in folgendem Git-Repository abgelegt:
\begin{itemize}
	\item \url{https://github.com/mweibel/BA-Dokumentation}
\end{itemize}

\subsection{Planung, Tracking \& Zeitrapportierung}
Zur Planung und Verfolgung des aktuellen Arbeitsfortschrittes sowie zur Zeitrapportierung wird eine Redmine-Installation verwendet:
\begin{itemize}
	\item \url{http://redmine.alabor.me/projects/ba2013}
\end{itemize}

\subsection{Code Repository}
Zur Versionierung von Quellcode sowie zentraler Verwaltung dessen wird ein Git-Repository auf github.com verwendet:
\begin{itemize}
	\item \url{https://github.com/mweibel/ba}
\end{itemize}

\subsection{Andere Artefakte}
Jegliche andere Artefakte werden zentral auf einem Dropbox-Share versioniert und verwaltet.


\section{Meetings}
\subsection{Wöchentliches Statusmeeting}
Es findet jeweils am Mittwoch um 10 Uhr ein woöchentliches Statusmeeting statt. Die Sitzung wird abwechslungsweise jeweils von einer Person aus dem Projektteam geführt und von einer anderen protokolliert.

\subsection{Protokollführung}
Alle Protokolle zu abgehaltenen Meetings sind im Thesis-Git-Repository einsehbar:
\begin{itemize}
	\item \url{https://github.com/mweibel/BA-Dokumentation/wiki/Meetings}
\end{itemize}

\section{Phasenplanung}
TODO

\section{Meilensteine}
TODO

\section{Personelles}
\subsection{Abwesenheiten}
Folgende Abwesenheiten sind bekannt und werden entsprechend kompensiert.

\begin{table}[H]
\tablestyle
\tablealtcolored
\begin{tabularx}{\textwidth}{l l l X}
\tableheadcolor
	\tablehead Wer? &
	\tablehead Von &
	\tablehead Bis &
	\tablehead Grund \tabularnewline
\tablebody
	\textit{MAL} & 02.04.2013 & 05.04.2013 & Militär \tabularnewline
	\textit{MAL} & 23.05.2013 & 31.05.2013 & Militär (Mitarbeit in der zweiten Hälfte machbar) \tabularnewline
\tableend
\end{tabularx}
\caption{Im Voraus bekannte Abwesenheiten}
\end{table}