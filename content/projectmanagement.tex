\chapter{Projektplanung}

\section{Infrastruktur}
\begin{table}[H]
\tablestyle
\tablealtcolored
\begin{tabularx}{\textwidth}{l X}
\tableheadcolor
	\tablehead Ressource &
	\tablehead URL \tabularnewline
\tablebody
	\textit{Projektverwaltung} &  \url{http://redmine.alabor.me/projects/ba2013}\tabularnewline
	\textit{Code: Git Repository} &  \url{https://github.com/mweibel/ba}\tabularnewline
	\textit{Code: \gls{CI}} &  \url{https://travis-ci.org/mweibel/BA}\tabularnewline
	\textit{Thesis: Git Repository} & \url{https://github.com/mweibel/BA-Dokumentation}\tabularnewline
	\textit{Thesis: PDF} & \url{http://mweibel.github.com/BA-Dokumentation/thesis.pdf}\tabularnewline
	\textit{Thesis: \gls{CI}} & \url{https://travis-ci.org/mweibel/BA-Dokumentation}\tabularnewline
	\textit{Meeting Protokollierung} & \url{https://github.com/mweibel/BA-Dokumentation/wiki/Meetings}\tabularnewline
\tableend
\end{tabularx}
\caption{Projektrelevante URL's}
\end{table}

\subsection{Projektverwaltung}
Für die komplette Projektplanung, die Zeitrapportierung sowie das Issue-Management wird Redmine eingesetzt.

\subsection{Entwicklungsumgebung}
Zur Entwicklung von Quellcode-Artefakten steht eine mit Vagrant\cite{Vagrant} provisionierte Virtual Machine bereit. Sie enthält alle notwendigen Abhängigkeiten und Einstellungen.

Eine \gls{IDE} im klassischen Sinne wird nicht vorgeschrieben.

\subsection{Git Repositories}
Sowohl Quellcodeartefakte als auch die in LaTeX formulierte Thesis (dieses Dokument) wird in auf GitHub abgelegten Git Repositories versioniert bzw. zentral gespeichert.

\subsection{Continous Integration}
Dieses Projekt verwendet Travis CI als Continous Integration Lösung.

Beide Git Repositories (Code \& Thesis) verfügen über einen \gls{PushHook} welcher automatisch einen Build im CI-System auslöst.



\section{Meetings}
\subsection{Regelmässiges Statusmeeting}
Während der gesamten Projektdauer findet jeweils am Mittwoch um 10 Uhr ein wöchentliches Statusmeeting statt. Die Sitzung wird abwechslungsweise jeweils von einer Person aus dem Projektteam geführt sowie von einer anderen protokolliert.

Das Projektteam stellt die Agenda der aktuellen Sitzung bis spätenstens am vorangehenden Dienstag Abend bereit.

\section{Phasenplanung}
Die Phasenplanung orientiert sich grob am \gls{RUP} und ist unterteilt in eine \emph{Inception-}, \emph{Elaboration-}, fünf \emph{Construction-} sowie jeweils eine \emph{Transition-} und \emph{Abschlussphase}.

% FIXME

!!ADD PHASES IMAGE HERE!!


\section{Meilensteine}
\begin{table}[H]
\tablestyle
\tablealtcolored
\begin{tabularx}{\textwidth}{l l l X}
\tableheadcolor
	\tablehead ID &
	\tablehead Meilenstein &
	\tablehead Termin &
	\tablehead Beschreibung \tabularnewline
\tablebody
	\textit{M1} & Abschluss Inception & 10.03.2013 & Die Aufgabenstellung wurde gem. Auftrag klar definiert und die Projektinfrastruktur ist aufgesetzt. Eine initale Projektplanung besteht.\tabularnewline
	\textit{M2} & Abschluss Elaboration & 17.03.2013 & Neben der Auswahl einer konkreten Technologie sind nun auch Guidelines definiert. Anforderungsdokumente sind erstellt und abgenommen. Zudem besteht ein initiales Software Architektur Dokument mit dazugehörigem Architekturprototypen.\tabularnewline
	\textit{M3} & Abschluss Construction 1 & 31.03.2013 & Das Fundament der Applikation wurde implementiert. Weiter wurden die ersten Use Cases der Priorität \emph{Hoch} umgesetzt.\tabularnewline
	\textit{M4} & Abschluss Construction 2 & 14.04.2013 & Alle Use Cases der Priorität \emph{Hoch} sind umgesetzt.\tabularnewline
	\textit{M5} & Abschluss Construction 3 & 28.04.2013 & \tabularnewline
	\textit{M6} & Abschluss Construction 4 & 12.05.2013 & \tabularnewline
	\textit{M7} & Abschluss Construction 5 & 26.05.2013 & \tabularnewline
	\textit{M8} & Abschluss Transition & 02.06.2013 &  \tabularnewline
	\textit{M9} & Abgabe HSR Artefakte & 07.06.2013 & Das A0-Poster sowie die Kurzfassung der Bachelorarbeit sind dem Betreuer zugestellt.\tabularnewline
	\textit{M10} & Abgabe Bachelorarbeit & 14.06.2013 & Alle abzugebenden Artefakte sind dem Betreuer zugestellt worden.\tabularnewline
\tableend
\end{tabularx}
\caption{Meilensteine}
\end{table}


\section{Artefakte}
Dieser Abschnitt beschreibt alle Arbeitsprodukte (Artefakte), welche zwingend erstellt und abgegeben werden müssen.


!!!!!! TODO ADD ARTEFACTS !!!!!!!!!!!

\begin{table}[H]
\tablestyle
\tablealtcolored
\begin{tabularx}{\textwidth}{l l l X}
\tableheadcolor
	\tablehead ID &
	\tablehead Meilenstein &
	\tablehead Artefakt &
	\tablehead Beschreibung \tabularnewline
\tablebody
	\textit{A10} & M1 & Projektplanung & ...\tabularnewline
	\textit{A10} & M1 & Projektplanung & ...\tabularnewline
	\textit{A10} & M1 & Projektplanung & ...\tabularnewline
	\textit{A10} & M1 & Projektplanung & ...\tabularnewline
\tableend
\end{tabularx}
\caption{Abzugebende Artefakte}
\end{table}