\section{Infrastruktur}
\begin{table}[H]
\tablestyle
\tablealtcolored
\begin{tabularx}{\textwidth}{l X}
\tableheadcolor
	\tablehead Ressource &
	\tablehead URL \tabularnewline
\tablebody
	\textit{Projektverwaltung} &  \url{http://redmine.alabor.me/projects/ba2013}\tabularnewline
	\textit{Code: Git Repository} &  \url{https://github.com/mweibel/ba}\tabularnewline
	\textit{Code: \gls{CI}} &  \url{https://travis-ci.org/mweibel/BA}\tabularnewline
	\textit{Thesis: Git Repository} & \url{https://github.com/mweibel/BA-Dokumentation}\tabularnewline
	\textit{Thesis: PDF} & \url{http://mweibel.github.com/BA-Dokumentation/thesis.pdf}\tabularnewline
	\textit{Thesis: \gls{CI}} & \url{https://travis-ci.org/mweibel/BA-Dokumentation}\tabularnewline
	\textit{Meeting Protokollierung} & \url{https://github.com/mweibel/BA-Dokumentation/wiki/Meetings}\tabularnewline
\tableend
\end{tabularx}
\caption{Projektrelevante URL's}
\end{table}

\subsection{Projektverwaltung}
Für die komplette Projektplanung, die Zeitrapportierung sowie das Issue-Management wird Redmine eingesetzt.

\subsection{Entwicklungsumgebung}
Zur Entwicklung von Quellcode-Artefakten steht eine mit Vagrant \cite{Vagrant} paketierte Virtual Machine bereit. Sie enthält alle notwendigen Abhängigkeiten und Einstellungen:

% FIXME bessere dokumentation der vagrant box

\begin{itemize}
	\item node.js 0.10.0
	\item PostgreSQL 9.1
	\item Ruby 2.0.0 (installiert via rvm)
	\item ZSH (inkl. oh-my-zsh)
\end{itemize}

Das Code Repository enthält ein \emph{Vagrantfile} welches durch den Befehl \emph{vagrant up} in der Kommandozeile automatisch das Image der vorbereiteten VM lokal verfügbar macht und startet.

\subsection{Git Repositories}
Sowohl Quellcodeartefakte als auch die in LaTeX formulierte Thesis (dieses Dokument) wird in auf GitHub abgelegten Git Repositories versioniert bzw. zentral gespeichert.



\section{Meetings}
\subsection{Regelmässiges Statusmeeting}
Während der gesamten Projektdauer findet jeweils am Mittwoch um 10 Uhr ein wöchentliches Statusmeeting statt. Die Sitzung wird abwechslungsweise jeweils von einer Person aus dem Projektteam geführt sowie von einer anderen protokolliert.

Das Projektteam stellt die Agenda der aktuellen Sitzung bis spätestens am vorangehenden Dienstag Abend bereit.