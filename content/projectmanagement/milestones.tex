\section{Meilensteine}
\begin{table}[H]
\tablestyle
\tablealtcolored
\begin{tabularx}{\textwidth}{l l l X}
\tableheadcolor
	\tablehead ID &
	\tablehead Meilenstein &
	\tablehead Termin &
	\tablehead Beschreibung \tabularnewline
\tablebody
	\textit{M1}\label{M1} & Ende Inception & 10.03.2013
		& Die Aufgabenstellung wurde gem. Auftrag klar definiert und die Projektinfrastruktur ist aufgesetzt. Eine initale Projektplanung besteht.\tabularnewline
	\textit{M2} & Ende Elaboration & 17.03.2013
		& Neben der Auswahl einer konkreten Technologie sind nun auch Guidelines definiert. Anforderungsdokumente sind erstellt und abgenommen. Zudem besteht ein initiales Software Architektur Dokument mit dazugehörigem Architekturprototypen.\tabularnewline
	\textit{M3} & Ende Construction 1 & 31.03.2013
		& Das Fundament der Applikation wurde implementiert. Weiter wurden die ersten Use Cases der Priorität \emph{Hoch} umgesetzt.\tabularnewline
	\textit{M4} & Ende Construction 2 & 14.04.2013
		& Alle Use Cases der Priorität \emph{Hoch} sind umgesetzt.\tabularnewline
	\textit{M5} & Ende Construction 3 & 28.04.2013
		& \tabularnewline
	\textit{M6} & Ende Construction 4 & 12.05.2013
		& \tabularnewline
	\textit{M7} & Ende Construction 5 & 26.05.2013
		& \tabularnewline
	\textit{M8} & Ende Transition & 02.06.2013
		&  Deploymentpakete und zuhgehörige Anleitungen sind bereit. Bugfixing abgeschlossen resp. ausstehnde Bugs dokumentiert.\tabularnewline
	\textit{M9} & Abgabe HSR Artefakte & 07.06.2013
		& Das A0-Poster sowie die Kurzfassung der Bachelorarbeit sind dem Betreuer zugestellt.\tabularnewline
	\textit{M10} & Abgabe Bachelorarbeit & 14.06.2013
		& Alle abzugebenden Artefakte sind dem Betreuer zugestellt worden.\tabularnewline
\tableend
\end{tabularx}
\caption{Meilensteine}
\end{table}