\section{Artefakte}
Dieser Abschnitt beschreibt alle Arbeitsprodukte (Artefakte), welche zwingend erstellt und abgegeben werden müssen.

Falls nicht anders vermerkt sind alle Artefakte Teil der Dokumentation.

\begin{table}[H]
\tablestyle
\tablealtcolored
\begin{tabularx}{\textwidth}{l l l X}
\tableheadcolor
	\tablehead ID &
	\tablehead Meilenstein &
	\tablehead Artefakt &
	\tablehead Beschreibung \tabularnewline
\tablebody
	\textit{A20} & M2 & Projektplanung
		& Projektablauf, Infrastrukturbeschreibung \& Phasenplanung\tabularnewline
	\textit{A21} & M2 & Aufgabenfindung
		& Bericht zu den Arbeiten während der Inception-Phase, Beschreibung wie zu verwendende Technologie gewählt wurde\tabularnewline
	\textit{A22} & M2 & Guidelines
		& Quellcode- und Testing-Guidelines\tabularnewline
	\textit{A23} & M2 & Anforderungsanalyse
		& Funktionale \& Nicht-funktionale Anforderungen, Use Cases\tabularnewline
	\textit{A24} & M2 & Domainmodel
		& Analyse der Problemdomäne\tabularnewline
	\textit{A25} & M2 & \gls{SAD}
		& Beschreibung der angestrebten Architektur für die Beispielapplikation.\tabularnewline
	\textit{A26} & M2 & Architekturprototyp
		& Exemplarische Implementierung der angestrebten Technologie/Architektur \emph{Typ: Quellcode/Applikation}\tabularnewline

	\textit{A80} & M8 & Quellcode Paket
		& Quellcode der Beispielapplikation zum eigenen, spezifischen Deployment. Bereits zur Weiterentwicklung. \emph{Typ: Quellcode}\tabularnewline
	\textit{A81} & M8 & Vagrant Paket
		& \gls{VM}-Image mit lauffähiger Version der Beispielapplikation. \emph{Typ: Vagrant Image}\tabularnewline
	\textit{A82} & M8 & Heroku Paket
		& Beispielapplikation ist so vorbereitet, dass ein Deployment auf Heroku problemlos möglich ist. \emph{Typ: Quellcode}\tabularnewline
	\textit{A83} & M8 & Installationsanleitung
		& Anleitung wie die verschiedenen Deployment-Pakete (Artefakte \emph{A80-82}) eingesetzt/installiert werden können.\tabularnewline

	\textit{A100} & M10 & A0-Poster
		& Gem. HSR Vorgaben zu erstellendes Poster mit Übersucht zu dieser Bachelorarbeit.\tabularnewline
	\textit{A101} & M10 & Kurzfassung
		& Gem. HSR Vorgaben zu erstellende Kurzfassung dieser Bachelorarbeit.\tabularnewline
	\textit{A102} & M10 & Dokumentation
		& Alle bisherigen Dokumentationsartefakte zusammengefasst in einem Bericht. Wo nötig, sind entsprechende Kapitel dem Projektablauf entsprechend nachgeführt (bspw. \emph{A25 SAD} etc.)\tabularnewline

\tableend
\end{tabularx}
\caption{Abzugebende Artefakte}
\end{table}