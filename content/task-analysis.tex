\chapter{Analyse der Aufgabenstellung}
\label{sec:analyse-der-aufgabenstellung}

Die Aufgabenstellung hat zum Ziel (siehe Anhang \ref{sec:aufgabenstellung}, ``\nameref{sec:aufgabenstellung}''), Architekturkonzepte anhand einer Beispielapplikation darzustellen und diese in das Modul Internettechnologien zu transferieren. Sie verzichtet dabei bewusst auf funktionale Anforderungen an die zu erstellende Applikation. Weiter werden auch keine spezifischen Technologien zur Umsetzung vorgegeben.

Diese sehr offene Ausgangssituation wird lediglich durch die folgenden Ansprüche eingegrenzt:

\begin{enumerate}
	\item Das Produkt soll unter Verwendung einer oder mehreren Internettechnologien konzipiert und umgesetzt werden.
	\item Der zu erstellende Quellcode soll \emph{State Of The Art} Architekturprinzipien (\cite{ROCA} \& \cite{TilkovSlides}) exemplarisch darstellen und der interessierten Fachperson als auch Studenten der Vorlesung \emph{Internettechnologien} als Anschauungsmaterial dienen können.
\end{enumerate}

Von diesen zwei Leitsätzen ausgehend kann angenommen werden, dass die Demonstration von Architekturprinzipien klar im Vordergrund stehen soll. Da diese Prinzipien aber auch einem lernenden Publikum (Punkt 2) so interessant wie möglich präsentiert werden sollen, ist eine attraktive Verpackung ebenfalls nicht zu vernachlässigen.

Dieses Kapitel beantwortet im weiteren Verlauf dementsprechend folgende Fragen genauer:

\begin{enumerate}
	\item Welche Schritte wurden im Bereich der Entwicklung der Applikationsidee (\nameref{sec:produktentwicklung-einleitung}) durchlaufen? Auf welche Aspekte wurde besonders eingegangen?
	\item Wie werden die vorgegebenen Architekturprinzipien am optimalsten auf eine Beispielapplikation abgebildet?
	\item Wie wurde die Technologie zur Umsetzung der generierten Produktidee ausgewählt und welche Kriterien waren dabei ausschlaggebend?
	\item Kann mit der Evaluierten Technologie ein verwendbarer Prototyp angefertigt und ein ``\gls{ProofOfConcept}'' erbracht werden?
\end{enumerate}

\newpage
\section{Architekturprinzipien}

Die Aufgabenstellung (Anhang \ref{sec:aufgabenstellung}) definiert zwei Dokumente mit Architekturprinzipien welche mittels einer Beispielapplikation demonstriert werden sollen:

\begin{itemize}
	\item \textit{Resource-oriented Client Architecture} kurz \textit{ROCA Principles} \cite{ROCA}\\
	Ein Satz von insgesamt 18 Richtlinien, sowohl für den Front- als auch Backendlayer.
	\item \textit{Building large web-based systems: 10 Recommendations} von Stefan Tilkov \cite{TilkovSlides}\\
	Präsentationsslides mit insgesamt 10 Empfehlungen welche teilweise layerübergreifend genutzt werden können.
\end{itemize}

Beide Quellen überschneiden sich in vielen Punkten. Dieser Abschnitt befasst sich mit der genaueren Analyse spezifischer Aspekte und zeigt auf, an welchen Stellen einer Beispielapplikation welche Prinzipien am einfachsten und effektivsten demonstriert werden können.

\subsection{Analyse: Resource-oriented Client Architecture}

\subsubsection*{Backend-Prinzipien}
\begin{table}[H]
\tablestyle
\tablealtcolored
\begin{tabularx}{\textwidth}{l l X}
\tableheadcolor
	\tablehead ID &
	\tablehead Prinzip &
	\tablehead Erläuterung\tabularnewline
\tablebody
	\textit{RP1} & REST &
	Kommunikation mit Serverressourcen folgt dem REST-Prinzip \cite{REST}
	\tabularnewline

	\textit{RP2} & Application Logic &
	Die Businesslogik der Applikation soll im Backend bleiben.
	\tabularnewline

	\textit{RP3} & HTTP &
	Ergänzend zu \emph{RP1} findet die Kommunikation mit Serverressourcen über wohldefinierte RESTful HTTP Requests \cite{HTTPRequest} statt 
	\tabularnewline

	\textit{RP4} & Link &
	Alle URI's weisen zu einer eindeutigen Ressource.
	\tabularnewline
	
	\textit{RP5} & Non-Browser &
	Die Serverkomponente kann ohne Browser resp. Frontendkomponente (z.B. mit \emph{wget} \cite{wget} oder \emph{curl} \cite{curl}) verwendet werden.
	\tabularnewline
	
	\textit{RP6} & Should-Formats &
	Serverressourcen können ihre Daten in verschiedenen Formaten (JSON, XML) ausliefern.
	\tabularnewline
	
	\textit{RP7} & Auth &
	\emph{HTTP Basic Authentication over SSL} \cite{HTTPBasicAuth} wird als grundlegender Sicherheitsmechanismus eingesetzt. Um ältere Clients abzudecken, können formularbasierte Logins in Verbindung mit Cookies eingesetzt werden. Cookies sollen dabei jegliche zustandsbezogene Informationen enthalten.
	\tabularnewline
	
	\textit{RP8} & Cookies &
	Cookies werden nur zur Authentifizierung oder zum Tracking des Benutzers verwendet.
	\tabularnewline
	
	\textit{RP9} & Session &
	Wo möglich soll auf Sessions verzichtet werden.
	\tabularnewline
\tableend
\end{tabularx}
\caption{Die ROCA Architekturprinzipien: Backend}
\end{table}

\subsubsection*{Frontend-Prinzipien}
\begin{table}[H]
\tablestyle
\tablealtcolored
\begin{tabularx}{\textwidth}{l l X}
\tableheadcolor
	\tablehead ID &
	\tablehead Prinzip &
	\tablehead Erläuterung\tabularnewline
\tablebody
	\textit{RP10} & Browser-Controls &
	Die Verwendung von Browser-Steuerelementen (Zurück, Aktualisieren usw.) müssen wie erwartet funktionieren und die Applikation nicht unerwartet beeinflussen.
	\tabularnewline
	
	\textit{RP11} & POSH &
	Vom Backend generiertes HTML ist semantisch korrekt \cite{SemanticHTML} und ist frei von Darstellungsinformationen
	\tabularnewline
	
	\textit{RP12} & Accessibility &
	Alle Ansichten können von Accessibility Tools (z.B. Screen Reader für Sehbehinderte) verarbeitet werden.
	\tabularnewline
	
	\textit{RP13} & Progressive Enhancement &
	Die Darstellung des Frontends soll auf aktuellsten Browsern top aussehen, aber auch auf älteren mit weniger Features verwendbar sein.
	\tabularnewline
	
	\textit{RP14} & Unobtrusive JavaScript &
	Die grundlegenden Funktionalitäten des Frontends müssen auch ohne JavaScript verwendbar sein.
	\tabularnewline
	
	\textit{RP15} & No Duplication &
	Eine Duplizierung von Businesslogik auf dem Frontend-Layer soll vermieden werden (vgl. \emph{RP2})
	\tabularnewline
	
	\textit{RP16} & Know Structure &
	Der Backendlayer soll so wenig wie möglich über die finale Struktur des HTML-Markups auf dem Frontend ``kennen''.
	\tabularnewline
	
	\textit{RP17} & Static Assets &
	Jeglicher JavaScript oder CSS Quellcode soll nicht dynamisch auf dem Backend generiert werden. Die Verwendung von Präprozessoren (SASS \cite{SASS}, LESS \cite{LESS} oder CoffeeScript \cite{CoffeeScript}) sind erlaubt und sollen sogar genutzt werden.
	\tabularnewline
	
	\textit{RP18} & History API &
	Von JavaScript ausgelöste Navigation soll über das HTML 5 History API \cite{HTML5HistoryAPI} im Browser abgebildet werden.
	\tabularnewline
\tableend
\end{tabularx}
\caption{Die ROCA Architekturprinzipien: Frontend}
\end{table}

\subsection{Analyse: Building large web-based systems: 10 Recommendations}

\begin{table}[H]
\tablestyle
\tablealtcolored
\begin{tabularx}{\textwidth}{l X}
\tableheadcolor
	\tablehead ID &
	\tablehead Empfehlung\tabularnewline
\tablebody
	\textit{TP1} & Aim for a web of looseley coupled, autonomous systems.
	\tabularnewline

	\textit{TP2} & Avoid session state wherever possible.
	\tabularnewline

	\textit{TP3} & Eat your own API dog food.
	\tabularnewline

	\textit{TP4} & Separate user identity, sign-up and self-care from product dependencies.
	\tabularnewline
	
	\textit{TP5} & Pick the low-hanging fruit of frond-end performance optimizations.
	\tabularnewline
	
	\textit{TP6} & Don't bother readers with write complexity.
	\tabularnewline
	
	\textit{TP7} & Apply the Web instead of working around it.
	\tabularnewline
	
	\textit{TP8} & Automate everything or you will be hurt.
	\tabularnewline
	
	\textit{TP9} & Know, design for \& use web components
	\tabularnewline
	
	\textit{TP10} & You can use new-fangled stuff, but you might not have to.
	\tabularnewline
\tableend
\end{tabularx}
\caption{Tilkovs Empfehlungen}
\end{table}

\newpage
\section{Produktentwicklung}
\label{sec:produktentwicklungeinleitung}
Die Findung einer passenden Produktidee gestaltete sich unter den im vorherigen Abschnitt erwähnten Bedingungen nicht unbedingt als einfach:

Zwar soll der grösste Teil des Arbeitsaufwandes in das Entwickeln einer beispielhaften Architektur fliessen, diese soll aber in einem für Studierende möglichst attraktiven Gewand präsentiert werden.

Der interessierte Leser findet im Anhang \ref{sec:produktentwicklung} ``\nameref{sec:produktentwicklung}'' weitere Details zum Prozess der konkreten Produktentwicklung. An dieser Stelle soll jedoch der Fokus auf der Grundlegenden Idee liegen.


\subsection*{Die Produktidee: Roomies}
\emph{Roomies} soll einer \gls{WG} ermöglichen, anfallende Aufgaben leicht unter den verschiedenen Bewohnern zu organisieren. Damit auch langweilige Ämtchen endlich erledigt werden, schafft Roomies durch ein Ranglisten- und Badgesystem (\gls{Gamification}) einen Anreiz seine Mitbewohner übertrumpfen zu wollen.

Durch das Aufgreifen einer Thematik aus dem Studentenalltag soll Roomies für Lernende aus dem Modul \emph{Internettechnologien} einen leichten Einstieg in die tendenziell trockene Materie der Softwarearchitektur bieten.

\subsubsection*{Logo \& Branding}
\begin{figure}[H]
	\centering
	\includegraphics[width=5cm]{content/images/roomies-withshadow.png}
	\caption{Roomies Logo im College Stil}
\end{figure}

Im Anhang \ref{sec:produktentwicklung} sind zum Thema \emph{\nameref{sec:appendix-branding}} Informationen wie Styleguide, Logovarianten etc. einsehbar.

\newpage
\section{Technologieevaluation}
Das Thema ``Architekturkonzepte moderner Web-Applikationen'' legt den Schluss nahe, nicht nur aktuellste Architekturprinzipien bei der Umsetzung der Beispielapplikation zu verwenden, sondern auch im Bereich der Technologiewahl auf etablierte Platzhirsche wie Java oder C\# (in Verbindung mit deren Web-Frameworks) zu verzichten.

Unter Berücksichtigung der persönlichen Erfahrungen und Einschätzungen aller Projektteilnehmer wurde in Vereinbarung mit dem Betreuer im Zuge einer Evaluation eine Shortlist mit folgenden Technologiekandidaten zusammengestellt:

\begin{itemize}
	\item \emph{Java}\\
	Trotz des einführenden Statements, Platzhirsche von einer engeren Auswahl auszuschliessen, war das Projektteam aufgrund der vorangegangenen Studienarbeit davon überzeugt, dass Java, insbesondere als Backendtechnologie, mit den ``jungen Wilden'' problemlos mithalten kann.

	\item \emph{JavaScript}\\
	Während den letzten zwei Jahren erlebte JavaScript eine Renaissance: Mit node.js schaffte es den Sprung vom Frontend-Layer ins Backend und erfreut sich in der OpenSource als auch der Industrie-Community grösster Beliebtheit.

	\item \emph{Ruby}\\
	Ruby hat sich in der näheren Vergangenheit zusammen mit Ruby On Rails im Markt etablieren können. Als relativ junge Technologie durfte es aus diesem Grund bei einer Evaluation nicht ignoriert werden.
\end{itemize}

In diesem Abschnitt werden übergreifende Bewertungskriterien definiert, welche anschliessend auf alle drei Technologiekandidaten, resp. deren Frameworks angwendet werden können.

\newpage
\subsection{Bewertungskriterien}
Die folgende Tabelle definert sechs Kriterien, welche zur Bewertung einer Technologie oder eines Frameworks jeweils mit 0-3 Sternen bewertet werden können.

Die Spalte \emph{Gewichtung} gibt an, als wie wichtig das betreffende Kriterium im Bezug auf die Aufgabenstellung anzusehen ist.

\begin{table}[H]
\tablestyle
\tablealtcolored
\begin{tabularx}{\textwidth}{l l X c}
\tableheadcolor
	\tablehead ID &
	\tablehead Kriterium &
	\tablehead Erläuterung &
	\tablehead Gewichtung \tabularnewline
\tablebody
\textit{TK1} &
	Eigenkonzepte &
	Wieviele eigene Konzepte \& Ideen bringt eine Technologie resp. ein Framework mit? Viele spezifische Konzepte bedeuten meistens eine steile Lernkurve für Neueinsteiger. \emph{Hohe Bewertung = Wenig Eigenkonzepte}&
	\faStar\faStar\faStar \tabularnewline
\textit{TK2} &
	Eignung &
	Wie gut eignet sich eine Technologie oder ein Framework für die Demonstration der Architekturrichtlinien? Geschieht alles ``hinter'' dem Vorhang oder sind einzelne Komponenten einsehbar? \emph{Hohe Bewertung = Hohe Eignung}&
	\faStar\faStar\faStar \tabularnewline
\textit{TK3} &
	Produktreife &
	Wie gut hat sich das Framework oder die Technologie bis jetzt in der Realität beweisen können? Wie lange existiert es schon? Gibt es eine aktive Community und wird es aktiv weiterentwickelt? \emph{Hohe Bewertung = Hohe Produktreife}&
	\faStar\faStar\faStar\tabularnewline
\textit{TK4} &
	Aktualität &
	Diese Arbeit kümmert sich um ``moderne Web-Applikationen''. So sollte auch die zu verwendende Technologie gewissermassen nicht von ``vorgestern'' sein. \emph{Hohe Bewertung = Hohe Aktualität}&
	\faStar \tabularnewline
\textit{TK5} &
	``Ease of use'' &
	Wie angenehm ist das initiale Erstellen, die Konfiguration und die Unterhaltung einer Applikation? Führt das Framework irgendwelchen ``syntactic sugar'' \cite{syntacticsugar} ein um die Arbeit zu erleichtern? \emph{Hohe Bewertung = Hoher ``Ease of use''-Faktor} &
	\faStar\faStar \tabularnewline
\textit{TK6} &
	Testbarkeit &
	Wie gut können die mit dem Framework oder der Technologie erstellte Komponenten durch Unit Tests getestet werden? \emph{Hohe Bewertung = Hohe Testbarkeit} &
	\faStar \tabularnewline
\tableend
\end{tabularx}
\caption{Bewertungskriterien für Technologieevaluation}
\label{tab:bewertungskriterien}
\end{table}

\subsection{Java}

Schon vor dieser Bachelorarbeit kann das Projektteam diverse Erfahrungen mit Java vorweisen. Zum Einen aus privaten und beruflichen Projekten, zum Anderen auch ganz themenspezifisch aus der Studienarbeit, welche ein Semester früher durchgeführt wurde.

Als Teil einer grösseren Applikation wurde dort ein Webservice mit REST-Schnittstelle umgesetzt. Zum Einsatz kamen diverse Referenzimplementierungen von Java Standard API's. Die sehr positiven Erfahrungen mit der dort orchestrierten Zusammenstellung von Bibliotheken legen den Schluss nahe, diese auch für eine potentielle Verwendung innerhalb dieser Bachelorarbeit wiederzuverwenden.

Der Studienarbeit-erprobten Kombination sollen jedoch auch andere Alternativen gegenübergestellt werden. Insgesamt ergeben sich so folgende Analysekandidaten im Bereich der Technologie \emph{Java}:

\begin{table}[H]
\tablestyle
\tablealtcolored
\begin{tabularx}{\textwidth}{l X l}
\tableheadcolor
	\tablehead Framework &
	\tablehead Erläuterung \tabularnewline
\tablebody
\textit{Studienarbeit-Zusammenstellung} &
	Die Zusammenstellung von \emph{Google Guice}, \emph{Jersey}, \emph{Codehaus Jackson} sowie \emph{EclipseLink} hat sehr gut harmoniert. Die Verwendung von einem Java-fremden Framework für die Implementierung des Frontends wäre jedoch erneut abzuklären.
	\tabularnewline
\textit{Spring} &
	Spring hat sich in den letzten Jahren in der Industrie etablieren können. Es bietet eine Vielzahl von Subkomponenten (MVC, Beanmapping etc.).
	\tabularnewline
\textit{Plain JEE} &
	Java Enterprise bietet von sich aus viele Features, welche die Frameworks von Dritten unter anderen Ansätzen umsetzen. Es gilt jedoch abzuwägen, wie gross der Aufwand ist, um beispielsweise eine REST-Serviceschnittstelle zu implementieren.
	\tabularnewline
\textit{Vaadin} &
	Vaadin baut auf Googles GWT und erlaubt die serverlastige Entwicklung von Webapplikationen.
	\tabularnewline
\textit{Play! Framework} &
	Seit dem Release der Version 2.0 im Frühjar 2012 erfreut sich das Play! Frameworks grosser Beliebtheit. Insbesondere die integrierten Scaffolding-Funktionalitäten und MVC-Ansätze werden gelobt.
	\tabularnewline
\tableend
\end{tabularx}
\caption{Shortlist Analysekandidaten Java}
\end{table}


\subsubsection*{Bewertungsmatrix}

\begin{table}[H]
\newcolumntype{s}{>{\centering\hsize=0.15\hsize}X}
\tablestyle
\tablealtcolored
\begin{tabularx}{\textwidth}{X s s s s s s s}

\tableheadcolor
	\tablehead &
	\rotatebox{90}{\bfseries\textit{TK1 Eigenkonzepte} } &
	\rotatebox{90}{\bfseries\textit{TK2 Eignung}} &
	\rotatebox{90}{\bfseries\textit{TK3 Produktreife}} &
	\rotatebox{90}{\bfseries\textit{TK4 Aktualität}} &
	\rotatebox{90}{\bfseries\textit{TK5 ``Ease of use''}} &
	\rotatebox{90}{\bfseries\textit{TK6 Testbarkeit}} &
	\rotatebox{90}{\bfseries\textit{Gesamtbewertung}}
	\tabularnewline
\tablebody
	\textit{Studienarbeit-Zusammenstellung}	&
	\threeStars &
	\threeStars &
		&
		&
	\threeStars &
	\twoStars &
	\directlua{
		tex.print(math.round(
			(3 * 3 +
			3 * 3 +
			0 * 3 +
			0 * 1 +
			3 * 2 +
			2 * 2) / 6
		))
	}
	\tabularnewline


	\textit{Spring} &
		&
		&
	\twoStars &
	\threeStars &
		&
	\oneStar &
	\directlua{
		tex.print(math.round(
			(0 * 3 +
			0 * 3 +
			2 * 3 +
			3 * 1 +
			0 * 2 +
			1 * 2) / 6
		))
	}
	\tabularnewline


	\textit{Plain JEE} &
		&
	\twoStars &
	\threeStars &
	\threeStars &
		&
	\threeStars &
	\directlua{
		tex.print(math.round(
			(0 * 3 +
			2 * 3 +
			3 * 3 +
			3 * 1 +
			0 * 2 +
			3 * 2) / 6
		))
	}
	\tabularnewline


	\textit{Vaadin} &
	\oneStar &
		&
	\threeStars &
	\twoStars &
		&
	\threeStars &
	\directlua{
		tex.print(math.round(
			(1 * 3 +
			0 * 3 +
			3 * 3 +
			2 * 1 +
			0 * 2 +
			3 * 2) / 6
		))
	}
	\tabularnewline


	\textit{Play! Framework} &
	\oneStar &
	&
	\twoStars &
	\twoStars &
	&
	\threeStars&
	\directlua{
		tex.print(math.round(
			(1 * 3 +
			0 * 3 +
			2 * 3 +
			2 * 1 +
			0 * 2 +
			3 * 2) / 6
		))
	}
	\tabularnewline
\tableend
\end{tabularx}
\caption{Bewertungsmatrix Java Frameworks}
\end{table}

\subsubsection*{Interpretation}
\emph{Plain JEE}, \emph{Vaadin} und \emph{Play! Framework} spielen ihre Stärken klar in der Produktreife und der dadurch hohen Wartbarkeit resp. Testbarkeit aus. Im Bezug auf die Eigenkonzepte benötigen alle Kandidaten einen gewissen initialen Lernaufwand. \emph{Studienarbeit-Zusammenstellung} arbeitet mit einem klar zugänglichen Schichtenmodell und verwendet über dies hinaus ein komplett vom Backend entkoppeltes Frontend. Zwar wäre eine solche Lösung auch mit \emph{Spring} und \emph{Plain JEE} möglich, jedoch versagen diese beiden Frameworks wiederum im Bezug auf die Eignung, die aufgestellten Architekturrichtlinien transparent demonstrieren zu können.

Die Produktreife von \emph{Studienarbeit-Zusammenstellung} ist zu vernachlässigen. Die einzelnen Komponenten für sich haben sich bereits in länger in der Praxis bewähren können und sind lediglich genau in dieser Kombination evtl. weniger oft erprobt.

Aufgrund der vorangegangenen Bewertung soll für Java die \emph{Studienarbeit-Zusammenstellung} mit den Frameworks der beiden anderen Technologien verglichen werden.

\subsection{JavaScript}
\label{sec:technology-evaluation-javascript}

\subsubsection*{Bewertungsmatrix}

\begin{table}[H]
\newcolumntype{s}{>{\centering\hsize=0.15\hsize}X}
\tablestyle
\tablealtcolored
\begin{tabularx}{\textwidth}{X s s s s s s s}
\tableheadcolor
	\tablehead &
	\rotatebox{90}{\bfseries\textit{TK1 Eigenkonzepte} } &
	\rotatebox{90}{\bfseries\textit{TK2 Eignung}} &
	\rotatebox{90}{\bfseries\textit{TK3 Produktreife}} &
	\rotatebox{90}{\bfseries\textit{TK4 Aktualität}} &
	\rotatebox{90}{\bfseries\textit{TK5 ``Ease of use''}} &
	\rotatebox{90}{\bfseries\textit{TK6 Testbarkeit}} &
	\rotatebox{90}{\bfseries\textit{Total}}
	\tabularnewline
\tablebody
	\textit{Express.js}	&
	\threeStars &
	\threeStars	&
	\twoStars &
	\threeStars &
	\twoStars &
	\threeStars &
	\directlua{
		tex.print(math.round(
			(3 * 3 +
			3 * 3 +
			2 * 3 +
			3 * 1 +
			2 * 2 +
			3 * 1) / 6
		))
	}
	\tabularnewline

	\textit{Tower.js} &
	\twoStars &
	\twoStars &
	\oneStar &
	\threeStars &
	\threeStars	&
		&
	\directlua{
		tex.print(math.round(
			(2 * 3 +
			2 * 3 +
			1 * 3 +
			3 * 1 +
			3 * 2 +
			0 * 1) / 6
		))
	}
	\tabularnewline


	\textit{derby} &
	\twoStars &
	\oneStar &
	\oneStar &
	\threeStars &
	\twoStars &
		&
	\directlua{
		tex.print(math.round(
			(2 * 3 +
			1 * 3 +
			1 * 3 +
			3 * 1 +
			2 * 2 +
			0 * 1) / 6
		))
	}
	\tabularnewline


	\textit{Geddy} &
	\threeStars &
	\oneStar &
	\twoStars &
	\twoStars &
	\threeStars &
	&
	\directlua{
		tex.print(math.round(
			(3 * 3 +
			1 * 3 +
			2 * 3 +
			2 * 1 +
			3 * 2 +
			0 * 1) / 6
		))
	}
	\tabularnewline


	\textit{Sails} &
	\twoStars &
	\twoStars &
	\oneStar &
	\threeStars &
	\twoStars &
	\oneStar &
	\directlua{
		tex.print(math.round(
			(2 * 3 +
			2 * 3 +
			1 * 3 +
			3 * 1 +
			2 * 2 +
			1 * 1) / 6
		))
	}
	\tabularnewline
\tableend
\end{tabularx}
\caption{Bewertungsmatrix JavaScript Frameworks}
\end{table}


\subsection{Ruby}

Insbesondere mit dem Framework \emph{Ruby on Rails} \cite{RubyOnRails} wurde Ruby für die Entwicklung von Umfangreichen Webapplikationen seit Veröffentlichung in den 90ern immer beliebter. Mit fast kindlicher Selbstverständlichkeit bringt Ruby viele Konzepte wie \gls{Multiple Inheritance} (in Form von Mixins) oder die funktionale Behandlung von jeglichen Werten/Objekten von Haus aus mit.

Für den Einsteiger etwas verwirrend setzt es zudem auf eine für den Menschen ``leserliche'' Syntax als beispielsweise von Java oder anderen verwandten Sprachen gewohnt. Folgende Codebeispiele bewirken die selbe Ausgabe auf der Kommandozeile, unterscheiden sich aber deutlich in ihrer Formulierung:

\begin{lstlisting}[language=Java, caption=Negierte if-Abfrage in Java]
if(!enabled) {
	System.out.println("Ich bin deaktiviert!");
}
\end{lstlisting}

\begin{lstlisting}[language=Ruby, caption=Negierte if-Abfrage in Ruby]
puts "Ich bin deaktiviert!" unless enabled
\end{lstlisting}

Während der kurzen Technologieevaluationsphase wurde im Bereich Ruby das Hauptaugenmerk auf \emph{Ruby on Rails} gelegt. Insbesondere die \gls{Scaffolding}tools und der von denen generierte Quellcode wurde näher begutachtet. Die Resultate sind wiederum auf die sechs Bewertungskriterien appliziert worden.

\subsubsection*{Bewertung Ruby on Rails}
\begin{table}[H]
\newcolumntype{s}{>{\centering\hsize=0.15\hsize}X}
\tablestyle
\tablealtcolored
\begin{tabularx}{\textwidth}{X s s s s s s s}
\tableheadcolor
	\tablehead &
	\rotatebox{90}{\bfseries\textit{TK1 Eigenkonzepte} } &
	\rotatebox{90}{\bfseries\textit{TK2 Eignung}} &
	\rotatebox{90}{\bfseries\textit{TK3 Produktreife}} &
	\rotatebox{90}{\bfseries\textit{TK4 Aktualität}} &
	\rotatebox{90}{\bfseries\textit{TK5 ``Ease of use''}} &
	\rotatebox{90}{\bfseries\textit{TK6 Testbarkeit}} &
	\rotatebox{90}{\bfseries\textit{Gesamtbewertung}}
	\tabularnewline
\tablebody
	\textit{Ruby on Rails} &
	\oneStar &
	\oneStar &
	\threeStars &
	\oneStar &
	\threeStars &
	\twoStars &
	\directlua{
		tex.print(math.round(
			(1 * 3 +
			1 * 3 +
			3 * 3 +
			1 * 1 +
			3 * 2 +
			2 * 2) / 6
		))
	}
	\tabularnewline
\tableend
\end{tabularx}
\caption{Bewertung Ruby on Rails}
\end{table}


\subsubsection*{Interpretation}
Nach genauerem Befassen mit Ruby on Rails sind die Einschätzung des Projektteams gespalten.

Zum Einen minimiert Ruby on Rails den Aufwand für das Erledigen von Routineaufgaben extrem (\gls{Scaffolding}). Der generierte Code ist sofort verwendbar und, gute Ruby-Kenntnisse vorausgesetzt, gut erweiterbar.

Zum Anderen ist aber gerade die Einfachheit, wie bspw. Controllers oder Models erzeugt und in den Applikationsablauf eingebunden werden, alles Andere als optimal wenn es darum geht, Architekturrichtlinien eindeutig und klar demonstrieren zu können.

Unter dem Vorbehalt, dass Ruby on Rails für die Demonstration der definierten Architekturrichtlinien evtl. nicht die richtige Wahl sein könnte, kann das Projektteam nur eine bedingte Empfehlung für das Ruby Framework abgeben.

\newpage
\section{Proof of Concept}
\label{sec:proof-of-concept}

In der Technologieevaluation zu \nameref{sec:technology-evaluation-javascript} stachen die beiden Frameworks Express.js \cite{Expressjs}
und Sails.js \cite{sails} heraus.

Obwohl Express.js stabiler ist und weitaus mehr genutzt wird, ist Sails.js aufgrund der neuen Ideen und interessanten Ansätzen für einen ersten Prototyp ausgewählt worden.

\subsection{Prototyp A: Sails.js}
Sails.js verwendet \gls{Scaffolding} um einerseits ein neues Projekt zu erstellen, andererseits  um verschiedene Komponenten wie Models oder Controllers zu erstellen. Wie im \nameref{sec:erdiagramm} (siehe Kapitel \ref{sec:sad}, Abschnitt \ref{sec:erdiagramm}) beschrieben, werden u.a. ein Task und ein Resident Model (mit entsprechenden Tabellen) benötigt.

Um das \gls{ORM} ``Waterline'' \cite{Waterline} zu testen, wurden diese beiden Models implementiert. Das Task-Model sieht so aus wie im Quelltext \ref{lst:sailsTaskModel} gezeigt.

\begin{lstlisting}[language=JavaScript, caption=Task Model in Sails.js, label=lst:sailsTaskModel]
var Task = {
	attributes: {
		name: "string"
		, description: "string"
		, points: "int"
		, userId: "int"
		, communityId: "int"
	}
};
module.exports = Task;
\end{lstlisting}

Mit der Definition eines Models wird automatisch eine \gls{REST}-API für dieses erstellt.

Damit lassen sich einerseits CRUD-Operationen direkt über HTTP ausführen, andererseits existiert auch die Möglichkeit, Models direkt über einen offenen \gls{Websocket} zu verwenden.

Dieses Feature macht Sails.js sehr nützlich für \gls{Realtime}-Applikationen.

Um eine effektive HTML-Seite darstellen zu können wird ein Controller sowie eine View benötigt. Dies wurde im Prototyp für Aufgaben (Tasks) implementiert.

\newpage
\begin{lstlisting}[language=JavaScript, caption=Task Controller in Sails.js, label=lst:sailsjstaskcontroller, escapeinside={@}{@}]
function findTaskAndUser(id, next) {
	Task.find(id).done(function(err, task) {
		@\label{lst:sailsjstaskcontroller_userfind}@User.find(task.userId).done(function(err, user) {
			next(task, user);
		});
	});
}

function renderResponse(req, res, response) {
	if (req.acceptJson) {
		res.json(response);
	} else if(req.isAjax && req.param('partial')) {
		response['layout'] = false;
		res.view(response);
	} else {
		res.view(response);
	}
}

var TaskController = {
	get: function(req, res) {
		@\label{lst:sailsjstaskcontroller_idparam}@var id = req.param('id');
		findTaskAndUser(id, function(task, user) {
			var response = {
				'task': task,
				'user': user,
				'title': task.name
			};
			renderResponse(req, res, response);
		});
	}
};
module.exports = TaskController;
\end{lstlisting}

In diesem Controller wird ein Task aufgrund des GET-Parameters ``id'' (Zeile \ref{lst:sailsjstaskcontroller_idparam}) geladen. Das Code-Stück zeigt die grosse Schwäche des ORMs Waterline \cite{Waterline}. In anderem ORMs könnte man auf dem ``task''-Objekt direkt ``.user'' aufrufen. Dies geht bei Waterline nicht und man muss den Umweg über ``User.find()'' (Zeile \ref{lst:sailsjstaskcontroller_userfind}) gehen.

Bei einem ausgereiften ORM würden solche Methoden wegen der Definition von Relationen direkt zur Verfügung stehen.

Der Controller verwendet das folgende Template, um HTML-Markup zu rendern:

\begin{lstlisting}[language=HTML, caption=Task Template]
<div id="task-display">
	<h1>Task: <%= task.name %></h1>
	<ul>
		<li>Points: <%= task.points %></li>
		<li>Created At: <%= task.createdAt %></li>
	</ul>
	<h2>User: <%= user.name %></h2>
	<ul>
		<li>Created At: <%= user.createdAt %></li>
	</ul>
</div>
<a href="#" id="reload">Reload!</a>
<script>
	$('#reload').on('click', function() {
		$.ajax('/task/get/?id=<%= task.id %>&partial=true', {
			success: function(response) {
				var $response = $('<div class="body-mock">' + response + '</div>');
				html = $response.find('#task-display');
				$('#task-display').replaceWith(html);
			}
		});
	});
</script>
\end{lstlisting}

Mit dem Skript am Ende des Task Templates wird aufgezeigt, wie man ohne Neuladen der Seite direkt das DOM ersetzen kann. Dies ist aber Framework-unabhängig.

\subsubsection*{Schlussfolgerung}

Wie bereits angemerkt, ist das ORM von Sails.js nicht ausgereift. Weder Assoziationen zwischen Models \cite{SailsjsModelAssociations} noch das setzen von Indizes ist möglich.

Für das Resident-Model ist es u.A. auch nötig, Facebook IDs zu speichern. Diese sind 64 Bit gross und wegen mangelhafter Unterstützung des ORMs wäre das gar nicht möglich.

Nebst dem ORM ist auch das Framework und die zugehörige Dokumentation wenig umfangreich. Die Community war zum Zeitpunkt der Evaluation (siehe Anhang \ref{sec:appendix-technology-evaluation} zur \nameref{sec:appendix-technology-evaluation}) sehr klein. Die Fakten deuten darauf hin, dass \emph{Sails.js} für die konkrete Beispielapplikation eher ungeeignet ist.
\subsection{Prototyp B: Express.js}

Express.js \cite{Expressjs} ist ein leichtgewichtiges Framework, welches mittels Connect-Middlewares \cite{connect} erweitert werden kann.

Der initiale Startpunkt des Express.js Prototyps ist die Datei ``app.js'' \cite{ExpressjsPrototypAppjs}. Dort werden alle benutzten Middlewares registriert, die Datenbank aufgesetzt und Controller registriert.

Ein Beispielhafter Controller ist im Quelltext \ref{lst:controllerInExpressjs} zu sehen.

\begin{lstlisting}[language=JavaScript, caption=Beispiel eines Controllers in Express.js, label=lst:controllerInExpressjs]
exports.index = function(req, res){
	// first Parameter: Template File to use
	// 2nd Parameter: Context to pass to the template
	res.render('index', { title: 'Express' });
};
\end{lstlisting}

Ein zugehöriges Template kann folgendermassen aussehen:

\begin{lstlisting}[language=HTML, caption=Template in Express.js, label=lst:templateInExpressjs]
<!DOCTYPE html>
<html>
	<head>
		<title><%= title %></title>
		<link rel='stylesheet' href='/stylesheets/style.css' />
		<%- LRScript %>
	</head>
	<body>
		<h1><%= title %></h1>
		<p>Welcome to <%= title %></p>
	</body>
</html>
\end{lstlisting}

In den vorangegangenen zwei Quelltexten \ref{lst:controllerInExpressjs} und \ref{lst:templateInExpressjs} ist ersichtlich, dass der Applikationsentwickler sehr grosse Kontrolle über Express.js hat.

Die Flexibilität von Express.js bietet sowohl Vor- als auch Nachteile für die Erstellung von Webapplikationen. Im Bezug auf die Veranschaulichung der \nameref{sec:architekturrichtlinien} ist es jedoch ein grosser Vorteil, da wenig Logik fix in Express.js eingebaut ist.
\subsubsection{\gls{ORM}}
Im Gegensatz zu den anderen evaluierten Frameworks ist bei Express.js kein ORM enthalten. Deswegen musste auch bzgl. des ORMs eine kurze Evaluation gemacht werden.
Tabelle \ref{tab:bewertungskriterienORM} zeigt die Kriterien und Gewichtung für die Evaluation des ORMs.

\begin{table}[H]
\tablestyle
\tablealtcolored
\begin{tabularx}{\textwidth}{l l X c}
\tableheadcolor
	\tablehead ID &
	\tablehead Kriterium &
	\tablehead Erläuterung &
	\tablehead Gewichtung \tabularnewline
\tablebody
\textit{OK1} &
	Unterstützung DBs &
	Wieviele unterschiedliche Datenbanken unterstützt das ORM? Werden auch \gls{NoSQL}-Datenbanken unterstützt? \emph{Hohe Bewertung = Grosse Anzahl an Datenbanken}&
	\faStar \tabularnewline
\textit{OK2} &
	Relationen &
	Sind Relationen definierbar zwischen Tabellen? Verwenden diese die Datenbank-spezifischen Foreign Keys dafür (falls möglich)? \emph{Hohe Bewertung = Relationen möglich und verwendet Datenbank-spezifische Datentypen}&
	\faStar\faStar\faStar \tabularnewline
\textit{OK3} &
	Produktreife &
	Wie gut hat sich das ORM bis jetzt in der Realität beweisen können? Wie lange existiert es schon? Gibt es eine aktive Community und wird es aktiv weiterentwickelt? \emph{Hohe Bewertung = Hohe Produktreife}&
	\faStar\faStar\faStar\tabularnewline
\textit{OK4} &
	``Ease of use'' &
	Wie angenehm ist das initiale Erstellen, die Konfiguration und die Unterhaltung von Models? Führt das ORM irgendwelchen ``syntactic sugar'' \cite{syntacticsugar} ein um die Arbeit zu erleichtern? \emph{Hohe Bewertung = Hoher ``Ease of use''-Faktor} &
	\faStar \tabularnewline
\textit{OK5} &
	Testbarkeit &
	Wie gut können die mit dem Framework oder der Technologie erstellte Komponenten durch Unit Tests getestet werden? \emph{Hohe Bewertung = Hohe Testbarkeit} &
	\faStar\faStar \tabularnewline
\tableend
\end{tabularx}
\caption{Bewertungskriterien für ORM-Evaluation}
\label{tab:bewertungskriterienORM}
\end{table}


\begin{table}[H]
\newcolumntype{s}{>{\centering\hsize=0.15\hsize}X}
\tablestyle
\tablealtcolored
\begin{tabularx}{\textwidth}{X s s s s s s}
\tableheadcolor
	\tablehead &
	\rotatebox{90}{\bfseries\textit{OK1 Unterstützung DBs} } &
	\rotatebox{90}{\bfseries\textit{OK2 Relationen}} &
	\rotatebox{90}{\bfseries\textit{OK3 Produktreife}} &
	\rotatebox{90}{\bfseries\textit{OK4 ``Ease of use''}} &
	\rotatebox{90}{\bfseries\textit{OK5 Testbarkeit}} &
	\rotatebox{90}{\bfseries\textit{Total}}
	\tabularnewline
\tablebody
	\textit{JugglingDB} &
	\threeStars &
	\oneStar &
	\oneStar &
	\twoStars &
	\twoStars &
	\directlua{
		tex.print(math.round(
			(3 * 1 +
			1 * 3 +
			1 * 3 +
			2 * 1 +
			2 * 2) / 5
		))
	}
	\tabularnewline

	\textit{Node-ORM2} &
	\twoStars &
	\twoStars	&
	\oneStar &
	\threeStars &
	\oneStar &
	\directlua{
		tex.print(math.round(
			(2 * 1 +
			2 * 3 +
			1 * 3 +
			3 * 1 +
			1 * 2) / 5
		))
	}
	\tabularnewline

	\textit{Sequelize} &
	\oneStar &
	\twoStars &
	\twoStars &
	\twoStars &
	\oneStar &
	\directlua{
		tex.print(math.round(
			(1 * 1 +
			2 * 3 +
			2 * 3 +
			2 * 1 +
			1 * 2) / 5
		))
	}
	\tabularnewline
\tableend
\end{tabularx}
\caption{Bewertungsmatrix JavaScript ORMs}
\label{tab:bewertungsmatrixORM}
\end{table}

Alle verglichenen ORMs haben eine ähnliche Gesamtbewertung. Bei ``Sequelize'' stechen jedoch die Produktreife und die Unterstützung für Relationen heraus.

Diese zwei Gründe zusammen mit der Roadmap \cite{RoadmapSequelize} von Sequelize haben schliesslich zur Überzeugung geführt, dass Sequelize die richtige Wahl ist.
