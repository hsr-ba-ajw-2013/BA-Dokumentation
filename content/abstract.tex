\begin{abstract}
Der Internetbrowser ist heute mehr denn je das Zuhause für komplexe Webapplikationen. Durch die Leistungssteigerung und immer ausgefeilteren Möglichkeiten nutzen ihn heutzutage nicht mehr nur Unternehmen als Schnittstelle zwischen Angestellten und Anwendungen.

Erwartungsgemäss können traditionelle Architekturstile mit der neuen Technologie nicht immer Schritt halten und Lösungen für wiederkehrende Probleme bieten. Diese Bachelorarbeit analysiert Trends auf dem Gebiet der Webapplikationen kritisch und liefert gleichzeitig zukunftsweisende Vorschläge für das Unterrichtsmodul ``Internettechnologien''.

In einer Vorstudie wurden drei grundlegende Fragen geklärt: Welche Softwarearchitekturkonzepte aus der Aufgabenstellung sollen bearbeitet werden? Was für eine Applikation kann die ausgewählten Konzepte optimal veranschaulichen? Und Mit welcher Technologie sollen die ausgewählten Konzepte demonstriert werden?

Nach der ausführlichen Evaluation des Konzeptkatalogs und der Auswahl von 22 Konzepten daraus entschied sich das Projektteam eine vergleichsweise simple Aufgabenverwaltung für Wohngemeinschaften mit dem Namen ``Roomies'' umzusetzen. Damit wird sichergestellt, dass der Fokus auf die Demonstration der Konzepte gelegt werden kann. Als wichtigen Schritt Richtung ``Bleeding Edge'' wurde zudem entschieden, sowohl server- als auch clientseitig auf JavaScript zu setzen.

Die abschliessende Studie bietet einen tiefen Einblick in die Entwicklung einer JavaScript-basierten Client-Server-Webapplikation. Ausführliche Analysen bewerten die untersuchten Konzepte kritisch. Ergänzend bietet der Quelltext von ``Roomies'' anschauliche Beispiele zu jedem untersuchten Konzept.

Besondere Aufmerksamkeit wurde dem Konzept ``Unobtrusive JavaScript'' zuteil. Basierend auf ``Backbone.js'' wurde eine eigene quelloffene Bibliothek namens ``barefoot'' entwickelt, welche ohne grösseren Mehraufwand den gleichen Quelltext sowohl im Browser des Endbenutzers als auch auf der Serverkomponente lauffähig macht.
\end{abstract}