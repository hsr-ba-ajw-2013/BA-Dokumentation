\clearpage
\thispagestyle{empty}

\section*{Abstract}

Der Internetbrowser ist heute mehr denn je das Zuhause für komplexe Webapplikationen.

Erwartungsgemäss können traditionelle Architekturstile mit der neuen Technologie nicht immer Lösungen für wiederkehrende Probleme bieten. Diese Bachelorarbeit analysiert Trends auf dem Gebiet der Webapplikationen und liefert gleichzeitig zukunftsweisende Vorschläge für das Unterrichtsmodul ``Internettechnologien''.

In einer Vorstudie wurden drei grundlegende Fragen geklärt: Welche Architekturstile aus der Aufgabenstellung sollen bearbeitet werden? Was für eine Applikation kann die ausgewählten Stile optimal veranschaulichen? Und mit welcher Technologie sollen die ausgewählten Stile demonstriert werden?

Nach der ausführlichen Evaluation des Architekturstilkatalogs und der Auswahl von 22 Konzepten entschied sich das Projektteam eine Aufgabenverwaltung für Wohngemeinschaften mit dem Namen ``Roomies'' umzusetzen. Der Fokus liegt jedoch auf der Demonstration der Konzepte. Als Technologie wird server- als auch clientseitig JavaScript eingesetzt.

Die abschliessende Studie bietet einen tiefen Einblick in die Entwicklung einer JavaScript-basierten Client-Server-Webapplikation. Ausführliche Analysen bewerten die untersuchten Konzepte kritisch. Ergänzend bietet der Quelltext von ``Roomies'' anschauliche Beispiele zu jedem untersuchten Konzept.

Besondere Aufmerksamkeit wurde dem Konzept ``Unobtrusive JavaScript'' zuteil. Basierend auf ``Backbone.js'' wurde ein eigenes quelloffenes Framework namens ``barefoot'' entwickelt, welches ohne grösseren Mehraufwand den gleichen Quelltext sowohl im Browser des Endbenutzers als auch auf der Serverkomponente lauffähig macht.

\clearpage