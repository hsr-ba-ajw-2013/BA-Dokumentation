\chapter{Richtliniendemonstration}
\label{sec:principle-demonstration}

\section*{Einleitung}

In Kapitel \ref{sec:analyse-der-aufgabenstellung}, Abschnitt ``\nameref{sec:architekturrichtlinien}'' ging das Projektteam auf die in der \nameref{sec:aufgabenstellung} vorgestellten Architekturrichtlinien und Prinzipien ein.

Als Ergebnis dieser Analyse entstand die unter Punkt \ref{sec:how-to-show-principles} vorgestellte, konsolidierte Liste mit Architekturrichtlinien.

Das vorliegende Kapitel soll aufzeigen, wie und insbesondere wo eine jeweilige Architekturrichtlinie in der entwickelten Beispielapplikation demonstriert werden konnte.

Dabei wird zuerst verifiziert, ob die potentielle Demonstrationsstelle aus Tabelle \ref{fig:how-to-show-principles-matrix} wie erwartet umgesetzt werden konnte.
Anschliessend wird das konkret zur Richtlinie entstandene Beispiel näher analysiert und erklärt.

Am Ende dieses Kapitels wird bewertet, wie gut die jeweiligen Richtlinien an der Beispielapplikation demonstriert werden konnten.

\newpage
\section{Übersicht}

Die Tabelle \ref{tab:overview-principle-demonstration} bietet eine Übersicht über die Analyse aller in Abschnitt \ref{sec:how-to-show-principles} ``\nameref{sec:how-to-show-principles}'' definierten Richtlinien.

Die Spalte ``Resultat'' kann dabei die Ausprägungen ``Positiv'', ``Neutral'' oder ``Negativ'' haben.

\begin{figure}[H]
	\begin{table}[H]
		\tablestyle
		\tablealtcolored
		\begin{tabularx}{\textwidth}{l c c X}
			\tableheadcolor
				\tablehead Richtlinie &
				\tablehead\rotatebox{90}{Demonstriert\hspace{3mm}} &
				\tablehead\rotatebox{90}{Resultat} &
				\tablehead Bemerkung
				\tabularnewline
			\tablebody
				\nameref{sec:principle-rp1-rest} & \faOk & \faMeh & Versionierung und Caching fehlt \tabularnewline
				\nameref{sec:principle-rp2-application-logic} & \faOk & \faSmile & \tabularnewline
				\nameref{sec:principle-rp3-http} & \faOk & \faSmile & \tabularnewline
				\nameref{sec:principle-rp4-link} & \faOk & \faSmile & \tabularnewline
				\nameref{sec:principle-rp5-non-browser} & \faOk & \faMeh & Aufgrund Facebook Login ist die Verwendung ohne Browser schwierig. \tabularnewline
				\nameref{sec:principle-rp6-should-formats} & \faOk & \faSmile & \tabularnewline
				\nameref{sec:principle-rp7-auth} & \faExclamation & \faSmile & \emph{HTTP Basic Authentication over SSL} nicht umgesetzt\tabularnewline
				\nameref{sec:principle-rp8-cookies} & & & \tabularnewline
				\nameref{sec:principle-rp9-session} & & & \tabularnewline
				\nameref{sec:principle-rp10-browser-controls} & \faOk & \faSmile & \tabularnewline
				\nameref{sec:principle-rp11-posh} & & & \tabularnewline
				\nameref{sec:principle-rp12-accessibility} & & & \tabularnewline
				\nameref{sec:principle-rp13-progressive-enhancement} & & & \tabularnewline
				\nameref{sec:principle-rp14-unobtrusive-javascript} & \faOk & \faSmile & Funktionalität bleibt ohne JavaScript erhalten\tabularnewline
				\nameref{sec:principle-rp15-no-duplication} & \faOk & \faSmile & Gemeinsame Codebasis für Client \& Server\tabularnewline
				\nameref{sec:principle-rp16-know-structure} & \faExclamation & \faFrown & Keine Anwendung unter Berücksichtigung des Prinzips RP14 \tabularnewline
				\nameref{sec:principle-rp17-static-assets} & \faOk & \faSmile & CSS \& View Template Preprocessing\tabularnewline
				\nameref{sec:principle-rp18-history-api} & \faOk & \faSmile & \tabularnewline
			\tableend
		\end{tabularx}
	\end{table}
	\caption{Übersicht Architekturrichtlinienanalyse (1/2)}
	\label{tab:overview-principle-demonstration}
\end{figure}

\begin{figure}[H]
	\begin{table}[H]
		\tablestyle
		\tablealtcolored
		\begin{tabularx}{\textwidth}{l c c X}
			\tableheadcolor
				\tablehead Richtlinie &
				\tablehead\rotatebox{90}{Demonstriert\hspace{3mm}} &
				\tablehead\rotatebox{90}{Resultat} &
				\tablehead Bemerkung
				\tabularnewline
			\tablebody
				\nameref{sec:principle-tp3-eat-your-own-api} & \faOk & \faSmile & Wiederverwendbare API umgesetzt\tabularnewline
				TP4 Separate user identity, sign-up (...) & \faOk & \faSmile & Facebook als Identity-Provider\tabularnewline
				\nameref{sec:principle-tp7-apply-the-web} & \faOk & \faMeh & Aufgrund fehlender Standardisierung nicht zufriedenstellend\tabularnewline
				\nameref{sec:principle-tp8-automate-everything} & \faOk & \faSmile & \gls{CI}, Make\tabularnewline
			\tableend
		\end{tabularx}
	\end{table}
	\caption{Übersicht Architekturrichtlinienanalyse (2/2)}
	\label{tab:overview-principle-demonstration-2}
\end{figure}


\newpage

\section{RP1 REST}
\label{sec:principle-rp1-rest}

\subsection*{Geplante Umsetzung}


\subsection*{Konkrete Umsetzung}


\subsection*{Diskussion}

\section{RP2 Application Logic}
\label{sec:principle-rp2-application-logic}

Die ``Application Logic'' ROCA-Richtlinie legt fest, dass jegliche Applikationslogik auf dem Server implementiert sein muss.

Falls der Server bereits \gls{REST} einsetzt (siehe vorheriger Abschnitt \ref{sec:principle-rp1-rest}, ``\nameref{sec:principle-rp1-rest}''), ist dieses Prinzip bereits vorgegeben. Die Trennung von Applikations- und Präsentationslogik wird dadurch erfüllt.

\subsection*{Geplante Umsetzung}

Die Beispielapplikation ``Roomies'' soll eine generalisierte REST-API zur Verfügung stellen, welche sowohl vom Client- wie auch vom Server-Code verwendet werden soll.

Diese API implementiert jegliche Businesslogik zum authentifizieren, authorisieren der Benutzer, sowie die Validierung und Speicherung der Daten.

Die Präsentationslogik wird als separates Layer implementiert. Dabei wird insbesondere dieser Quelltext mittels ``barefoot'' geteilt.

\subsection*{Konkrete Umsetzung}
Die geplante API wurde umgesetzt. Client und Serverside Code verwendet die gleichen API-Definitionen um Daten zu manipulieren. Dabei wird immer sichergestellt, dass nicht authentifizierte oder authorisierte Benutzer keinen Zugriff haben.
Weiter wird bei Datenmodifikationen sichergestellt, dass die Daten valid sind.

Durch die Trennung von API und Präsentation ist erfolgreich eine Trennung der Aufgaben (Separation of Concerns \cite{SeparationOfConcerns}) umgesetzt worden. Dies erlaubt es zum Beispiel in Zukunft, die API auf separaten Servern laufen zu lassen.

Weitere Informationen zur Softwarearchitektur können im Kapitel \ref{sec:sad} \nameref{sec:sad} nachgelesen werden.

\subsection*{Diskussion}
Um zu garantieren dass alle Daten korrekt sind und von den richtigen Benutzern ausgeführt werden, muss die Businesslogik auf dem Server liegen.
Falls jedoch insgesamt die User Expierence verbessert werden soll, können \emph{zusätzlich} einzelne Teile der Validierung von Daten auf dem Client ausgeführt werden.

In der Beispielapplikation wird keine Validierung auf Clientseite vorgenommen. Durch den Einsatz von Validatoren, welche auch auf dem Client ausführbar sind (siehe \cite{nodevalidator}) und der Unterstützung von Backbone.Models für Validation (siehe \cite{BackboneModelValidation}) könnte das zusätzlich jedoch auch noch auf dem Client gemacht werden.

Im Projektteam ist man sich einig: Die Applikationslogik muss grösstenteils auf dem Server liegen. Einzelne Validierungen o.ä. kann auch auf dem Client ausgeführt werden, jedoch nur \emph{zusätzlich} zur bereits bestehenden Validierung auf dem Server.
Diese Richtlinie ist für Client-Server Applikationen ein Muss. Ist das Programm allerdings keine klassische Client-Server Software (zum Beispiel Peer-to-Peer), kann das nicht so angewendet werden.
Mit der Verwendung der \gls{WebRTC} APIs wird somit dieses Prinzip für Teile der Applikation nicht wirklich einsetzbar. Dementsprechend müssen dafür neue Ideen gefunden werden, beziehungsweise bereits etablierte Peer-to-Peer Software- und Netzwerktechniken auf den Browser adaptiert werden.
\section{RP3 HTTP}
\section{RP4 Link}
\label{sec:principle-rp4-link}

\subsection*{Geplante Umsetzung}
Jede Seite innerhalb einer Web-Applikation muss mit einer eindeutigen URL adressierbar
sein.

Dies ist eines der ersten Paradigmas des Webs und auch eines der wichtigstens. In der Vergangenheit
ist es öfters vorgekommen, dass neuere Applikationen auf diesen ``Komfort'' verzichtet haben.
Heutzutage ist es aber u.a. dank der History API \cite{HistoryAPI} einiges
einfacher geworden, eindeutige URLs auch in JavaScript-lastigen Applikationen zu
verwenden.


\subsection*{Konkrete Umsetzung}
Weil die Beispielapplikation so oder so den REST \cite{REST} Prinzipien entspricht,
ist diese Richtlinie ein muss.

Jede URL entspricht einer eindeutigen Ressource und kann angesprochen werden.
Mithilfe der History API \cite{HistoryAPI} wird auch auf dem Browser die URL
geändert, obwohl u.U. nur ein AJAX-Request gemacht wird.

\subsection*{Diskussion}
Schon Tim Berners-Lee hat vor Jahren geschrieben: ``Cool URIs don't change'' \cite{CoolURIsTBL}.
Damit eine URL ``cool'' ist, muss sie zuerst mal vorhanden sein.

Auch aus einem weiteren Grund sind URLs nur dann ``cool'', wenn sie vorhanden sind
und niemals ändern: Mit den heuten Möglichkeiten des ``Sharing'' auf diversen Sozialen
Netzwerken, muss es möglich sein, direkt auf die momentane Seite zu verlinken.
\section{RP5 Non Browser}
\label{sec:principle-rp5-non-browser}

Das ``Non Browser''-Prinzip beschreibt, dass die Applikationslogik auch ohne die üblichen Browser verfügbar sein muss. Dies ist normalerweise der Fall, wenn man die bereits vorangegangenen Prinzipien einhaltet, insbesondere Abschnitt \ref{sec:principle-rp1-rest}, ``\nameref{sec:principle-rp1-rest}''.

\subsection*{Geplante Umsetzung}
Eine \gls{REST}-API wird laut Abschnitt \ref{sec:principle-rp1-rest} umgesetzt und somit ist es möglich, Ressourcen mit z.B. ``cURL'' \cite{curl} oder ``wget'' \cite{wget} abzurufen.
Durch die Verwendung von Facebook Login wird es aber eher schwierig, eine authentifizierte Session zu erhalten. Ein weiterer Login-Mechanismus wird trotzdem nicht geplant.

\subsection*{Konkrete Umsetzung}
Eine REST-API wurde umgesetzt. Durch die Anbindung an den Identity Provider ``Facebook'' (siehe Abschnitt \ref{sec:principle-tp4-seperate-user-identity}) ist es jedoch unumgänglich, vorher ein gültiges Login-Cookie anzufordern.

Um dies per ``cURL'' auf der Kommandozeile zu erreichen, kann wie folgt vorgegangen werden:
\begin{enumerate}
	\item Ein ``cURL'' Cookie-Jar erstellen lassen
	\item Mit einem anderen Browser auf ``Roomies'' einloggen
	\item Der Wert des Cookies ``connect.sid'' kopieren
	\item Im Cookie-Jar den Wert einsetzen
	\item Einen Request auf eine geschützte API-Ressource machen
\end{enumerate}

\begin{lstlisting}[language=Bash, caption=cURL Request auf Roomies, label=lst:curlRoomiesAPI]
# Schritt 1: Erstellung eines cURL Cookie-Jars in cookies.txt
~ $> curl -X 'GET' 'http://localhost:9001' --cookie-jar cookies.txt --verbose --location

# --------------------------------------------- #
# Jetzt müssten die Schritte 2-4 gemacht werden #
# --------------------------------------------- #

# Schritt 5: Request auf eine geschützte API Ressource
~ $> curl -X GET 'http://localhost:9001/api/community/ba/tasks' --cookie cookies.txt  --verbose --location
\end{lstlisting}

Es ist auch möglich, ohne den Dritt-Browser ein valides Cookie zu bekommen \cite{facebookLoginGist}. Auf diese Variante wird hier aber nicht eingegangen.

Wie geplant wurde aus Zeitgründen kein weiterer Login-Mechanismus implementiert.

\subsection*{Diskussion}
Durch den Einsatz einer generalisierten API-Schnittstelle mittels \gls{REST} kann eine Software auch ohne HTML-Parser die entsprechenden Daten auslesen.

Falls dabei Facebook Login eingesetzt wird, ist eine valide Session auf Facebook unumgänglich. Um solche Bedingungen für eine interne Kommunikation zwischen Komponenten nicht zu haben, kann zum Beispiel OAuth \cite{oauth} eingesetzt werden.

Wie auch bei Abschnitt \ref{sec:principle-rp1-rest} empfiehlt das Projektteam den Einsatz einer REST Schnittstelle. Es muss für die Authentifizierung aber beachtet werden, dass z.B. auch interne Komponenten auf die API zugreifen müssen. Deswegen empfiehlt es weiter, einen zusätzlichen Authentisierung-Provider (z.B. OAuth \cite{oauth}) zu implementieren, falls andere Komponenten auf die API zugreifen sollen.
\section{RP6 Should-Formats}
\section{RP7 Auth}
\label{sec:principle-rp7-auth}


Mit \emph{Basic Access Authentication} \cite{HTTPBasicAuth} steht seit Version 1.0 von \emph{HTTP} eine einfache Möglichkeit zur Verfügung, Ressourcen vor dem Zugriff Unbefugter zu schützen. Dank der hohen Verbreitung, bedingt durch die frühe Integration in den HTTP Standard, bringt der Verzicht auf Cookies, Session ID's oder Anmeldeseiten Eleganz durch Verwendung grundlegender Protokollfeatures.

Der Server kann durch Senden eines \emph{WWW-Authentication} Headers die Authentifizierung des Clients verlangen. Dieser wiederum sendet über den \emph{Authorization}-Header in seiner Antwort Benutzername und Passwort, welche mittels Base64 \cite{Base64} kodiert sind.

Wird \emph{HTTP Basic Auth} in dieser Form verwendet, werden Benutzername und Passwort in Klartext über das Internet übertragen. Abhilfe schafft \emph{Digest Access Authentication}, wenn auch nur unbefriedigend. Hierbei werden die Identifizierungsmerkmale vor dem Übertragen mit einem Hashing-Algorithmus, bspw. MD5 maskiert. Der Server vergleicht dann den übertragenen Wert mit dem selbst berechneten Hash.

Insbesondere die Verwendung des MD5-Hashes gilt als unsicher \cite{MD5Broken}. Die Einfachheit von \emph{Basic Access Authentication} kann unter Verwendung des \emph{Secure Socket Layer} Protokolls \cite{SSL} problemlos beibehalten werden. Dabei wird die Kommunikation zwischen Client und Server in einen verschlüsselten \emph{\gls{SSL}}-Tunnel verpackt. Alle übertragenen Informationen sind so nicht mehr ohne weiteres von Dritten lesbar.

Um eine effiziente Sicherung von Ressourcen zu gewährleisten, schlägt \emph{RP7 Auth} die Verwendung des vorgestellten \emph{HTTP Basic Access Authentication over SSL} vor.


\subsection*{Geplante Umsetzung}

Die Umsetzung von \emph{HTTP Basic Authentication over SSL} ist nicht geplant. Im Bereich \emph{Security} und \emph{Authentication} soll der Fokus auf ``\nameref{sec:principle-tp4-seperate-user-identity}'' gelegt werden.

\subsection*{Konkrete Umsetzung}

Den Erwartungen entsprechend wurde das von \emph{RP7} vorgeschlagene \emph{HTTP Basic Authentication over SSL} nicht in der Beispielapplikation \emph{Roomies} implementiert.


\subsection*{Diskussion}

Die ROCA Richtlinie \emph{Auth} ergänzt die Forderungen von ``\nameref{sec:principle-rp5-non-browser}'': Die Kombination ermöglicht den browserfreien Zugriff auf geschützte REST-API-Ressourcen.

Für die Beispielapplikation \emph{Roomies} wurde aufgrund der umzusetzenden Anforderungen auf die Demonstration von \emph{RP7 Auth} verzichtet. Zwar bietet auch die in \ref{sec:principle-tp4-seperate-user-identity} ``\nameref{sec:principle-tp4-seperate-user-identity}'' vorgestellte Lösung eine sichere Variante, einen Benutzer für den Zugang zur Applikation zu authentisieren. Wie der Abschnitt \ref{sec:principle-rp5-non-browser} jedoch ausführlich beschreibt, kann dies nur mit einem Browser praktikabel funktionieren.

Nach Umsetzung der Beispielapplikation empfiehlt das Projektteam darum, sollte es mit den Aufwänden vereinbar sein, \emph{RP7 Auth} umzusetzen und die API via \emph{HTTP Basic Auth over SSL} zugänglich zu machen.
\section{RP8 Cookies}
\label{sec:principle-rp8-cookies}
Die ROCA Richtlinie RP8 legt fest, dass Cookies lediglich zur Authentifizierung oder zur statistischen Analyse (Tracking) eines Benutzers verwendet werden soll.

\subsection*{Geplante Umsetzung}
Die Beispielapplikation soll lediglich im Bereich der Anmeldung des Benutzers auf Cookies zurückgreifen.

\subsection*{Konkrete Umsetzung}
Das Geplante konnte mit Erfolg umgesetzt werden.
Um einem User eine Session zuzuweisen zu können verwendet Express.js \cite{Expressjs} die \gls{Middleware} ``\emph{connect-session}'' \cite{ConnectSession}.

Konnte ein Benutzer erfolgreich authentifiziert werden, sendet der Server dem Client ein Cookie mit einer eindeutigen Session-ID. Diese ID muss mit allen künftigen Requests mitgeschickt werden.

In der Beispielapplikation wird die Session-Middleware wie in Quelltext \ref{lst:connect-session-middleware} initialisiert:

\begin{lstlisting}[language=JavaScript, caption=Connect Session Middleware \cite{RoomiesMiddlewareHttp}, label=lst:connect-session-middleware, firstnumber=31]
app.use(express.session({
	store: new SequelizeStore({
		db: db
	})
	, secret: config.sessionSecret
}));
\end{lstlisting}

Der ``SequelizeStore'' \cite{SequelizeStore} übernimmt die Speicherung der Session-Daten innerhalb der Datenbank.

\subsection*{Diskussion}
Das Einbinden von mehr Informationen in Cookies hat mehrere Nachteile:
\begin{itemize}
	\item Sicherheit \\
		Wenn Benutzername und Passwort in ein Cookie gespeichert werden, kann dies ohne Probleme von Drittpersonen mitgelesen werden. Falls HTTPS eingesetzt wird, zwar nur direkt im Browser, trotzdem ist dies ein nicht akzeptierbares Risiko. \\
		Ausserdem ist es einem User möglich, Cookies abzuändern. Ohne Validierung und/oder dem Einsetzen eines \glspl{MAC} ist dies auf dem Server nicht erkennbar.
	\item Datenmenge \\
		Cookies werden mit jeder Anfrage und jeder Antwort zwischen Client und Server mitgeschickt. Zwar ist die Grösse eines Cookies beschränkt, trotzdem wird die Datenmenge unnötig grösser.
\end{itemize}

Aus diesen Gründen sollte unbedingt darauf Verzichtet werden, Cookies für andere Zwecke als für die Wiederkehrende Authentifizierung über eine Session-ID oder für Tracking zu verwenden.
\section{RP9 Session}
\label{sec:principle-rp9-session}

Die Session sollte ausschliesslich für simple Authentifizierungsinformationen gebraucht werden. Simple ist z.B. eine eindeutige Session-ID. Damit kann der angemeldete User eindeutig erkennt werden und reicht für die Authentifizierung vollkommen aus.

\subsection*{Geplante Umsetzung}
Die Session ist kein Datenspeicher, aus diesen Grund sollen keine Datensätzen oder Seitenvariablen darin gespeichert werden.

Es werden nur benötigte Logindaten mittels Session übermittelt.

\subsection*{Konkrete Umsetzung}
Passport, die verwendete Bibliothek für die Authentifikation der User (siehe Abschnitt \ref{sec:principle-tp4-seperate-user-identity} ``\nameref{sec:principle-tp4-seperate-user-identity}'') speichert beim Anmelden die Benutzerdaten serialisiert in die Session. Ab diesen Punkt hat man jederzeit, solange der User angemeldet ist, zugriff auf seine Informationen.

Die Anwendung der Session wurde nicht nur für Authentifikationsdaten gebraucht. Temporäre Daten, die vom Server benötigt werden, wurden in die Session gespeichert. Hauptsächlich ging es darum die Benutzerfreundlichtkeit zu erhöhen. Konkret wurde es in ``Roomies'' benutzt, um den Benutzer nach einem Login an die gewünschte Seite zu weiterleiten.

\begin{quotation}
Ein Benutzer von ``Roomies'' liebt es, sich mit den anderen Mitbewohner zu vergleichen. Deswegen besucht er oft das Ranking. Auch einen Favorit hat er auf die Seite gesetzt, damit er direkt auf die Rangliste landet.

Ist der Benutzer noch nicht angemeldet und möchte die Rankingseite besuchen, wird er vom System aufgefordert sich anzumelden. Nach dem Anmeldevorgang wird ihm, wie gewünscht, die Rangliste gezeigt.
\end{quotation}

Das Weiterleiten auf die Rangliste ist nur möglich, weil der Server die Seite vor dem Login in die Session speichert. Siehe unter stehenden Quellcode \ref{lst:router-set-redirecturl}.

\begin{lstlisting}[language=JavaScript, caption=Router - Autorisationskontrolle \cite{roomiesRouter}, label=lst:router-set-redirecturl, firstnumber=225]

/** Function: redirectIfNotAuthorized
 * If the client is not authorized it will redirect. Otherwise it returns
 * false to indicate that the calling method can continue work.
 */
, redirectIfNotAuthorized: function redirectIfNotAuthorized() {
	if(!this.isAuthorized()) {
		var req = this.apiAdapter.req;
		req.session.redirectUrl = req.originalUrl;
		this.navigate('', {trigger: true});
		return true;
	}
	return false;
}
\end{lstlisting}

Nur so ist es möglich der Benutzer nach dem Anmelden an die, ursprünglich, aufgerufene Seite weiterzuleiten. Wäre diese Url nicht zwischengespeichert, würde das System dem Benutzer an eine fix definierte Seite weiterleiten, was der Benutzerfreundlichkeit schaden würde.

\subsection*{Diskussion}

\section{RP10 Browser-Controls}
\label{sec:principle-rp10-browser-controls}
Man befindet sich auf einem Online-Shop und sucht mit Hilfe der eingebauten Suche, Zubehör für seine neu ergatterte Fotokamera. Das Suchresultat ist aber noch zu grob. Deshalb klick man auf dem Zurückknopf des Browsers um auf die Suche zurück zu gelangen. Aber was uns hier erwartet, ist nicht etwa das Suchformular, welches man erwartet hätte, sondern die Startseite.
Dies ist nicht etwa ein fiktives Beispiel, sondern ein reeller Szenario aus dem Online-Shop des Elektronikgrosshändlers Digitec \cite{Digitec}.

Das ROCA-Prinzip ``Browser-Controls'' will genau dieses unangenehme Erleben verhindern.

\subsection*{Geplante Umsetzung}


\subsection*{Konkrete Umsetzung}


\subsection*{Diskussion}

\section{RP11 POSH}
\label{sec:principle-rp11-posh}

``POSH'', ausgeschrieben ``Plain Old Semantic HTML'', bezeichnet das Erstellen von HTML-Seiten mithilfe von semantischen Elementen. \cite{SemanticHTML}

Semantisches HTML bietet den Vorteil, dass ``\glspl{SearchEngineSpider}'' die Seite besser verstehen, interpretieren und kategorisieren können. Wird eine Seite gut kategorisiert erscheint sie eher bei einer Suche in einer Suchmaschine wie \emph{Google} oder \emph{Bing}.

\subsection*{Geplante Umsetzung}
Das HTML Markup der Beispielapplikation sollen eine klare und logische Struktur aufweisen.

Alle Seiten haben einen Titel, eine Überschrift und entsprechenden Inhalt.

Für Überschriften werden die Tags \emph{h1}, für die grösste und wichtigste Überschrift, bis \emph{h6} für untergeordnete.

Tabellen sollen für tabellarische Daten verwendet werden und nicht zur gestalterischen Strukturierung einer Seite.


\subsection*{Konkrete Umsetzung}
Um semantische Inkorrektheiten durch Wiederholungsfehler zu reduzieren, wurde in \emph{Roomies} mit Templates gearbeitet. So können bspw. Menüs oder Fusszeilen einmalig definiert und wiederverwendet werden. Dies geht sehr stark nach dem ``\gls{DRY}'' Prinzip. Fehler werden verhindert, indem man sie gar nicht zu schreiben hat.

Quelltext \ref{lst:layoutDefinition} zeigt einen Auschnitt des HTML-Markups vom Haupttemplate von \emph{Roomies}''. Die Seitenstruktur ist klar zu erkennen: Oben beginnt der Header mit dem Menü, gefolgt von einem Platzhalter für allfällige Fehler-, Warnungs- oder Informationsmeldungen. Weiter ist der Hauptbereich mit der ID ``main'' ersichtlich, welcher zur Laufzeit mit eigentlichen Applikationsinhalten gefüllt wird. Abschliessend steht ein semantisch korrektes \emph{footer}-Element, welches final den Link zur Abmeldung des Benutzers enthalten wird.

\begin{lstlisting}[language=HTML, caption=Layout Definition \cite{roomiesHtmlSkeleton}, label=lst:layoutDefinition, firstnumber=27]
<body>
	<header id="menu"></header>
	<div id="flash-messages" class="row flash-messages"></div>
	<section id="main"></section>
	<footer id="footer"></footer>
</body>
\end{lstlisting}

\subsection*{Diskussion}
Semantisches HTML ist aus zwei Gründen eine gute Idee:
\begin{itemize}
	\item Suchmaschinenoptimierung
	\item Barrierefreiheit
\end{itemize}

\subsubsection*{Suchmaschinenoptimierung}
Suchmaschinen wenden komplexe Algorithmen an, um Webseiten zu analysieren und in verwertbare Suchresultate zu transformieren. Ein Entwickler von Webseiten und -applikationen kann durch die Verwendung von semantisch korrekten Tag-Elementen viel zur Optimierung des Suchprozesses beitragen. Diese ermöglichen den Suchmaschinen ein fehlerfreies Interpretieren der untersuchten Informationen.
\\ \\
Ergänzend dazu helfen Standards wie \emph{schema.org} \cite{SchemaOrg}. Unterstützt von Google, Yahoo und Microsoft, hat dieser zum Ziel, die Maschinenlesbarkeit von Webseiten zu erhöhen und spezifischen Teilen einer Seite analyisierbaren \emph{Sinn} zu geben.

Mithilfe von \emph{schema.org} können bspw. Attribute von Produkten (Name, Preis etc.) gekennzeichnet und gezielt für Suchmaschinen verwertbar gemacht werden. Sucht ein Benutzer nun nach Produkten, kann die Suchmaschine die nun strukturierten Informationen besser durchsuchen und massgeschneiderte Suchergebnisse dem Benutzer präsentieren.

Für Dienstanbieter ergibt sich so ein enormes Potential zur Monetarisierung.

\subsubsection*{Barrierefreiheit}
Nebst der Verbesserung der Sichtbarkeit bei Suchmaschinen ist ein semantisch korrektes HTML-Markup auch aus Sicht der Barrierefreiheit sinnvoll.

Ein signifikanter Teil der Bevölkerung ist auf barrierefreie Webseiten angewiesen \cite{BarrierefreiesInternet}. Das Erstellen von barrierefreien Webseiten unterstützt Geräte wie \emph{Screen Reader} Inhalte zu analysieren und in einem für den gehandicapten Benutzer gerechten Format zugänglich zu machen.

Semantische Tags sind einer der wichtigsten Schritte auf dem Weg zu einem barrierefreien Webangebot.
\\ \\
Das Projektteam ist aufgrund der genannten Gründe der Meinung, dass semantische Tags und somit \emph{RP11 POSH} sehr wichtig ist und auch für eine Webapplikation umgesetzt werden sollte.
\section{RP12 Accessibility}
\label{sec:principle-rp12-accessibility}
Aufbauend auf dem Prinzip ``\nameref{sec:principle-rp11-posh}'' besagt das Prinzip \emph{Accessibility}, dass alle Seiten für Hilfesoftware/-geräte wie \emph{Screen Reader} oder \emph{Braillezeile} zugänglich sein müssen.


``Barrierefreiheit schliesst sowohl Menschen mit und ohne Behinderungen als auch Benutzer mit technischen (Textbrowser oder PDA) oder altersbedingten Einschränkungen (Sehschwächen) [\ldots] ein.'' -- \cite{BarrierefreiesInternet}

\emph{RP12 Accessibility} beschränkt sich dabei auf Zugänglichkeit über Hilfeanwendungen. So wird die im Zitat erwähnte Problematik \emph{Farbblindheit} (richtige Farbenwahl etc.) oder die Navigation ohne Maus nicht abgedeckt.

\subsection*{Geplante Umsetzung}
Damit Hilfesoft- und Hilfehardware das User Interface der Beispielapplikation interpretieren kann, soll das HTML Markup, wie unter \ref{sec:principle-rp11-posh} ``\nameref{sec:principle-rp11-posh}'' erklärt, eine semantisch korrekte Struktur aufweisen.

\subsection*{Konkrete Umsetzung}

\emph{Roomies} verwendet, wo nötig, semantisch korrekte HTML-Elemente.

\subsection*{Diskussion}
In ``\nameref{sec:principle-rp11-posh}'' wird ausführlich über die Wichtigkeit eines barrierefreien Internets diskutiert. Das Prinzip \emph{RP12 Accessibility} strebt durch die notwendigen Vorkehrungen in die selbe Richtung, hat aber die Unterstützung gehandicapter Personen zum Ziel.

Leider gehört die Berücksichtigung von Farbenblindheit o.Ä. nicht zum Repertoire von \emph{RP12}. Für das Projektteam gehören aber auch diese Themen klar zum Aufgabenbereich der Richtlinie \emph{Accessibility}.

Das Projektteam schlägt darum die Ergänzung von \emph{RP12 Accessibility} um die erwähnten Punkte vor. Machen es die Anforderung nötig, empfiehlt das Projektteam die Umsetzung von \emph{RP12}.
\section{RP13 Progressive Enhancement}
\label{sec:principle-rp13-progressive-enhancement}

Die Client-Technologien HTML und CSS haben sich über Jahre weiterentwickelt. Dies auch meistens mit dem Hintergedanken ``Progressive Enhancement'' oder Rückwärtskompatibilität. Das bedeutet, dass die Entwicklung vielfach versucht hat, Rücksicht auf die älteren Browserversionen zu nehmen.
Deutlich sieht man diese Rückwärtskompatibilität beim neuen HTML5 Standard. Neue Formularfeldtypen wie ``tel'' oder ``email''wurden eingeführt. Browser die HTML5 nicht unterstützen, wechseln bei einem für sie unbekannten Typ zum normalen Textfeld zurück.

\begin{lstlisting}[language=HTML, caption={Formularfeld mit HTML5, welches eine Telefonnummer erwartet}, label={lst:html5TelInput}]
<input type="tel" name="telefon">
\end{lstlisting}

Natürlich gibt es keine Regel ohne Ausnahme. Manche Tags, die im HTML 5 Standard eingeführt wurden, werden überhaupt nicht beachtet. Dies kann vor allem bei Versionen \emph{kleiner Neun} des Internet Explorers beobachtet werden.

\subsection*{Geplante Umsetzung}
Die Beispielapplikation soll, wie bereits in Kapitel \ref{sec:requirments-engineering-nonfunctionals} erwähnt, folgende Versionen unterstützen: Internet Explorer 8 und höher, Chrome 25 und höher, Firefox 19 und höher und Safari 6 und höher.

Auch Browser auf Smartphones sollten nicht vernachlässigt werden. Hierfür sollen Safari 6 und höher und Android Browser 4.0 und höher unterstützt werden.

\subsection*{Konkrete Umsetzung}
Da nicht alle geplanten und zu unterstützenden Browser HTML5 interpretieren können, musste ein kleiner Trick angewendet werden, um so die Rückwärtskompatibilität zu erhöhen. Dieser Trick heisst ``modernizr'' \cite{modernizr}, eine JavaScript-Bibliothek, welche gezielt den verwendeten Browser auf seinen Funktions- und Unterstützungsumfang von HTML und CSS überprüft. Das Resultat wird dann in einem JavaScript-Objekt gespeichert und zur Verfügung gestellt.  Zusätzlich läuft modernizr in eine kleine Schleife, um die neuen Tags (head, section, article, nav, u.w.) zu aktivieren. Das bedeutet, dass diese Tags nicht mit einem ``div''-Tag ersetzt werden müssen, wie dies sonst der Fall wäre.

Um diese Bibliothek einzubinden, hat man nichts anderes zu tun als das Skript im Header hinzuzufügen.

\begin{lstlisting}[language=HTML, caption=Einbinden von modernizr \cite{roomiesLayout}, firstnumber=12, label=lst:mdernizrLayoutServer]
<script src="/javascripts/lib/custom.modernizr.js"></script>
\end{lstlisting}

Mit ``modernizr'' konnte erreicht werden, dass in allen geplanten Browser alle Elemente erscheinen, wenn auch nicht immer korrekt. Nicht immer korrekt, weil im Internet Explorer 8 die Darstellung des Bildes im oberen linken Rand nicht dem entspricht, was erwartet wurde.

\begin{figure}[H]
	\centering
	\includegraphics[width=12cm]{content/principle-demonstration/images/progressive-enhancement-ie8.png}
	\caption{Fehlerhafte Darstellung im Internet Explorer 8}
	\label{fig:iossafari-datepicker}
\end{figure}

\subsection*{Diskussion}
Abhängig vom Zielpublikum der zu erstellenden Webapplikation gewinnt oder verliert die Unterstützung älterer Browser und somit ``Progressive Enhancement'' an Wichtigkeit. Mithilfe von Tools wie \emph{modernizr} kann Kompatibilität einfacher gewährleistet werden. Die Implementation der Beispielapplikation hat jedoch bewiesen, dass diese Werkzeuge kein Allheilmittel darstellen.

Manuelles Testen und insbesondere eventuelle manuelle Korrekturen sind und bleiben wichtig.
\\ \\
Das Projektteam ist sich einig: \emph{Progressive Enhancement} ist wichtig. Insbesondere die Sicherstellung der Kompatibilität zu älteren Versionen von Internet Explorer stellt immer noch eine Herausforderung dar und verursacht meist immense Kosten.

Weiter ist das Projektteam der Meinung, dass diese Probleme in Zukunft zwar nicht komplett wegfallen werden, aber immerhin reduziert werden könnten.

Microsoft verfolgt mit Internet Explorer 10 eine ähnliche Updatestrategie wie Mozilla und Google mit ihren Browsern: Häufige Veröffentlichung von Updates in kurzen Abständen \cite{MicrosoftQuickensIEReleaseCycle} ermöglichen Microsoft künftig die vermehrte Partizipation bei der Entwicklung von Web-Standards. In absehbarer Zeit wird dies Webapplikationsentwicklern helfen, bestmögliche Kompatibilität zwischen verschiedenen Browsern innerhalb kürzester Zeit zu gewährleisten.
\section{RP14 Unobtrusive JavaScript}
\label{sec:principle-rp14-unobtrusive-javascript}

Aktuelle Webapplikationen können grob in zwei Kategorien eingeteilt werden:

\begin{figure}[H]
	\begin{table}[H]
		\tablestyle
		\tablealtcolored
		\begin{tabularx}{\textwidth}{l l X}
			\tableheadcolor
				\tablehead Kategorie &
				\tablehead Beispiel &
				\tablehead Erläuterung
				\tabularnewline
			\tablebody
				Server-Rendering &
				\emph{GitHub} \cite{GitHub} &
				User Interface wird auf dem Server gerendert, JavaScript bringt lediglich dynamisch geladene Inhalte, Effekte oder zusätzliche ``optionale'' Features.
				\tabularnewline

				JavaScript Client &
				\emph{Google Drive} \cite{GoogleDrive} &
				User Interface wird komplett im Browser mittels JavaScript aufgebaut. Ohne JavaScript keine Funktionalität oder schlechtere User Experience.
				\tabularnewline
			\tableend
		\end{tabularx}
	\end{table}
	\caption{Kategorien aktueller Webapplikationen}
	\label{tab:current-webapplication-categories}
\end{figure}

Die Kategorie \emph{Server-Rendering} zeichnet sich durch eine hohe Kompatibilität mit allen möglichen Internetbrowsern aus. Durch die Generierung des HTML Markups losgelöst vom schlussendlichen Zielclient, liegt die ganze Verantwortung, vom beschaffen anzuzeigender Daten bis hin zum zusammenstellen des HTML \gls{DOM}'s komplett bei der Serverkomponente.

Zwar kommt auch bei diesem Typus oftmals JavaScript zur Anwendung, meist beschränkt sich dessen Anwendung aber auf die Ergänzung des bereits statisch geladenen Inhaltes. So lädt \emph{Mila} \cite{Mila} beim Seitenwechsel, sofern JavaScript aktiviert, neuen Inhalt über einen \gls{AJAX} Request. Nach Erhalt des vorgerenderten HTML Markups aus der Antwort ersetzt es entsprechende Inhalte im aktuell angezeigten HTML \gls{DOM} des Browsers.

Als Programmiersprache auf dem Applikationsserver kommt hier oft Java, PHP, Ruby o.Ä. zum Einsatz.

Beim puren \emph{JavaScript Client} liegt der Programmcode für das Rendern des User Interfaces mit all seinen Inhalten als JavaScript Quelltext vor. Nach erfolgreicher Übertragung zum Internetbrowser initiiert dieser die Erstellung der Applikationsoberfläche im HTML \gls{DOM} des Clients.

Sollen dynamische Informationen angezeigt werden, müssen diese über eine Serviceschnittstelle beim entsprechenden Anbieter angefragt werden (siehe bspw. Abschnitt \ref{sec:principle-rp1-rest} ``\nameref{sec:principle-rp1-rest}'').

Es ist zu erahnen, dass diese Art von Webapplikation ohne JavaScript-Unterstützung im Browser des Endbenutzers nicht ausgeführt werden kann. Ein Beispiel hierfür liefert der \emph{Google Drive} \cite{GoogleDrive} Webclient. Wie in Abbildung \ref{fig:googleDriveNoJs} ersichtlich verweigert dieser ohne aktiviertes JavaScript die Funktion und zeigt ein leeres Standardlayout mit einer entsprechenden Meldung an.

\begin{figure}[H]
	\centering
	\includegraphics[width=12cm]{content/principle-demonstration/images/googledrive-nojs.png}
	\caption{\emph{Google Drive} in Firefox 21.0 mit deaktiviertem JavaScript}
	\label{fig:googleDriveNoJs}
\end{figure}



\subsection*{Geplante Umsetzung}


\subsection*{Konkrete Umsetzung}


\subsection*{Diskussion}

\section{RP15 No Duplication}
\section{RP16 Know Structure}
\section{RP17 Static Assets}
\label{sec:principle-rp17-static-assets}

Die ``Static Assets'' ROCA Richtlinie will dass jeglicher JavaScript Code oder CSS Stylesheets für den Client von statischer Natur ist. Dies bedeutet, dass der Server keine dynamische Generierung von eben diesem Code vornimmt.

\subsection*{Geplante Umsetzung}
\subsubsection*{CSS Stylesheets}
Die verwendeten CSS Stylesheets sollen mit dem SASS Präprozessor \cite{SASS} erstellt werden. Zwar muss der eigentliche Stylesheet Code zwar vorneweg einmalig übersetzt werden, die dadurch entstehenden Vorteile beim Entwickeln der Formatierungsinformationen sind jedoch bei Weitem grösser.

Da zudem statische SASS-Quelldateien übersetzt werden, kann der Anspruch keine dynamischen CSS Formatierungen zu generieren befriedigt werden.

\subsubsection*{Clientside JavaScript}
Die JavaScript Applikation für den Client soll auf verschiedene Dateien aufgeteilt werden. Für den späteren produktiven Betrieb ist das Übertragen vieler kleinen Quellcode-Dateien jedoch nicht effektiv.

Aus diesem Grund sollen auf dem Backend alle Client-Quellcode-Dateien zusammengefasst werden können und mit gängigen Methoden zur schnellstmöglichen Übertragung über das Internet optimiert werden.

Ähnlich wie beim SASS-Präprozessor soll auch hier kein dynamischer Code entstehen. Es wird lediglich eine Optimierung der zu übertragenden Informationen vorgenommen.


\subsection*{Konkrete Umsetzung}
Beide geplanten Umsetzungen konnten erfolgreich implementiert werden.

\subsubsection*{CSS Stylesheets}
Bevor die eigentliche Beispielapplikation gestartet wird, wird die Umwandlung des SASS-Quellcodes in gewöhnlichen CSS-Quellcode vorgenommen.

Die daraus entstehende Datei kann anschliessend ohne weitere Veränderungen wie gewohnt vom Webserver an den Browser des Benutzers übertragen werden.

\subsubsection*{Clientside JavaScript}
Um den auf verschiedene Dateien verteilten JavaScript Code für den Client zusammenzufassen und zu optimieren wird \emph{browserify-middleware} für Express.JS \cite{browserifymiddleware} verwendet.

Der Client-seitige Quellcode kann wie in Node.JS gewohnt in CommonJS Modulen \cite{commonjsmodules} strukturiert werden. \emph{browserify-middleware} untersucht den Quellcode anschliessend und fasst die eingebundenen Module anhand der vorhandenen \emph{require}-Befehlen automatisch zu einer grossen JavaScript-Datei zusammen.

Falls konfiguriert, wird diese abschliessend von Kommentaren und unnötigen Füllzeichen befreit und mittels Gzip \cite{gzip} komprimiert an den Browser des Benutzers übertragen.


\subsection*{Diskussion}
Das Projektteam ist davon überzeugt, dass die umgesetzten und aufgezeigten Methoden und Mechanismen für moderne Webapplikationen ein extrem hilfreiches Werkzeug sind.

Im Bereich der clientseitigen JavaScript Entwicklung ermöglichen das erwähnte \emph{browserify-middleware} oder andere Bibliotheken wie RequireJS \cite{requirejs} erst das effiziente verteilen von Quellcode auf verschiedene Dateien.

In die gleiche Richtung strebt SASS: Es ergänzt CSS um viele nützliche Features wie Variablen und Mixins. Daneben ermöglicht es aber, entsprechende Bibliotheken vorausgesetzt, eine extreme Vereinfachung der Entwicklung von CSS-Stylesheets, welche mit verschiedenen Browsern kompatibel ist.

Sollten die vorgestellten Techniken für jede Webapplikation verwendet werden?

Das Projektteam ist der Meinung, dass ab einem gewissen Projektumfang uneingeschränkt auf JavaScript-Modularisierungsmethoden und CSS-Präprozessoren gesetzt werden sollte.
\section{RP18 History API}
\label{sec:principle-rp18-history-api}

\subsection*{Geplante Umsetzung}


\subsection*{Konkrete Umsetzung}


\subsection*{Diskussion}

\section{TP3 Eat your own API dog food}
\label{sec:principle-tp3-eat-your-own-api}

Eine Applikation mit einer verteilten, entkoppelten Architektur kommt unweigerlich zu einem Punkt, an welchem die einzelnen Komponenten Schnittstellen zur gegenseitigen Interaktion definieren müssen.

Die durch diesen Prozess entstehenden API's sind klassischerweise auf spezifische Anwendungsfälle zugeschnitten da so schnellstmöglich die applikationseigenen Anforderungen umsetzen können.

Als weitere Konsequenz werden ``unschöne'' Interfacemethoden meist gar nicht erst für externe Konsumenten verfügbar gemacht.

Langfristig besteht die Gefahr, dass ein Flickwerk aus anwendungsfallspezifischen Interfaces resp. Interfacemethoden entsteht.

Mit ``\emph{Eat your own API dog food}'' forciert Tilkov von Beginn an die Konzipierung und Umsetzung guter und generischer Schnittstellen für Applikationskomponenten. Als Motivationsfaktor gehört deshalb auch der Grundsatz zu seiner Forderung, dass keine privaten Methoden existieren sollen.


\subsection*{Geplante Umsetzung}

Für die Beispielapplikation \emph{Roomies} soll ein Servicelayer auf Basis einer HTTP \gls{REST} Architektur entwickelt werden. Als Datenformat soll \gls{JSON} verwendet werden.

Jegliche Interaktion mit den Objekten aus der Problemdomäne soll innerhalb dieses Layers gekapselt werden.

Entsprechend der \gls{REST} Richtlinien (siehe \ref{sec:principle-rp1-rest} ``\nameref{sec:principle-rp1-rest}'') soll jedes dieser Objekte gezielt abgefragt und manipuliert werden können.

Es sind keine privaten Methoden geplant. Soll ein Objekt vor Zugriffen unbefugter Konsumenten geschützt werden, sind entsprechende Sicherheitsmechanismen umzusetzen.


\subsection*{Konkrete Umsetzung}

Wie bereits in Abschnitt \ref{sec:principle-rp1-rest-concrete-solution} des Kapitels ``\nameref{sec:principle-demonstration}'' erläutert, konnte die generische Serviceschnittstelle für alle Objekte aus der \emph{Roomies} Problemdomäne (siehe \ref{sec:sad-domain-model} ``\nameref{sec:sad-domain-model}'') umgesetzt werden.

Es wurde komplett auf private Methoden verzichtet. Zum Schutz sensibler Daten wurde wie in den Abschnitten \ref{sec:principle-rp7-auth}, \ref{sec:principle-rp8-cookies} sowie \ref{sec:principle-rp9-session} beschrieben ein Session-basierter Authentifizierungsmechanismus via Facebook (siehe \ref{sec:principle-tp4-seperate-user-identity} ``\nameref{sec:principle-tp4-seperate-user-identity}'') implementiert.


\subsection*{Diskussion}

Klar strukturiertes und wohldefiniertes Schnittstellendesign ist bereits bei einer kleineren Applikation hilfreich, ab einem Umfang von mehreren Komponenten sogar ein absolutes Muss.

Gerade in einer grösseren Service-Landschaft, wie diese oftmals in grossen Konzernen wie Banken anzutreffen ist (Beispiel \emph{BIAN Service Landscape 2.0} \cite{BIANServiceLandscape}), ist es jedoch üblich, auch private Interfacemethoden zu unterhalten.

Bis zu einem gewissen Punkt mag es also durchaus berechtigt sein, die eigene Motivation für korrektes Interfacedesign aus der Zurschaustellung nach Aussen zu beziehen. Je nach Anforderungen kann es aber durchaus Sinn machen, von diesem Grundsatz abzuweichen.

Für eine Non-Enterprise-Applikation kann das Projektteam \emph{TP3} anstandslos befürworten. Die ausführliche Auseinandersetzung mit der eigenen API resultiert in einer Schnittstelle, welche ohne Weiteres von einer Smartphone App oder einer beliebigen anderen Applikation konsumiert werden kann.
\section{TP4 Separate user identity and sign-up (...)}
\section{TP7 Apply the Web instead of working around}
\label{sec:principle-tp7-apply-the-web}

\subsection*{Geplante Umsetzung}


\subsection*{Konkrete Umsetzung}


\subsection*{Diskussion}

\section{TP8 Automate everything or you will be hurt}
\label{sec:principle-tp8-automate-everything}

TP8 greift einen allgemein gültigen Vorsatz aus dem Software Engineering auf: Mit ``\emph{Don't repeat yourself}'' wird zum einen sich wiederholender Quellcode minimiert, als auch wiederkehrende Aufgaben automatisiert.

Entwickelt man den \emph{DRY}-Ansatz weiter, so landet man unweigerlich bei der Verwendung von automatisierten Tests und Deployments mittels Continuous Integration Systemen.

\subsection*{Geplante Umsetzung}
In der Tabelle \ref{fig:how-to-show-principles-matrix} war geplant, die Richtlinie TP8 mittels der Entwicklungsumgebungstools zu demonstrieren.

Folgende Aufgaben konnten im Zuge dieser Planung automatisiert werden:

\begin{table}[H]
\tablestyle
\tablealtcolored
\begin{tabularx}{\textwidth}{l X}
\tableheadcolor
	\tablehead ID &
	\tablehead Aufgabe
	\tabularnewline
\tablebody
	\textit{TP8.1} & Starten der Beispielapplikation\tabularnewline
	\textit{TP8.2} & Ausführung der Unit Tests\tabularnewline
	\textit{TP8.3} & Qualitative Überprüfung des Quellcodes (Code Style Guidelines)\tabularnewline
	\textit{TP8.4} & Umwandlung von SASS zu CSS Stylesheets\tabularnewline
	\textit{TP8.5} & Erstellung von Quellcode Dokumentation\tabularnewline
	\textit{TP8.6} & Veröffentlichung von Dokumentation (dieses Dokument, aber auch Quellcode Dokumentation), Testergebnisse und Test Code Coverage Berichten\tabularnewline
	\textit{TP8.7} & Umwandlung des LaTeX Quellcodes zur finalen PDF Dokumentation\tabularnewline
\tableend
\end{tabularx}
\caption{Automatisierte Aufgaben}
\end{table}


\subsection*{Konkrete Umsetzung}
Als Kernkomponente für die Automatisierung der definierten Aufgaben wird ein \emph{Makefile} (\cite{RoomiesMakefile} und \cite{ThesisMakefile}) verwendet. Dieses wird von \emph{GNU Make} \cite{make} interpretiert und ermöglicht so das Ausführen von verschiedensten Operationen.

Das Listing \ref{lst:makefileTesting} zeigt exemplarisch die Befehlsdefinition für das Ausführen der Unit sowie funktionalen Tests.

\begin{lstlisting}[language=Bash, caption=Ausschnitt Makefile: Testing \cite{RoomiesMakefile}, label=lst:makefileTesting]
REPORTER = spec
TEST_CMD = NODE_ENV=test ./node_modules/.bin/mocha --require test/runner.js --globals config

test: test-unit test-functional

test-unit:
	@echo "Running Unit Tests:"
	@$(TEST_CMD) --reporter $(REPORTER) src/lib/*/test.js

test-functional:
	@echo "Running Functional Tests:"
	@$(TEST_CMD) --reporter $(REPORTER) test/*-test.js
\end{lstlisting}

Soll nun der Task \emph{test} aufgerufen werden, geschieht dies durch den einfachen Befehl \emph{make test} auf der Kommandozeile.


\subsection*{Continuous Integration}
Die Vorgestellten Aufgaben können sowohl lokal auf dem Entwicklerrechner als auch auf dem Continuous Integration System ausgeführt werden.

Wie im Kaptiel \ref{sec:qualitymanagement} ``\nameref{sec:qualitymanagement}'' definiert, wird hierzu die Open Source Plattform Travis CI \cite{TravisCI} verwendet. Der folgende Quelltext zeigt die \emph{.travis.yml} Datei \cite{RoomiesTravisYML}. Sie steuert die Ausführung eines Builds auf Travis:

\begin{lstlisting}[language=XML, caption=.travis.yml \cite{RoomiesTravisYML}, label=lst:roomiesTravisYML]
before_install:
    - ./travis/before_install.sh

after_success:
    - ./travis/after_success.sh

language: node_js
node_js:
  - 0.8

script: "make docs test-coverage lint"

env:
    global:
        - secure: "eWkg0QTrYfWbPpitHRZWvtnTD2gTZsHEfZI1/cVA3mUs1Vycx/bNHzAt4fNm\nWR7EsTkaNLkc6p3JA797kwIyCugt+0D2tTdn7ra532Gye9u/KGjJB38HuJ0A\nQxibB+ahyAOJL+NjRNAQsME2lFmNv950/4sRbXYybijxJD6fqcw="

        - GH_USER_NAME: hsr-ba-ajw-2013
        - GH_PROJECT_NAME: BA
        - GIT_AUTHOR_NAME: "TravisCI"
        - GIT_AUTHOR_EMAIL: ""

        - CI_HOME=`pwd`/$TRAVIS_REPO_SLUG
        - RESULT_UNIT_COVERAGE_PATH=$CI_HOME/unit-coverage.html
        - RESULT_FUNCTIIONAL_COVERAGE_PATH=$CI_HOME/functional-coverage.html
        - RESULT_DOCS=$CI_HOME/docs
\end{lstlisting}

\subsection*{Diskussion}
Grundsätzlich rechtfertigt jede Aktion, welche mehr als einmal durchgeführt wird das Automatisieren dieser. Dies trifft umso mehr zu, wenn mehrere Entwickler am Entwicklungsprozess beteiligt sind.

Im Normalfall hat nicht jeder einzelne Projektmitarbeiter Zeit, sich bspw. um das Aufsetzen seiner eigenen Testumgebung zu kümmern. Dabei steigert sich die Effektivität eines Teams ungemein, wenn mittels einem einfachen Befehl die Ausführung der Unit oder gar Integration Tests angestossen werden kann.

Das Credo ``Don't repeat yourself'' ist somit ganz klar Trumpf.

Das Projektteam war zudem positiv davon überrascht, was sich mit frei zugänglichen Continuous Integration Lösungen wie Travis CI \cite{TravisCI} umsetzen lässt. Von der Generierung von Dokumentationen bis hin zur regelmässigen Ausführung von Unit Tests lassen sich ohne grossen Aufwand unbeliebte Aufgaben problemlos Automatisieren und Auslagern.
