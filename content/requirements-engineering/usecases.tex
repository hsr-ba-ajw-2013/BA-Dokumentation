\newpage

\section{Use Cases}

\begin{figure}[h!]
	\centering{
		\resizebox{0.9\textwidth}{!} {
			\begin{tikzpicture}
				\begin{umlsystem}[x=5]{Roomies}

					\umlusecase[y=0]{UC1: Anmelden}
					\umlusecase[y=-2]{UC2: WG erstellen}
					\umlusecase[y=-4]{UC3: WG beitreten}
					\umlusecase[y=-6]{UC4: WG verlassen}
					\umlusecase[y=-8]{UC5: Aufgabe erstellen}
					\umlusecase[y=-10]{UC6: Aufgabe bearbeiten}
					\umlusecase[y=-12]{UC7: Aufgabe erledigen}
					\umlusecase[y=-14]{UC8: Rangliste anzeigen}
					\umlusecase[y=-16]{UC9: WG auflösen}
					\umlusecase[y=-18]{UC10: Benutzer verwalten}
					\umlusecase[x=8, y=-8, width=2cm]{UC11: auf Social Media Plattform teilen}

				\end{umlsystem}

				\umlactor[y=-7]{Bewohner}
				\umlactor[y=-16]{WG-Verwalter}
				%\umlactor[y=-25]{Administrator}
				\umlactor[y=-8, x=17]{Facebook}

				\umlinherit{WG-Verwalter}{Bewohner}

				\umlextend{usecase-11}{usecase-3}
				\umlextend{usecase-11}{usecase-7}
				\umlextend{usecase-11}{usecase-8}

				\umlassoc{Bewohner}{usecase-1}
				\umlassoc{Bewohner}{usecase-2}
				\umlassoc{Bewohner}{usecase-3}
				\umlassoc{Bewohner}{usecase-4}
				\umlassoc{Bewohner}{usecase-5}
				\umlassoc{Bewohner}{usecase-6}
				\umlassoc{Bewohner}{usecase-7}
				\umlassoc{Bewohner}{usecase-8}
				% \umlassoc{Bewohner}{usecase-11}

				\umlassoc{WG-Verwalter}{usecase-9}
				\umlassoc{WG-Verwalter}{usecase-10}

				\umlassoc{Facebook}{usecase-11}
				\umlassoc{Facebook}{usecase-1}

				\umlnote[x=13, y=0]{usecase-1}{Die erste Anmeldung wird als eine Registrierung interpretiert}
				\umlnote[x=13, y=-3]{usecase-3}{Eine WG kann nur über eine Einladung (Link) beigetreten werden}
			\end{tikzpicture}
		}
	}

	\caption{Use Case Diagramm}
\end{figure}

\subsection{Aktoren}
\begin{table}[H]
	\tablestyle
	\tablealtcolored
	\begin{tabularx}{\textwidth}{lX}
		\tableheadcolor
		\tablehead Name &
		\tablehead Beschreibung \tabularnewline
		\tablebody
			Bewohner &
			Als Bewohner wird ein Anwender der Roomies Anwendung bezeichnet, der zu einer WG gehört. \newline
			Dieser besitzt die Rechte Aufgaben seiner zugehörigen WG zu verwalten.
			\tabularnewline
			WG-Verwalter &
			Der WG-Verwalter ist eine Erweiterung des Aktors Bewohner. Der Ersteller einer WG wird automatisch als WG-Verwalter ernannt. Diese Rolle kann an Bewohner weitergegeben werden.
			\tabularnewline
			Facebook &
			TODO
			\tabularnewline
		\tableend
	\end{tabularx}
	\caption{Aktoren}
\end{table}

% Use Case 1
\subsection{UC1: Anmelden}
\begin{table}[H]
	\tablestyle
	\tablealtcolored
	\begin{tabularx}{\textwidth}{lX}
		\tablebody
			\textit{Use Case Name} &
			UC1: Anmelden
			\tabularnewline
			\textit{Scope} &
			Roomies
			\tabularnewline
			\textit{Mapped Requirement} &
			-
			\tabularnewline
			\textit{Primary Actor} &
			Bewohner
			\tabularnewline
			\textit{Secondary Actor} &
			Facebook
			\tabularnewline
			\textit{Story} &
			Der Bewohner startet Roomies. Hier zeigt ihm das System das Anmeldeformular. Der Benutzer meldet sich mittels seines Facebook-Logins an. Das System überprüft die Daten. Sind sie gültig wird der Benutzer auf die Aufgabenseite weitergeleitet.
			\tabularnewline
		\tableend
	\end{tabularx}
	\caption{UC1: Anmelden}
\end{table}


% Use Case 2
\subsection{UC2: WG erstellen}
\begin{table}[H]
	\tablestyle
	\tablealtcolored
	\begin{tabularx}{\textwidth}{lX}
		\tablebody
			\textit{Use Case Name} &
			UC2: WG erstellen
			\tabularnewline
			\textit{Scope} &
			Roomies
			\tabularnewline
			\textit{Mapped Requirement} &
			F1
			\tabularnewline
			\textit{Primary Actor} &
			Bewohner
			\tabularnewline
			\textit{Story} &
			Ein Bewohner hat die Möglichkeit eine WG zu erstellen, falls er noch zu Keiner angehört. Hierfür wählt der Bewohner die Option 'WG erstellen'. Das System leitet ihn auf das entsprechende Formular und fordert den Bewohner die WG-Daten einzugeben. Nachdem der Bewohner diese eingegeben hat, überprüft das System die Gültigkeit der Daten und leitet der Bewohner auf die Aufgabenseite der WG weiter. Ebenfalls die Rolle WG-Verwalter werden dem Bewohner automatisch zugeteilt.
			\tabularnewline
		\tableend
	\end{tabularx}
	\caption{UC2: WG erstellen}
\end{table}


% Use Case 3
\subsection{UC3: WG beitreten}
\begin{table}[H]
	\tablestyle
	\tablealtcolored
	\begin{tabularx}{\textwidth}{lX}
		\tablebody
			\textit{Use Case Name} &
			UC3: WG beitreten
			\tabularnewline
			\textit{Scope} &
			Roomies
			\tabularnewline
			\textit{Mapped Requirement} &
			F2
			\tabularnewline
			\textit{Primary Actor} &
			Bewohner
			\tabularnewline
			\textit{Story} &
			Um in einer WG beizutreten, muss der Link zur Einladung dem 'zukünftigen' Bewohner bekannt sein. Ein solcher Link wird vom System erzeugt. Die Verbreitung jedoch, ist völlig dem WG-Verwalter überlassen und ist nicht Teil der Anwendung. Hat ein Bewohner den Link geöffnet, wird vom System eine Bestätigung gefordert. Bestätigt der Bewohner diese, wird er als Bewohner dessen WG 'registriert' und zur Aufgabenseite der WG weitergeleitet.
			\tabularnewline
		\tableend
	\end{tabularx}
	\caption{UC3: WG beitreten}
\end{table}


% Use Case 4
\subsection{UC4: WG verlassen}
\begin{table}[H]
	\tablestyle
	\tablealtcolored
	\begin{tabularx}{\textwidth}{lX}
		\tablebody
			\textit{Use Case Name} &
			UC4: WG verlassen
			\tabularnewline
			\textit{Scope} &
			Roomies
			\tabularnewline
			\textit{Mapped Requirement} &
			F5
			\tabularnewline
			\textit{Primary Actor} &
			Bewohner
			\tabularnewline
			\textit{Story} &
			Wählt ein Bewohner die Option 'WG verlassen', wird er vom System gefordert dies zu Bestätigen. Nach der Bestätigung setzt das System der Bewohner als inaktiv und leitet der 'Ex'-Bewohner auf eine 'Aufwiedersehen-Seite' weiter. \newline
			Hat der Bewohner, als einziger, die Rolle 'WG-Verwalter', so muss er, vor dem inaktiv Setzen, seine Rolle einem anderen Bewohner übetragen. \newline//TODO: neuer UC???
			\tabularnewline
		\tableend
	\end{tabularx}
	\caption{UC4: WG verlassen}
\end{table}


% Use Case 5
\subsection{UC5: Aufgabe erstellen}
\begin{table}[H]
	\tablestyle
	\tablealtcolored
	\begin{tabularx}{\textwidth}{lX}
		\tablebody
			\textit{Use Case Name} &
			UC5: Aufgabe erstellen
			\tabularnewline
			\textit{Scope} &
			Roomies
			\tabularnewline
			\textit{Mapped Requirement} &
			F3
			\tabularnewline
			\textit{Primary Actor} &
			Bewohner
			\tabularnewline
			\textit{Story} &
			Bewohner wählt die Option 'Aufgabe erstellen'. Das System zeigt das zugehörige Formular. Nachdem der Bewohner es ausgefüllt hat, überprüft das System die Daten, speichert es und leitet der Bewohner zurück auf die Aufgabenseite.
			\tabularnewline
		\tableend
	\end{tabularx}
	\caption{UC5: Aufgabe erstellen}
\end{table}


% Use Case 6
\subsection{UC6: Aufgabe bearbeiten}
\begin{table}[H]
	\tablestyle
	\tablealtcolored
	\begin{tabularx}{\textwidth}{lX}
		\tablebody
			\textit{Use Case Name} &
			UC6: Aufgabe bearbeiten
			\tabularnewline
			\textit{Scope} &
			Roomies
			\tabularnewline
			\textit{Mapped Requirement} &
			F6
			\tabularnewline
			\textit{Primary Actor} &
			Bewohner
			\tabularnewline
			\textit{Story} &
			Der Bewohner wählt die Aufgabe, welche beabeitet werden soll. Parallel zum Use Case 5: Aufgabe erstellen wird ein Formular dargestellt, jedoch mit den bereits eingefügten Daten der Aufgabe. Der Bewohner ändert die Daten. Das System überprüft diese und leitet dann der Beewohne auf die Aufgabenliste.
			\tabularnewline
		\tableend
	\end{tabularx}
	\caption{UC6: Aufgabe bearbeiten}
\end{table}


% Use Case 7
\subsection{UC7: Aufgabe erledigen}
\begin{table}[H]
	\tablestyle
	\tablealtcolored
	\begin{tabularx}{\textwidth}{lX}
		\tablebody
			\textit{Use Case Name} &
			UC7: Aufgabe erledigen
			\tabularnewline
			\textit{Scope} &
			Roomies
			\tabularnewline
			\textit{Mapped Requirement} &
			F4
			\tabularnewline
			\textit{Primary Actor} &
			Bewohner
			\tabularnewline
			\textit{Story} &
			In der Aufgabenliste kann ein Bewohner eine Aufgabe als erledigt setzten. Hierfür wählt er für die entsprechende Aufgabe die Option 'erledigt'. Das System setzt den Bewohner für die Eigenschaft 'Erledigt durch' und blendet die Aufgabe aus.
			\tabularnewline
		\tableend
	\end{tabularx}
	\caption{UC7: Aufgabe erledigen}
\end{table}


% Use Case 8
\subsection{UC8: Rangliste anzeigen}
\begin{table}[H]
	\tablestyle
	\tablealtcolored
	\begin{tabularx}{\textwidth}{lX}
		\tablebody
			\textit{Use Case Name} &
			UC8: Rangliste anzeigen
			\tabularnewline
			\textit{Scope} &
			Roomies
			\tabularnewline
			\textit{Mapped Requirement} &
			F7
			\tabularnewline
			\textit{Primary Actor} &
			Bewohner
			\tabularnewline
			\textit{Story} &
			Der Bewohner wählt die Option 'Rangliste anzeigen'. Das System zeigt grafisch eine Rangliste aller Bewohner der WG. \newline //TODO: tbd (Filter, Wall of Shame/Fame)
			\tabularnewline
		\tableend
	\end{tabularx}
	\caption{UC8: Rangliste anzeigen}
\end{table}


% Use Case 9
\subsection{UC9: WG auflösen}
\begin{table}[H]
	\tablestyle
	\tablealtcolored
	\begin{tabularx}{\textwidth}{lX}
		\tablebody
			\textit{Use Case Name} &
			UC9: WG auflösen
			\tabularnewline
			\textit{Scope} &
			Roomies
			\tabularnewline
			\textit{Mapped Requirement} &
			F9
			\tabularnewline
			\textit{Primary Actor} &
			WG-Verwalter
			\tabularnewline
			\textit{Story} &
			Als WG-Verwalter hat man die Möglichkeit eine WG aufzulösen. Wird diese Option gewählt, so wird vom WG-Verwalter verlangt dies zu bestätigen. Danach setzt das System alle Bewohner der WG inaktiv und die WG selbst als inaktiv. Der WG-Verwalter wird auf die Startseite weitergeleitet.
			\tabularnewline
		\tableend
	\end{tabularx}
	\caption{UC9: WG auflösen}
\end{table}


% Use Case 10
\subsection{UC10: Benutzer verwalten //TODO: UC aufteilen???}
\begin{table}[H]
	\tablestyle
	\tablealtcolored
	\begin{tabularx}{\textwidth}{lX}
		\tablebody
			\textit{Use Case Name} &
			UC10: Benutzer verwalten
			\tabularnewline
			\textit{Scope} &
			Roomies
			\tabularnewline
			\textit{Mapped Requirement} &
			F10
			\tabularnewline
			\textit{Primary Actor} &
			WG-Verwalter
			\tabularnewline
			\textit{Story} &
			TODO
			\tabularnewline
		\tableend
	\end{tabularx}
	\caption{UC10: Benutzer verwalten}
\end{table}


% Use Case 11
\subsection{UC11: auf Social Media Plattform teilen}
\begin{table}[H]
	\tablestyle
	\tablealtcolored
	\begin{tabularx}{\textwidth}{lX}
		\tablebody
			\textit{Use Case Name} &
			UC11: auf Social Media Plattform teilen
			\tabularnewline
			\textit{Scope} &
			Roomies
			\tabularnewline
			\textit{Mapped Requirement} &
			F11
			\tabularnewline
			\textit{Primary Actor} &
			Bewohner
			\tabularnewline
			\textit{Story} &
			TODO
			\tabularnewline
		\tableend
	\end{tabularx}
	\caption{UC11: auf Social Media Plattform teilen}
\end{table}
