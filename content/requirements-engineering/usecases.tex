\newpage

\section{Use Cases}

\begin{figure}[ht!]
	\centering{
		\resizebox{0.9\textwidth}{!} {
			\begin{tikzpicture}
				\begin{umlsystem}[x=5]{Roomies}

					\umlusecase[y=0]{UC1: Anmelden}
					\umlusecase[y=-2]{UC2: WG erstellen}
					\umlusecase[y=-4]{UC3: WG beitreten}
					\umlusecase[y=-6]{UC4: WG verlassen}
					\umlusecase[y=-8]{UC5: Aufgabe erstellen}
					\umlusecase[y=-10]{UC6: Aufgabe bearbeiten}
					\umlusecase[y=-12]{UC7: Aufgabe erledigen}
					\umlusecase[y=-14]{UC8: Rangliste anzeigen}
					\umlusecase[y=-16]{UC9: WG auflösen}
					\umlusecase[y=-18]{UC10: Benutzer verwalten}
					\umlusecase[x=8, y=-8, width=2cm]{UC11: auf Social Media Plattform teilen}

				\end{umlsystem}

				\umlactor[y=-7]{Bewohner}
				\umlactor[y=-16]{Administrator}

				\umlinherit{Administrator}{Bewohner}

				\umlextend{usecase-11}{usecase-3}
				\umlextend{usecase-11}{usecase-7}
				\umlextend{usecase-11}{usecase-8}

				\umlassoc{Bewohner}{usecase-1}
				\umlassoc{Bewohner}{usecase-2}
				\umlassoc{Bewohner}{usecase-3}
				\umlassoc{Bewohner}{usecase-4}
				\umlassoc{Bewohner}{usecase-5}
				\umlassoc{Bewohner}{usecase-6}
				\umlassoc{Bewohner}{usecase-7}
				\umlassoc{Bewohner}{usecase-8}
				\umlassoc{Bewohner}{usecase-11}

				\umlassoc{Administrator}{usecase-9}
				\umlassoc{Administrator}{usecase-10}

				\umlnote[x=11, y=-1]{usecase-1}{Die erste Anmeldung wird als eine Registrierung interpretiert}
				\umlnote[x=11, y=-4]{usecase-3}{Eine WG kann nur über eine Einladung (Link) beigetreten werden}
			\end{tikzpicture}
		}
	}

	\caption{Use Case Diagramm}
\end{figure}

% Use Case 1
\subsection{UC1: Anmelden}
\begin{table}[H]
	\tablestyle
	\tablealtcolored
	\begin{tabularx}{\textwidth}{lX}
		\tablebody
			Use Case Name &
			UC1: Anmelden
			\tabularnewline
			Scope &
			Roomies
			\tabularnewline
			Primary Actor &
			Bewohner
			\tabularnewline
			Stakeholders and Interest &
			Der Bewohner möchte sich in das System anmelden.
			\tabularnewline
			Preconditions &
			\begin{itemize}
				\item Der Bewohner besitzt einen Account bei Facebook.
			\end{itemize}
			\tabularnewline
			Postconditions &
			\begin{itemize}
				\item Der Bewohner ist in das System eingelogt.
			\end{itemize}
			\tabularnewline
			Main Success Scenario &
			\begin{enumerate}
				\item TODO: login mit FB
				\item tbd
			\end{enumerate}
			\tabularnewline
			Extensions &
			-
			\tabularnewline
			Frequency of Occurence &
			Oft
			\tabularnewline
		\tableend
	\end{tabularx}
	\caption{UC1: Anmelden}
\end{table}


% Use Case 2
\subsection{UC2: WG erstellen}
\begin{table}[H]
	\tablestyle
	\tablealtcolored
	\begin{tabularx}{\textwidth}{lX}
		\tablebody
			Use Case Name &
			UC2: WG erstellen
			\tabularnewline
			Scope &
			Roomies
			\tabularnewline
			Primary Actor &
			Bewohner
			\tabularnewline
			Stakeholders and Interest &
			Der Bewohner möchte eine WG erstellen.
			\tabularnewline
			Preconditions &
			\begin{itemize}
				\item Der Bewohner ist keiner WG zugeteilt.
				\item Der Bewohner ist in das System angemeldet.
			\end{itemize}
			\tabularnewline
			Postconditions &
			\begin{itemize}
				\item Neue WG ist erstellt.
				\item Der Bewohner ist der WG zugeteilt.
				\item Der Bewohner ist der Administrator der WG.
			\end{itemize}
			\tabularnewline
			Main Success Scenario &
			\begin{enumerate}
				\item Bewohner klickt auf 'WG erstellen'
				\item System zeigt das WG-erstell-Formular an
				\item Bewohner trägt der Name der WG ein
				\item TODO: weitere infos?
				\item Bewohner klickt auf 'erstellen'
				\item System zeigt Aufgaben der neu erstellten WG
			\end{enumerate}
			\tabularnewline
			Extensions &
			\begin{itemize}
				\item[2a.] Bewohner gibt keinen Namen an
				\begin{itemize}
					\item[6.] System fordert (Fehlermeldung) Bewohner einen Namen für die WG einzugeben
				\end{itemize}
			\end{itemize}
			\tabularnewline
			Frequency of Occurence &
			Selten
			\tabularnewline
		\tableend
	\end{tabularx}
	\caption{UC2: WG erstellen}
\end{table}


% Use Case 3
\subsection{UC3: WG beitreten}
\begin{table}[H]
	\tablestyle
	\tablealtcolored
	\begin{tabularx}{\textwidth}{lX}
		\tablebody
			Use Case Name &
			UC3: WG beitreten
			\tabularnewline
			Scope &
			Roomies
			\tabularnewline
			Primary Actor &
			Bewohner
			\tabularnewline
			Stakeholders and Interest &
			Der Bewohner möchte einer WG beitreten
			\tabularnewline
			Preconditions &
			\begin{itemize}
				\item Der Bewohner besitzt einen Account bei Facebook.
				\item Der Bewohner ist die Internetadresse der Einladung bekannt.
				\item Der Bewohner ist in keine WG aktiv.
			\end{itemize}
			\tabularnewline
			Postconditions &
			\begin{itemize}
				\item Der Bewohner ist in der WG angemeldet
				\item Der Bewohner ist in der WG aktiv.
			\end{itemize}
			\tabularnewline
			Main Success Scenario &
			\begin{enumerate}
				\item System zeigt Einladugsformular
				\item Bewohner klickt auf 'WG beitreten'
				\item System zeigt Aufgaben der WG
			\end{enumerate}
			\tabularnewline
			Extensions &
			-
			\tabularnewline
			Frequency of Occurence &
			Gelegentlich
			\tabularnewline
		\tableend
	\end{tabularx}
	\caption{UC3: WG beitreten}
\end{table}


% Use Case 4
\subsection{UC4: WG verlassen}
\begin{table}[H]
	\tablestyle
	\tablealtcolored
	\begin{tabularx}{\textwidth}{lX}
		\tablebody
			Use Case Name &
			UC4: WG verlassen
			\tabularnewline
			Scope &
			Roomies
			\tabularnewline
			Primary Actor &
			Bewohner
			\tabularnewline
			Stakeholders and Interest &
			Der Bewohner möchte einer WG beitreten
			\tabularnewline
			Preconditions &
			\begin{itemize}
				\item Der Bewohner ist kein Administrator der WG.
			\end{itemize}
			\tabularnewline
			Postconditions &
			\begin{itemize}
				\item Der Bewohner ist in der WG inaktiv
			\end{itemize}
			\tabularnewline
			Main Success Scenario &
			\begin{enumerate}
				\item Bewohner klickt auf 'Einstellungen'
				\item System zeigt die Einstellungen vom Bewohner
				\item Bewohner klickt auf 'WG verlassen'
				\item System fordert eine Bestätigung des Bewohners
				\item Bewohner bestätigt das Verlassen
				\item System zeigt 'Auf Wiedersehen' Meldung
			\end{enumerate}
			\tabularnewline
			Extensions &
			\begin{itemize}
				\item[5a.] Bewohner verneigt das Verlassen
				\begin{itemize}
					\item[6] System zeigt die Einstellungen vom Bewohner
				\end{itemize}
			\end{itemize}
			\tabularnewline
			Frequency of Occurence &
			Gelegentlich
			\tabularnewline
		\tableend
	\end{tabularx}
	\caption{UC4: WG verlassen}
\end{table}


% Use Case 5
\subsection{UC5: Aufgabe erstellen}
\begin{table}[H]
	\tablestyle
	\tablealtcolored
	\begin{tabularx}{\textwidth}{lX}
		\tablebody
			Use Case Name &
			UC5: Aufgabe erstellen
			\tabularnewline
			Scope &
			Roomies
			\tabularnewline
			Primary Actor &
			Bewohner
			\tabularnewline
			Stakeholders and Interest &
			Der Bewohner möchte eine Aufgabe der WG hinzufügen
			\tabularnewline
			Preconditions &
			\begin{itemize}
				\item Der Bewohner ist angemeldet
			\end{itemize}
			\tabularnewline
			Postconditions &
			\begin{itemize}
				\item Die Aufgabe wurde erstellt
			\end{itemize}
			\tabularnewline
			Main Success Scenario &
			\begin{enumerate}
				\item Bewohner klickt auf 'Aufgabe hinzufügen'
				\item System zeigt 'Neue Aufgabe'-Formular
				\item Bewohner gibt Name der Aufgabe ein.
				\item //TODO: fill some more data?
				\item Bewohner klickt auf 'Speichern'
				\item System zeigt Erfolgsmeldung
			\end{enumerate}
			\tabularnewline
			Extensions &
			\begin{itemize}
				\item[5a.] Bewohner klickt auf ''
			\end{itemize}
			\tabularnewline
			Frequency of Occurence &
			Oft
			\tabularnewline
		\tableend
	\end{tabularx}
	\caption{UC5: Aufgabe erstellen}
\end{table}


% Use Case 6
\subsection{UC6: Aufgabe bearbeiten}
\begin{table}[H]
	\tablestyle
	\tablealtcolored
	\begin{tabularx}{\textwidth}{lX}
		\tablebody
			Use Case Name &
			UC6: Aufgabe bearbeiten
			\tabularnewline
			Scope &
			Roomies
			\tabularnewline
			Primary Actor &
			Bewohner
			\tabularnewline
			Stakeholders and Interest &
			Der Bewohner möchte eine Aufgabe der WG bearbeiten
			\tabularnewline
			Preconditions &
			\begin{itemize}
				\item Der Bewohner ist angemeldet
				\item Aufgabe ist vorhanden
				\item Aufgabe ist nicht als 'Erledigt' markiert
			\end{itemize}
			\tabularnewline
			Postconditions &
			\begin{itemize}
				\item Geänderte Properties wurden gespeichert
			\end{itemize}
			\tabularnewline
			Main Success Scenario &
			\begin{enumerate}
				\item Bewohner wählt auf Aufgabe aus
				\item System zeigt die Eigenschaften der Aufgabe an
				\item Bewohner verändert die Eigenschften
				\item Bewohner klickt auf 'Speichern'
				\item System zeigt Erfolgsmeldung
			\end{enumerate}
			\tabularnewline
			Extensions &
			-
			\tabularnewline
			Frequency of Occurence &
			Oft
			\tabularnewline
		\tableend
	\end{tabularx}
	\caption{UC6: Aufgabe bearbeiten}
\end{table}


% Use Case 7
\subsection{UC7: Aufgabe erledigen}
\begin{table}[H]
	\tablestyle
	\tablealtcolored
	\begin{tabularx}{\textwidth}{lX}
		\tablebody
			Use Case Name &
			UC7: Aufgabe erledigen
			\tabularnewline
			Scope &
			Roomies
			\tabularnewline
			Primary Actor &
			Bewohner
			\tabularnewline
			Stakeholders and Interest &
			Der Bewohner möchte eine Aufgabe der WG als 'erledigt' markieren.
			\tabularnewline
			Preconditions &
			\begin{itemize}
				\item Der Bewohner ist angemeldet
				\item Aufgabe ist vorhanden
			\end{itemize}
			\tabularnewline
			Postconditions &
			\begin{itemize}
				\item Die Aufgabe ist als 'erledigt' markiert.
			\end{itemize}
			\tabularnewline
			Main Success Scenario &
			\begin{enumerate}
				\item Bewohner klickt auf 'Erledigt' bei einer ausgewählten Aufgabe
				\item System blendet die Aufgabe aus der Aufgabenliste aus
			\end{enumerate}
			\tabularnewline
			Extensions &
			-
			\tabularnewline
			Frequency of Occurence &
			Oft
			\tabularnewline
		\tableend
	\end{tabularx}
	\caption{UC7: Aufgabe erledigen}
\end{table}