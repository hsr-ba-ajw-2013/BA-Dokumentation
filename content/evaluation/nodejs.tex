\section{\gls{nodejs}}

\subsection*{Web-Frameworks}

\subsubsection*{Mögliche Auswahl}
\begin{table}[H]
\tablestyle
\tablealtcolored
\begin{tabularx}{\textwidth}{lXlX}
\tableheadcolor
	\tablehead Name &
	\tablehead Beschreibung &
	\tablehead Pro/Kontra &
	\tablehead Referenz \tabularnewline
\tablebody
	\textit{Express.js} &
		Minimales web application framework für single und multi-page sowie hybrid applikationen &
		\parbox{0.3\textwidth}{
		\begin{itemize}
			\item[+] Leichtgewichtige Abstraktion über Node.js native
			\item[+] Sehr einfach zu sehen was passiert
			\item[-] Ist kein fertiges MVC-Framework
		\end{itemize}}
		&
		\cite{Expressjs} \tabularnewline
	\textit{Geddy} &
		Simples, strukturiertes web framework &
		\parbox{0.3\textwidth}{
		\begin{itemize}
			\item[+] MVC-Framework ähnlich wie Ruby on Rails \cite{RoR}
			\item[+] Sehr einfach zu sehen was passiert
			\item[-] Ist kein fertiges MVC-Framework
		\end{itemize}}
		&
		\cite{Geddy} \tabularnewline
	\textit{Tower.js} &
		Full Stack Webframework für Node.js und den Browser &
		Inhalt &
		\cite{Towerjs} \tabularnewline
\tableend
\end{tabularx}
\caption{Web-Frameworks in Node.js}
\end{table}