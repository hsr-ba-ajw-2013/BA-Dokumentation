\section{Node.js}

\gls{nodejs} ..TODO

\subsection*{Web-Frameworks}

\subsubsection*{Mögliche Auswahl}
\begin{table}[H]
\tablestyle
\tablealtcolored
\begin{tabularx}{\textwidth}{b{0.1\textwidth}b{0.3\textwidth}m{0.6\textwidth}}
\tableheadcolor
	\tablehead Name &
	\tablehead Beschreibung &
	\tablehead Pro/Kontra \tabularnewline
\tablebody
	\vtop{\textit{Express.js \cite[Express.js]{Expressjs}}} &
	\vtop{Minimales web application framework für single und multi-page sowie hybrid applikationen} &
	\vtop{\parbox{0.6\textwidth}{
	\begin{itemize}
		\item[+] Leichtgewichtige Abstraktion über Connect \cite{connect}
		\item[+] Sehr einfach zu sehen was passiert
		\item[+] Für anbindung an \gls{REST}-Schnittstelle im Backend \\ quasi optimal
		\item[-] Kein ORM
	\end{itemize}}}
	\tabularnewline

	\vtop{\textit{Geddy \cite{Geddy}}} &
	\vtop{Simples, strukturiertes web framework} &
	\vtop{\parbox{0.5\textwidth}{
	\begin{itemize}
		\item[+] MVC-Framework ähnlich wie Ruby on Rails \cite{RoR}
		% TODO: Glossary
		\item[+] ORM mit Adapter u.a. für Postgres, in-memory, MongoDB und Riak
	\end{itemize}}}
	\tabularnewline

	\vtop{\textit{Sails \cite{sails}}} &
	\vtop{Realtime MVC Framework for Node.js} &
	\vtop{\parbox{0.5\textwidth}{
	\begin{itemize}
		\item[+] MVC-Framework ähnlich wie Ruby on Rails \cite{RoR}
		\item[+] Optimiert für den data-oriented style der web applikations-entwicklung
	\end{itemize}}}
	\tabularnewline
\tableend
\end{tabularx}
\caption{Web-Frameworks in Node.js}
\end{table}