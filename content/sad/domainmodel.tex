\section{Domainmodel}

Das folgende Domainmodel in Abbildung \ref{fig:domainmodel} zeigt eine Übersicht über alle in der Problemdomäne enthaltenen Objekte. Zu jedem dieser ist im Anschluss eine genauere Erklärung zu finden.

\begin{figure}[ht!]
	\centering{
		\resizebox{0.9\textwidth}{!} {
			\begin{tikzpicture}
				% HACK: don't display guillemots on stereotypes
				\newcommand\myguillemotleft\guillemotleft
				\newcommand\myguillemotright\guillemotright
				\renewcommand\guillemotleft{}
				\renewcommand\guillemotright{}

				% row by row
				\umlclass[x=15,y=-3]{Community}{
					shareLink\\
					enabled
				}{
				}
				\umlemptyclass[x=5,y=-3]{Rule}

				% \umlemptyclass[x=5,y=-7]{Role}
				\umlclass[x=10,y=-7]{Resident}{
					isAdmin\\
					enabled
				}{
				}
				\umlemptyclass[x=5,y=-7]{AchievementDefinition}

				\umlclass[x=15,y=-11]{Task}{rewardPoints}{}
				\umlclass[x=5,y=-11]{Achievement}{rewardedAt:Date}{}

				% \umlemptyclass[x=3,y=-11]{Admin}
				% \umlemptyclass[x=7,y=-11]{Resident}

				% \umlemptyclass[x=7,y=-15]{Community-Admin}

				% notes
				\umlnote[x=9,y=-3]{Rule}{- Timebased\\*
										- Completion based\\*
										- Points based\\[2mm]
										Rules können gesammten Datenbestand auswerten.}

				% relation
				\umlassoc[mult1=1,mult2=1..*,stereo=<has]{Rule}{AchievementDefinition}
				\umlassoc[mult1=1,mult2=0..*,stereo=defines>]{AchievementDefinition}{Achievement}
				\umlassoc[mult1=0..*,mult2=1,stereo=belongs to>]{Achievement}{Resident}
				\umlassoc[mult1=1,mult2=0..*,stereo=<belongs to]{Community}{Task}
				\umlassoc[mult1=1..*,mult2=1,stereo=belongs to>]{Resident}{Community}
				%\umlassoc[mult1=*,mult2=*,stereo=<has]{Role}{Resident}
				\umlassoc[mult1=1,mult2=0..*,stereo=creates>]{Resident}{Task}
				\umlassoc[geometry=|-,arm1=-2cm,mult1=1,mult2=0..*,pos2=1.8,stereo=fulfill>,pos stereo=1.5]{Resident}{Task}
				% \umlassoc[geometry=-|-,arm1=11cm,mult1=1..*,mult2=1,pos1=0.05,pos2=2.8,stereo=<has]{Resident}{Community}

				% inherit
				% \umlinherit[geometry=|-|]{Resident}{Role}
				% \umlinherit[]{Community-Admin}{Resident}
				% \umlinherit[geometry=|-|]{Admin}{Role}

				% reset guillemot resets

				\renewcommand\guillemotleft{\myguillemotleft}
				\renewcommand\guillemotright{\myguillemotright}
			\end{tikzpicture}
		}
	}
	\caption{Domainmodel}
	\label{fig:domainmodel}
\end{figure}

\subsection*{Achievement (Erfolg)}
Achievements werden an einen Resident vergeben, sobald diese bestimmte Regeln ausreichend befriedigt haben.

\subsection*{Rule (Regel)}
Rules beschreiben, welche Aktionen oder Verhalten notwendig sind, damit ein Resident ein bestimmtes Achievment erhalten kann.

Eine Rule kann verschiedene Ausprägungen haben:

\begin{itemize}
	\item Zeitbasiert\\Beispiel: Ein Resident ist bereits zwei Monate Teil einer Community.
	\item Punktebasiert\\
	Beispiel: Ein Resident hat durch das erledigen von Tasks 50 Punkte gesammelt.
	\item Spezifische Aktionen\\
	Beispiel: Ein Resident hat 10 Tasks erledigt.
\end{itemize}

\subsection*{AchievementDefinition (Erfolgsdefinition)}
Die AchievementDefinition verknüpft ein Achievment mit den umzusetzenden Rules.

\subsection*{Community (\gls{WG})}
Eine Community ist eine Wohngemeinschaft in welcher mehrere Residents wohnen können. Zudem gehören Tasks immer zu einer spezifischen Community.

\subsection*{Resident (Bewohner)}
Residents sind Bewohner einer Community und die eigentlichen Benutzer des Systems. Ein Resident kann über die \emph{isAdmin}-Eigenschaft erweiterte Berechtigungen zum Verwalten von Communities sowie deren Tasks und Residents erhalten.

\subsection*{Task (Aufgabe)}
Eine Community führt eine Liste von zu erledigenden Aufgaben, den Tasks. Ein Resident einer Community kann Tasks erstellen, bearbeiten, löschen und erledigen. Über das Erledigen von Tasks erhält der entsprechende Resident Punkte, welche ihn auf der Community-internen Rangliste emporsteigen lassen.