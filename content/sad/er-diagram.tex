\section{Entity-Relationship Diagramm}
\label{sec:erdiagramm}

Folgendes ER-Diagram repräsentiert die Abbildung des Domainmodels auf Datenbankebene.

Für beide Modelle ist die selbe Terminologie gültig.

\begin{figure}[ht!]
	\begin{tikzpicture}
		% row by row
		\umlclass[x=20,y=0]{Community}{
			id : Integer \\
			name : String \\
			shareLink : String \\
			createdAt : Date \\
			updatedAt : Date \\
			enabled: Boolean
		}{
			PRIMARY(id)
		}

		\umlclass[x=12,y=-6]{Resident}{
			id: Integer \\
			facebookId: Integer \\
			communityId: Integer \\
			name: String \\
			createdAt: Date \\
			updatedAt: Date \\
			enabled: Boolean \\
			isAdmin: Boolean
		}{
			PRIMARY(id) \\
			UNIQUE(facebookId)
		}

		\umlclass[x=12,y=-12]{Achievement}{
			id: Integer \\
			type: Enum \\
			residentId: Integer \\
			createdAt: Date
		}{
			PRIMARY(id)
		}

		\umlclass[x=20,y=-6]{Task}{
			id: Integer \\
			communityId: Integer \\
			name: String \\
			description: Text \\
			reward: Integer \\
			fullfilledAt: Date \\
			createdAt: Date \\
			updatedAt: Date \\
			creatorId: Resident \\[1mm]
			fulfillorId: Resident
		}{
			PRIMARY(id)
		}

		% relation
		\umlassoc[mult1=0..*,mult2=1]{Achievement}{Resident}
		\umlassoc[mult1=1,mult2=0..*]{Community}{Task}
		\umlassoc[geometry=|-, mult1=1..*,mult2=1, pos2=1.9]{Resident}{Community}
		% creator
		\umlassoc[geometry=-|-, arm1=3, anchor1=40, anchor2=-28, pos2=2.0, mult2=1..*]{Resident}{Task}
		% fulfillor
		\umlassoc[geometry=-|-, anchor1=40, anchor2=-38, pos2=2.25, mult1=1,mult2=0..*]{Resident}{Task}
	\end{tikzpicture}

	\caption{Entity-Relationship Diagramm}
\end{figure}