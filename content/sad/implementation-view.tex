\section{Implementations-Sicht}
\subsection{Code-Struktur}
In Node.JS ist es unter den erfahreneren (\cite{TJH_ComponentStructure}, \cite{IZS_ComponentStructure}) Entwicklern relativ üblich eine Applikation in mehr oder weniger komplett eigenständige Komponenten zu unterteilen.\\
Dies sieht man u.a. auch an den installierbaren Komponenten auf NPM \cite{NPM}, in welchen es für sehr kleine Logiken eigene Komponenten hat.\\
Die Beispielapplikation ``Roomies'' folgt diesem Beispiel ebenfalls. Möglichst jede Komponente (z.B. auch ein Controller) ist abgetrennt von der Applikation und wird von dieser aufgerufen um eigene Initialisierungen zu machen.\\[0.5mm]

Dies veranschaulicht die folgende Ordner-Struktur:
\begin{verbatim}
├── src
│   ├── lib
│   │   ├── cluster
│   │   ├── community
│   │   │   └── views
│   │   │   └── controller.js
│   │   │   └── index.js
│   │   │   └── model.js
│   │   │   └── test.js
│   │   ├── facebook-channel
│   │   │   └── views
│   │   ├── home
│   │   │   └── views
│   │   ├── login
│   │   │   └── views
│   │   ├── middleware
│   │   ├── resident
│   │   │   └── views
│   │   └── task
│   │       └── views
│   ├── public
│   │   ├── font
│   │   ├── images
│   │   ├── javascripts
│   │   │   └── lib
│   │   └── stylesheets
│   └── shared
│       ├── exceptions
│       ├── layouts
│       ├── locales
│       ├── partials
│       ├── policies
│       ├── sass
│       │   └── vendor
│       ├── test
│       ├── utils
│       ├── validators
│       └── views
├── test
└── travis
\end{verbatim}

Jede Komponente unter lib/ ist in sich abgeschlossen und ist über die Datei ``index.js''
ansprechbar.