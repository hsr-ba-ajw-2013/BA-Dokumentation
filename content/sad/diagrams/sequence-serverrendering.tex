\begin{figure}[H]
	\centering{
		\resizebox{0.9\textwidth}{!} {
			\begin{tikzpicture}
				\begin{umlseqdiag}
					\umlactor[class=Browser]{client}
					\umlboundary[class=ExpressJS]{app}
					\umlobject[class=Router]{router}
					\umlobject[class=View]{view}
					\umlobject[class=Model]{model}
					\umlobject[class=ApiAdapter]{apiAdapter}
					\umlobject[class=DataStore]{dataStore}

					\begin{umlcall}[op={Click Menu Item, triggers GET Request}, return=HTML]{client}{expressJs}
						\begin{umlcall}[op=route(), with return]{expressJs}{router}
							\begin{umlcallself}[op=render()]{router}
								\begin{umlcallself}[op=initDOM()]{router}
								\end{umlcallself}
								\begin{umlcallself}[op=renderMainView()]{router}
								\end{umlcallself}

								\begin{umlcallself}[op=renderView()]{router}
									\begin{umlcall}[op=beforeRender(), with return]{router}{view}
										\begin{umlcall}[op={fetch()}, return=success()]{view}{model}
											\begin{umlcall}[op={sync()}, return=success()]{model}{apiAdapter}
											\end{umlcall}
										\end{umlcall}
									\end{umlcall}

									\begin{umlcall}[op=renderView(), return=HTML]{router}{view}
									\end{umlcall}

									\begin{umlcall}[op=afterRender(), with return]{router}{view}
									\end{umlcall}
								\end{umlcallself}

								\begin{umlcall}[op=toJSON(), with return]{router}{dataStore}
								\end{umlcall}

								\begin{umlcallself}[op=writeHTTPResponse()]{router}
								\end{umlcallself}
							\end{umlcallself}
						\end{umlcall}
					\end{umlcall}
				\end{umlseqdiag}
			\end{tikzpicture}
		}
	}
	\caption{Sequenzdiagramm: Rendering und Event-Verarbeitung auf dem Server}
	\label{dig:serverrendering}
\end{figure}