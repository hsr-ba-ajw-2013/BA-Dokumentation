\begin{figure}[H]
	\centering{
		\resizebox{0.9\textwidth}{!} {
			\begin{tikzpicture}
				\begin{umlseqdiag}
					\umlactor[class=Caller]{caller}
					\umlboundary[class=ExpressJS]{app}
					\umlobject[class=APIAdapter]{apiAdapter}
					\umlmulti[class=Controllers]{controller}
					\umldatabase[class=Persistency]{db}

					\begin{umlfragment}[type=alt, label={Server-local}, inner xsep=12]
						\begin{umlcall}[op={sync()}, return={success()}, dt=8]{caller}{apiAdapter}
							\begin{umlcallself}[op={dispatchLocalApiCall()}]{apiAdapter}
							\end{umlcallself}
							\begin{umlcallself}[op={processCallbacks()}]{apiAdapter}
							\end{umlcallself}
							\begin{umlcall}[op={handler()}, return={success()}]{apiAdapter}{controller}
								\begin{umlcall}[op={consume}, with return]{controller}{db}
								\end{umlcall}
							\end{umlcall}
						\end{umlcall}

						\umlfpart[HTTP REST]

						\begin{umlcall}[op={HTTP Request}, return={JSON Response}]{caller}{expressJs}
							\begin{umlcallself}[op={Dispatch}]{expressJs}
							\end{umlcallself}
							\begin{umlcall}[op={handler()}, return={JSON Object}]{expressJs}{apiAdapter}
								\begin{umlcallself}[op={processCallbacks}]{apiAdapter}
								\end{umlcallself}
								\begin{umlcall}[op={handler()}, return={success()}]{apiAdapter}{controller}
									\begin{umlcall}[op={consume}, with return]{controller}{db}
									\end{umlcall}
								\end{umlcall}
							\end{umlcall}
						\end{umlcall}
					\end{umlfragment}

				\end{umlseqdiag}
			\end{tikzpicture}
		}
	}
	\caption{Sequenzdiagramm: APIAdapter im server-lokalen sowie HTTP REST Modus}
	\label{dig:apiAdapter}
\end{figure}