\section{Einleitung}

Die Architektur besteht aus den folgenden grundlegenden Elementen:
\begin{itemize}
	\item Backend-Applikation mit \emph{Express.js} \cite{Expressjs}
	%\item Frontend-Applikation mit \emph{Backbone.js} \cite{Backbonejs}
	\item \emph{PostgreSQL} \cite{PostgreSQL} Datenbank, welche mit dem \gls{ORM}
		\emph{Sequelize} \cite{Sequelize} angesprochen wird
\end{itemize}

\subsection{Backend-Applikation}

Die Backend-Applikation ist eine Komponenten-basierte MVC Applikation und
ermöglicht dadurch eine gute Trennung von Zuständigkeiten (Separation of
Concerns \cite{SeparationOfConcerns}) und niedrige Kopplung.

Durch die Verwendung von \emph{Express.js} \cite{Expressjs} ist es einerseits
sehr einfach, die ROCA-Richtlinien zu demonstrieren, andererseits ist das Layout
der Applikation einem komplett selber überlassen.

Nach mehreren Versuchen verschiedener Applikationsstrukturen, wurde
aufgrund von einem Artikel \cite{TJH_ComponentStructure} des Haupt-Entwicklers
von Express.js, TJ Holowaychuck \cite{TJH}, die erwähnte Komponenten-Struktur
implementiert.

% TODO: Add if done
% Um das Prinzip ``\gls{DRY}'' zu erfüllen, werden Controllers und Templates
% mit dem Frontend geteilt.

% \subsection{Frontend-Applikation}

% Die Frontend-Applikation ist ebenfalls MVC-basiert und holt die entsprechende
% Logik vom geteilten Backend-Code.
% Im Optimalfall (Der Client-Browser hat JavaScript aktiviert) wird das Backend
% nur für die Datenübertragung angesprochen.
% Alles andere passiert direkt im Browser des Clients.
% Für Suchmaschinen und Browsern mit JavaScript deaktiviert, funktioniert die
% Applikation aber ohne jegliche Probleme.

\subsection{ORM}

Das \gls{ORM} verwendet PostgreSQL \cite{PostgreSQL} als Datenbank und stellt
einfache Schnittstellen zum Abfragen oder Speichern von Daten zur Verfügung.