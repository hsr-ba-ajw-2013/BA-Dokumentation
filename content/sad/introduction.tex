\section{Einleitung}

Wie in Kapitel \ref{sec:evaluation-frameworks-nodejs}, ``\nameref{sec:evaluation-frameworks-nodejs}'' beschrieben, wurde schlussendlich Express.js \cite{Expressjs} für
die Implementation der Beispielapplikation ausgewählt.

\subsection{Code-Struktur}
In Node.JS ist es unter den erfahreneren (\cite{TJH_ComponentStructure}, \cite{IZS_ComponentStructure}) Entwicklern relativ üblich eine Applikation in mehr oder weniger komplett eigenständige Komponenten zu unterteilen.\\
Dies sieht man u.a. auch an den installierbaren Komponenten auf NPM \cite{NPM}, in welchen es für sehr kleine Logiken eigene Komponenten hat.\\
Die Beispielapplikation ``Roomies'' folgt diesem Beispiel ebenfalls. Möglichst jede Komponente (z.B. auch ein Controller) ist abgetrennt von der Applikation und wird von dieser aufgerufen um eigene Initialisierungen zu machen.\\[0.5mm]

%TODO: z.B.