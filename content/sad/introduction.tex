\section{Einleitung}

\subsection*{Frameworks}
In der Technologie Evaluation zu \nameref{sec:technology-evaluation-javascript} stachen die beiden Frameworks Express.js \cite{Expressjs}
und Sails.js \cite{sails} heraus.\\
Während Express.js zwar viel stabiler ist und weitaus am Meisten genutzt wird für Javascript Web-Applikationen, ist Sails.js aufgrund der neuen
Ideen für einen ersten Prototyp ausgewählt worden.

\subsubsection*{Sails.js Prototyp}
Um sich ein Bild von Sails.js zu machen wurde ein einfacher Prototyp \cite{SailsPrototyp} erstellt.\\[1mm]

Sails.js verwendet \gls{Scaffolding} um einerseits ein neues Projekt zu erstellen, andererseits auch um z.B. Models oder Controllers zu erstellen.
Wie im \nameref{sec:erdiagramm} beschrieben, werden u.a. ein Task und ein User Model (sowie entsprechende Tabelle) benötigt.\\
Um das \gls{ORM} zu testen, wurden diese beiden Models erstellt und verwendet.\\
Das Task-Model mittels Sails.js definiert sieht so aus:\\

\begin{lstlisting}[language=JavaScript, caption=Task Model in Sails.js]
module.exports = {

	attributes: {
		name: "string",
		description: "string",
		points: "int",
		userId: "int",
		communityId: "int",
	},

	getUser: function() {
		return User.find(this.userId);
	},

	getCommunity: function() {
		return Community.find(this.communityId);
	}

};
\end{lstlisting}

Mit einer Definition eines Models wird zusätzlich zur eigentlichen Verwendung des Models in der Applikation automatisch eine \gls{REST}-API erstellt.\\
Damit lassen sich einerseits CRUD-Operationen direkt über den Web-Browser ausführen, andererseits existiert auch die Möglichkeit Socket.IO \cite{SocketIO} zu aktivieren und Models direkt von einem offenen \gls{Websocket} verwenden.\\
Dieses Feature macht Sails.js sehr nützlich für \gls{RealTime}-Applikationen.\\[1mm]

Um nun eine effektive HTML-Seite darstellen zu können wird ein Controller sowie eine View benötigt. Dies wurde im Prototyp für Aufgaben (Tasks) implementiert.\\

\begin{lstlisting}[language=JavaScript, caption=Task Controller in Sails.js, label=lst:sailsjstaskcontroller]
var TaskController = {
	get: function(req, res) {
		var id = req.param('id');
		Task.find(id).done(function(err, task) {
			User.find(task.userId).done(function(err, user) {
				var response = {
					'task': task,
					'user': user,
					'title': task.name
				};

				if (req.acceptJson) {
					res.json(response);
				} else if(req.isAjax && req.param('partial')) {
					response['layout'] = false;
					res.view(response);
				} else {
					res.view(response);
				}
			});
		});
	}

};
module.exports = TaskController;
\end{lstlisting}

In diesem Controller wird ein Task aufgrund des GET-Parameters ``id'' geladen. Das Code-Stück zeigt die grosse Schwäche von Waterline \cite{Waterline}, dem \gls{ORM} von Sails.js. Statt dass man auf dem ``task''-Objekt direkt ``.user'' aufrufen könnte, muss man den Umweg über ``User.find()'' gehen. \\
Dies zeigt auf, dass Joins nicht unterstützt sind.\\[1mm]

Der Controller schickt, falls HTML als Content-Type mitgeschickt wird, folgendes Template zurück.

\begin{lstlisting}[language=HTML, caption=Task Template]
<div id="task-display">
	<h1>Task: <%= task.name %></h1>
	<ul>
		<li>Points: <%= task.points %></li>
		<li>Created At: <%= task.createdAt %></li>
	</ul>
	<h2>User: <%= user.name %></h2>
	<ul>
		<li>Created At: <%= user.createdAt %></li>
	</ul>
</div>
<a href="#" id="reload">Reload!</a>
<script>
	$('#reload').on('click', function() {
		$.ajax('/task/get/?id=<%= task.id %>&partial=true', {
			success: function(response) {
				var $response = $('<div class="body-mock">' + response + '</div>');
				html = $response.find('#task-display');
				$('#task-display').replaceWith(html);
			}
		});
	});
</script>
\end{lstlisting}

Mit dem einfachen Script am Schluss des Task Templates wird aufgezeigt, wie man ohne Neuladen der Seite direkt HTML ersetzen kann. Dies ist grundsätzlich Framework-unabhängig und es wurde dabei auch nicht auf ``RP14'' (Unobtrusive JavaScript) Wert gelegt.

\subsubsection*{Schlussfolgerung}

Wie bereits angemerkt ist das ORM von Sails.js nicht ausgereift. Weder Assoziationen zwischen Models \cite{SailsjsModelAssociations} noch das setzen von Indizes ist möglich.\\
Weiter ist es für das User-Model nötig, BIGINTs definieren zu können. Dies ist unumgänglich, weil die Facebook User Id 64 Bit gross ist.

Nebst dem ORM ist auch das Framework und die zugehörige Dokumentation zu wenig ausgereift. Auch die Community war zum Zeitpunkt der Evaluation (siehe das Appendix zur \nameref{sec:appendix-technology-evaluation}) sehr klein.
Alles in allem spricht relativ viel dagegen, Sails.js weiter zu verwenden. Deshalb ist ein zweiter Prototyp unter Verwendung von Express.js erstellt worden.

\subsection*{Express.js Prototyp}

Express.js \cite{Expressjs} ist ein sehr leichtgewichtiges Framework, welches mittels Connect-Middlewares \cite{connect}, sehr leicht erweitert werden kann.\\[1mm]
Damit die Eignung von Express.js überprüft werden kann, wurde ebenfalls ein einfacher Prototyp \cite{ExpressjsPrototyp} erstellt.\\
Nach der initialen Prototypen-Phase wurde der Branch allerdings mit dem Master-Branch zusammengeführt, damit der Prototyp für die eigentliche Arbeit weiterverwendet werden kann.

Der initiale Startpunkt einer Express.js Applikation ist die Datei ``app.js'' \cite{ExpressjsPrototypAppjs}.\\
Dort werden alle benutzten Middlewares registriert, die Datenbank aufgesetzt und die Routen deklariert (Bzw. die Controller registriert).\\[0.5mm]

Ein ganz einfacher Controller sieht z.B. so aus:

\begin{lstlisting}[language=JavaScript, caption=Controller in Express.js, label=lst:controllerInExpressjs]
exports.index = function(req, res){
	// first Parameter: Template File to use
	// 2nd Parameter: Context to pass to the template
	res.render('index', { title: 'Express' });
};
\end{lstlisting}

Das zugehörige Template könnte dann folgendermassen aussehen:

\begin{lstlisting}[language=HTML, caption=Template in Express.js, label=lst:templateInExpressjs]
<!DOCTYPE html>
<html>
	<head>
		<title><%= title %></title>
		<link rel='stylesheet' href='/stylesheets/style.css' />
		<%- LRScript %>
	</head>
	<body>
		<h1><%= title %></h1>
		<p>Welcome to <%= title %></p>
	</body>
</html>
\end{lstlisting}

Mit den vorangegangenen zwei Quellcode-Listings \ref{lst:controllerInExpressjs} und \ref{lst:templateInExpressjs} ist ersichtlich, dass der Applikations-Entwickler sehr grosse Kontrolle über Express.js hat.\\
Automatisch passiert in Express.js eigentlich nichts und dies ist ein grosser Vorteil im Bezug auf die Veranschaulichung der \cite{ROCA}-Prinzipien.

\subsubsection*{\gls{ORM}}
Im Gegensatz zu den anderen evaluierten Frameworks ist bei Express.js kein ORM enthalten. Deswegen musste auch bzgl. des ORMs eine kurze Evaluation gemacht werden.
\nameref{tab:bewertungskriterienORM} zeigt die Kriterien und Gewichtung für die Evaluation des ORMs.

\begin{table}[H]
\tablestyle
\tablealtcolored
\begin{tabularx}{\textwidth}{l l X c}
\tableheadcolor
	\tablehead ID &
	\tablehead Kriterium &
	\tablehead Erläuterung &
	\tablehead Gewichtung \tabularnewline
\tablebody
\textit{OK1} &
	Unterstützung DBs &
	Wieviele unterschiedliche Datenbanken unterstützt das ORM? Werden auch \gls{NoSQL}-Datenbanken unterstützt? \emph{Hohe Bewertung = Grosse Anzahl an Datenbanken}&
	\faStar \tabularnewline
\textit{OK2} &
	Relationen &
	Sind Relationen definierbar zwischen Tabellen? Verwenden diese die Datenbank-spezifischen Foreign Keys dafür (falls möglich)? \emph{Hohe Bewertung = Relationen möglich und verwendet Datenbank-spezifische Datentypen}&
	\faStar\faStar\faStar \tabularnewline
\textit{OK3} &
	Produktreife &
	Wie gut hat sich das ORM bis jetzt in der Realität beweisen können? Wie lange existiert es schon? Gibt es eine aktive Community und wird es aktiv weiterentwickelt? \emph{Hohe Bewertung = Hohe Produktreife}&
	\faStar\faStar\faStar\tabularnewline
\textit{OK4} &
	``Ease of use'' &
	Wie angenehm ist das initiale Erstellen, die Konfiguration und die Unterhaltung von Models? Führt das ORM irgendwelchen ``syntactic sugar'' \cite{syntacticsugar} ein um die Arbeit zu erleichtern? \emph{Hohe Bewertung = Hoher ``Ease of use''-Faktor} &
	\faStar \tabularnewline
\textit{OK5} &
	Testbarkeit &
	Wie gut können die mit dem Framework oder der Technologie erstellte Komponenten durch Unit Tests getestet werden? \emph{Hohe Bewertung = Hohe Testbarkeit} &
	\faStar\faStar \tabularnewline
\tableend
\end{tabularx}
\caption{Bewertungskriterien für ORM-Evaluation}
\label{tab:bewertungskriterienORM}
\end{table}

Die Tabelle \ref{tab:bewertungsmatrixORM} veranschaulicht den Grund für die Verwendung von Sequelize \cite{Sequelize}.

\begin{table}[H]
\newcolumntype{s}{>{\centering\hsize=0.15\hsize}X}
\tablestyle
\tablealtcolored
\begin{tabularx}{\textwidth}{X s s s s s s}
\tableheadcolor
	\tablehead &
	\rotatebox{90}{\bfseries\textit{OK1 Unterstützung DBs} } &
	\rotatebox{90}{\bfseries\textit{OK2 Relationen}} &
	\rotatebox{90}{\bfseries\textit{OK3 Produktreife}} &
	\rotatebox{90}{\bfseries\textit{OK4 ``Ease of use''}} &
	\rotatebox{90}{\bfseries\textit{OK5 Testbarkeit}} &
	\rotatebox{90}{\bfseries\textit{Total}}
	\tabularnewline
\tablebody
	\textit{JugglingDB} &
	\threeStars &
	\oneStar &
	\oneStar &
	\twoStars &
	\twoStars &
	\directlua{
		tex.print(math.round(
			(3 * 1 +
			1 * 3 +
			1 * 3 +
			2 * 1 +
			2 * 2) / 5
		))
	}
	\tabularnewline

	\textit{Node-ORM2} &
	\twoStars &
	\twoStars	&
	\oneStar &
	\threeStars &
	\oneStar &
	\directlua{
		tex.print(math.round(
			(2 * 1 +
			2 * 3 +
			1 * 3 +
			3 * 1 +
			1 * 2) / 5
		))
	}
	\tabularnewline

	\textit{Sequelize} &
	\oneStar &
	\twoStars &
	\twoStars &
	\twoStars &
	\oneStar &
	\directlua{
		tex.print(math.round(
			(1 * 1 +
			2 * 3 +
			2 * 3 +
			2 * 1 +
			1 * 2) / 5
		))
	}
	\tabularnewline
\tableend
\end{tabularx}
\caption{Bewertungsmatrix JavaScript ORMs}
\label{tab:bewertungsmatrixORM}
\end{table}

Das Total ist zwar jeweils ähnlich, jedoch stechen bei ``Sequelize'' die Produktreife und die Unterstützung für Relationen heraus.