\section{TP8 Automate everything or you will be hurt}
\label{sec:principle-tp8-automate-everything}

TP8 greift einen allgemein gültigen Vorsatz aus dem Software Engineering auf: Mit ``\emph{Don't repeat yourself}'' wird zum einen sich wiederholender Quellcode minimiert, als auch wiederkehrende Aufgaben automatisiert.

Entwickelt man den \emph{DRY}-Ansatz weiter, so landet man unweigerlich bei der Verwendung von automatisierten Tests und Deployments mittels Continuous Integration Systemen.

\subsection*{Geplante Umsetzung}
In der Tabelle \ref{fig:how-to-show-principles-matrix} war geplant, die Richtlinie TP8 mittels der Entwicklungsumgebungstools zu demonstrieren.

Folgende Aufgaben konnten im Zuge dieser Planung automatisiert werden:

\begin{table}[H]
\tablestyle
\tablealtcolored
\begin{tabularx}{\textwidth}{l X}
\tableheadcolor
	\tablehead ID &
	\tablehead Aufgabe
	\tabularnewline
\tablebody
	\textit{TP8.1} & Starten der Beispielapplikation\tabularnewline
	\textit{TP8.2} & Ausführung der Unit Tests\tabularnewline
	\textit{TP8.3} & Qualitative Überprüfung des Quellcodes (Code Style Guidelines)\tabularnewline
	\textit{TP8.4} & Umwandlung von SASS zu CSS Stylesheets\tabularnewline
	\textit{TP8.5} & Erstellung von Quellcode Dokumentation\tabularnewline
	\textit{TP8.6} & Veröffentlichung von Dokumentation (dieses Dokument, aber auch Quellcode Dokumentation), Testergebnisse und Test Code Coverage Berichten\tabularnewline
	\textit{TP8.7} & Umwandlung des LaTeX Quellcodes zur finalen PDF Dokumentation\tabularnewline
\tableend
\end{tabularx}
\caption{Automatisierte Aufgaben}
\end{table}


\subsection*{Konkrete Umsetzung}
Als Kernkomponente für die Automatisierung der definierten Aufgaben wird ein \emph{Makefile} (\cite{RoomiesMakefile} und \cite{ThesisMakefile}) verwendet. Dieses wird von \emph{GNU Make} \cite{make} interpretiert und ermöglicht so das Ausführen von verschiedensten Operationen.

Das Listing \ref{lst:makefileTesting} zeigt exemplarisch die Befehlsdefinition für das Ausführen der Unit sowie funktionalen Tests.

\begin{lstlisting}[language=Bash, caption=Ausschnitt Makefile: Testing \cite{RoomiesMakefile}, label=lst:makefileTesting]
REPORTER = spec
TEST_CMD = NODE_ENV=test ./node_modules/.bin/mocha --require test/runner.js --globals config

test: test-unit test-functional

test-unit:
	@echo "Running Unit Tests:"
	@$(TEST_CMD) --reporter $(REPORTER) src/lib/*/test.js

test-functional:
	@echo "Running Functional Tests:"
	@$(TEST_CMD) --reporter $(REPORTER) test/*-test.js
\end{lstlisting}

Soll nun der Task \emph{test} aufgerufen werden, geschieht dies durch den einfachen Befehl \emph{make test} auf der Kommandozeile.


\subsection*{Continuous Integration}
Die Vorgestellten Aufgaben können sowohl lokal auf dem Entwicklerrechner als auch auf dem Continuous Integration System ausgeführt werden.

Wie im Kaptiel \ref{sec:qualitymanagement} ``\nameref{sec:qualitymanagement}'' definiert, wird hierzu die Open Source Plattform Travis CI \cite{TravisCI} verwendet. Der folgende Quelltext zeigt die \emph{.travis.yml} Datei \cite{RoomiesTravisYML}. Sie steuert die Ausführung eines Builds auf Travis:

\begin{lstlisting}[language=XML, caption=.travis.yml \cite{RoomiesTravisYML}, label=lst:roomiesTravisYML]
before_install:
    - ./travis/before_install.sh

after_success:
    - ./travis/after_success.sh

language: node_js
node_js:
  - 0.8

script: "make docs test-coverage lint"

env:
    global:
        - secure: "eWkg0QTrYfWbPpitHRZWvtnTD2gTZsHEfZI1/cVA3mUs1Vycx/bNHzAt4fNm\nWR7EsTkaNLkc6p3JA797kwIyCugt+0D2tTdn7ra532Gye9u/KGjJB38HuJ0A\nQxibB+ahyAOJL+NjRNAQsME2lFmNv950/4sRbXYybijxJD6fqcw="

        - GH_USER_NAME: hsr-ba-ajw-2013
        - GH_PROJECT_NAME: BA
        - GIT_AUTHOR_NAME: "TravisCI"
        - GIT_AUTHOR_EMAIL: ""

        - CI_HOME=`pwd`/$TRAVIS_REPO_SLUG
        - RESULT_UNIT_COVERAGE_PATH=$CI_HOME/unit-coverage.html
        - RESULT_FUNCTIIONAL_COVERAGE_PATH=$CI_HOME/functional-coverage.html
        - RESULT_DOCS=$CI_HOME/docs
\end{lstlisting}

\subsection*{Diskussion}
Grundsätzlich rechtfertigt jede Aktion, welche mehr als einmal durchgeführt wird das Automatisieren dieser. Dies trifft umso mehr zu, wenn mehrere Entwickler am Entwicklungsprozess beteiligt sind.

Im Normalfall hat nicht jeder einzelne Projektmitarbeiter Zeit, sich bspw. um das Aufsetzen seiner eigenen Testumgebung zu kümmern. Dabei steigert sich die Effektivität eines Teams ungemein, wenn mittels einem einfachen Befehl die Ausführung der Unit oder gar Integration Tests angestossen werden kann.

Das Credo ``Don't repeat yourself'' ist somit ganz klar Trumpf.

Das Projektteam war zudem positiv davon überrascht, was sich mit frei zugänglichen Continuous Integration Lösungen wie Travis CI \cite{TravisCI} umsetzen lässt. Von der Generierung von Dokumentationen bis hin zur regelmässigen Ausführung von Unit Tests lassen sich ohne grossen Aufwand unbeliebte Aufgaben problemlos Automatisieren und Auslagern.