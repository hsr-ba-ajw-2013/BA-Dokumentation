\section{TP8 Automate everything or you will be hurt}
\label{sec:principle-tp8-automate-everything}

\emph{TP8} greift einen allgemein gültigen Vorsatz aus dem Software Engineering auf: Mit ``\emph{\gls{DRY}}'' wird zum einen sich wiederholender Quellcode minimiert, zum anderen werden auch wiederkehrende Routineaufgaben automatisiert.

Entwickelt man den ``\emph{\gls{DRY}}''-Ansatz weiter, so landet man unweigerlich bei der Verwendung von automatisierten Tests und Deployments mittels Continuous Integration Systemen.

\subsection*{Geplante Umsetzung}
Gemäss Tabelle \ref{fig:how-to-show-principles-matrix} ``\nameref{fig:how-to-show-principles-matrix}'' im Kapitel ``\nameref{sec:analyse-der-aufgabenstellung}'' soll \emph{TP8} durch die Verwendung verschiedener unterstützender Tools umgesetzt und demonstriert werden.

Tabelle \ref{fig:automated-tasks-planned} zeigt eine Auflistung der zu automatisierenden Aufgaben:

\begin{table}[H]
\tablestyle
\tablealtcolored
\begin{tabularx}{\textwidth}{l X}
\tableheadcolor
	\tablehead ID &
	\tablehead Aufgabe
	\tabularnewline
\tablebody
	\textit{TP8.1} & Starten der Beispielapplikation\tabularnewline
	\textit{TP8.2} & Ausführung der Unit Tests\tabularnewline
	\textit{TP8.3} & Formale Überprüfung des Quellcodes\tabularnewline
	\textit{TP8.4} & Umwandlung von SASS zu CSS Stylesheets\tabularnewline
	\textit{TP8.5} & Erstellung von Quellcode Dokumentation\tabularnewline
\tableend
\end{tabularx}
\caption{Automatisierte Aufgaben (geplant)}
\label{fig:automated-tasks-planned}
\end{table}


\subsection*{Konkrete Umsetzung}
Neben den oben erwähnten Aufgaben wurden in der konkreten Implementation des Projektes weitere Tasks erfolgreich automatisiert:

\begin{table}[H]
\tablestyle
\tablealtcolored
\begin{tabularx}{\textwidth}{l X l}
\tableheadcolor
	\tablehead ID &
	\tablehead Aufgabe &
	\tablehead \gls{CLI} Befehl
	\tabularnewline
\tablebody
	\textit{TP8.1} & Starten der Beispielapplikation & \emph{npm start}\tabularnewline
	\textit{TP8.2} & Ausführung der Unit Tests & \emph{make test}\tabularnewline
	\textit{TP8.3} & Qualitative Überprüfung des Quellcodes (Code Style Guidelines) & \emph{make lint}\tabularnewline
	\textit{TP8.4} & Umwandlung von SASS zu CSS Stylesheets & \emph{make precompile-sass}\tabularnewline
	\textit{TP8.5} & Erstellung von Quellcode Dokumentation & \emph{make docs}\tabularnewline
	\textit{TP8.6} & Vorbereitung von View Templates & \emph{make precompile-templates}\tabularnewline
	\textit{TP8.7} & Veröffentlichung von Dokumentation (dieses Dokument, aber auch Quellcode Dokumentation), Testergebnissen und Test Code Coverage Reports & Travis CI\tabularnewline
	\textit{TP8.8} & Umwandlung des LaTeX Quellcodes zur finalen PDF Dokumentation (Dokumentations Repository) & \emph{make}\tabularnewline
	\textit{TP8.9} & Installation der Beispielapplikation & \emph{./install.sh}\tabularnewline
\tableend
\end{tabularx}
\caption{Automatisierte Aufgaben (umgesetzt)}
\label{fig:automated-tasks-concrete}
\end{table}

Als Kernkomponente für die Automatisierung der Aufgaben in Tabelle \ref{fig:automated-tasks-concrete} wird ein \emph{Makefile} (\cite{RoomiesMakefile} und \cite{ThesisMakefile}) verwendet. Dieses wird von \emph{GNU Make} \cite{make} interpretiert und ermöglicht so das Ausführen verschiedenster Operationen.

Der Quelltext \ref{lst:makefileDocs} zeigt exemplarisch die Befehlsdefinition für die Erstellung der Quellcode Dokumentation mittels \emph{Natural Docs} \cite{NaturalDocs}.

\begin{lstlisting}[language=Bash, firstnumber=99, caption={Ausschnitt Makefile: Code Dokumentation erstellen} \cite{RoomiesMakefile}, label=lst:makefileDocs]
docs:
	-mkdir ./docs
	@NaturalDocs -i ./src -o HTML ./docs -p ./.naturaldocs -xi ./src/server/public -s Default style
\end{lstlisting}

Durch den einfache Kommandozeilenbefehl \emph{make docs} wird der in \ref{lst:makefileDocs} definierte Befehl ausgeführt.


\subsection*{Continuous Integration}

Die in Tabelle \ref{fig:automated-tasks-concrete} vorgestellten Aufgaben \emph{TP8.2} bis \emph{8.8} werden sowohl lokal auf dem Entwicklerrechner als auch auf dem Continuous Integration System ausgeführt.

Entsprechend der Definition in Kapitel \ref{sec:qualitymanagement} ``\nameref{sec:qualitymanagement}'' wird hierzu die Open Source Plattform Travis CI \cite{TravisCI} verwendet. Der Quelltext \ref{lst:roomiesTravisYML} zeigt einen Ausschnitt der Datei \emph{.travis.yml} \cite{RoomiesTravisYML}. Diese steuert den Build auf Travis CI.

\begin{lstlisting}[language=XML, firstnumber=5, caption={Ausschnit aus .travis.yml \cite{RoomiesTravisYML}}, label=lst:roomiesTravisYML]
before_install:
    - ./travis/before_install.sh

after_success:
    - ./travis/after_success.sh

language: node_js
node_js:
  - 0.8

script: "make docs test-coveralls lint"
\end{lstlisting}

\subsection*{Diskussion}
Muss eine Aufgabe zweimal ausgeführt werden, sollte dies als Rechtfertigung zur Automatisierung einer solchen Arbeit bereits genügen. Dies trifft umso mehr zu, wenn mehrere Entwickler am Entwicklungsprozess beteiligt sind.

Die Praxis zeigt, dass Zeitersparnis und Effektivitätssteigerung die positiven Folgen der Automatisierung von Routineaufgaben sind. Wird die Ausführung der Unit Tests erleichtert, lässt sich zudem die Akzeptanz eines Test Driven Development Prozesses erhöhen.

Das Projektteam war positiv davon überrascht, was sich mit frei zugänglichen Continuous Integration Lösungen wie Travis CI \cite{TravisCI} umsetzen lässt. Von der Generierung von Dokumentationen bis hin zur regelmässigen Ausführung von Unit Tests lassen sich ohne grossen Aufwand unbeliebte Aufgaben problemlos automatisieren und auslagern.

Das Credo ``Don't repeat yourself'' ist somit ganz klar Trumpf und so wird auch die Richtlinie \emph{TP8 Automate everything or you will be hurt} vom Projektteam unterstützt.