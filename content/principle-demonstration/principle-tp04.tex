\section{TP4 Separate user identity and sign-up (...)}
\label{sec:principle-tp4-seperate-user-identity}

Ein Grossteil der Dienste im Internet verlangen heute nach der Erstellung eines Benutzerkontos. Ein Forum lässt beispielsweise das verfassen von Beiträgen erst zu, nachdem sich der potentielle Autor mit seiner E-Mail-Adresse, einem Benutzernamen und einem Passwort erfolgreich registriert hat.

Eine solche Registrierung hat für den Besucher als auch für einen Dienstbetreiber auf den ersten Blick nur Vorteile:

\begin{itemize}
	\item Die User Experience vom Benutzer angepasst/optimiert werden (Bspw. das Speichern der eigenen Zeitzone für korrekte Datums- und Zeitangaben, Personalisierung der Frontseite usw.)
	\item Sicherstellung der Identität: Jeder Benutzername wird immer von derselben Person verwendet
	\item Der Betreiber kann seine Besucher über Neuerungen Informieren
	\item Durch gezielte Auswertung und Statistiken kann der Betreiber sein Angebot optimieren
\end{itemize}

Bei genauerer Analyse ergeben sich aber auch nicht zu vernachlässigende negative Faktoren:

\begin{itemize}
	\item Der Benutzer muss bei jedem neuen Dienst erneut ein Konto erstellen und möglicherweise wiederholt die gleichen Angaben machen
	\item Der Betreiber muss sich um die Speicherung und Sicherheit der Benutzerinformationen kümmern
	\item Nach einer gewissen Zeit hat ein Benutzer tendenziell keine Kontrolle mehr darüber, wo er sich überall einmal angemeldet hat und seine Informationen hinterlegt hat
\end{itemize}

Beide Auflistungen von Vor- und Nachteilen sind exemplarisch und nicht abschliessend. Trotzdem ergeben sie jedoch einen guten Überblick über die Thematik.

Tilkovs TP4 schlägt die Separierung von Benutzerinformationen und Registrierung vor. Der Vorreiter OpenID \cite{OpenID}, jedoch insbesondere die immer beliebter werdenden sozialen Netzwerke ermöglichen eben diese Auftrennung.

Hierbei erstellt der Benutzer bspw. bei Facebook einmalig ein Konto.


\subsection*{Geplante Umsetzung}


\subsection*{Konkrete Umsetzung}


\subsection*{Diskussion}
