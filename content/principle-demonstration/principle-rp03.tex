\section{RP3 HTTP}
\label{sec:principle-rp3-http}

Das HTTP Prinzip von ROCA geht in die gleiche Richtung wie Kapitel \ref{sec:principle-rp1-rest} ``\nameref{sec:principle-rp1-rest}''.
Der Client soll laut diesem Prinzip mit dem Server mittels REST kommunizieren.

\subsection*{Geplante Umsetzung}
Wie bereits in Kapitel \ref{sec:principle-rp1-rest} erwähnt, soll der Webclient mit dem Server über REST kommunizieren.
Dies geschieht über Backbone Models welche eine eindeutige \gls{URL} zugewiesen bekommen und dann mittels den beiden Methoden ``save()'' und ``sync()'' bzw. ``fetch()'' (siehe \cite{BackboneSync}) mit dem Server kommunizieren.


\subsection*{Konkrete Umsetzung}
Die geplante Umsetzung ist so implementiert worden. Als Beispiel zeigt Quelltext \ref{lst:roomiesCommunityModel} ein \emph{Backbone.Model}, welches eine URL definiert hat und damit synchronisiert werden kann.

\begin{lstlisting}[language=JavaScript, caption=Community Model \cite{roomiesCommunityModel}, label=lst:roomiesCommunityModel]
/** Class: Models.Community
 * Community model as a subclass of <Barefoot.Model at
 * http://swissmanu.github.io/barefoot/docs/files/lib/model-js.html>
 */
var Barefoot = require('node-barefoot')()
	, Model = Barefoot.Model
	, CommunityModel = Model.extend({
		urlRoot: '/api/community'
		, idAttribute: 'id'
		, toString: function toString() {
			return 'CommunityModel';
		}
	});

module.exports = CommunityModel;
\end{lstlisting}

\subsection*{Diskussion}
Dieses ROCA-Prinzip überschneidet sich wie erwähnt sehr stark mit \nameref{sec:principle-rp1-rest}.

Aus diesem Grund wird hier auf die Diskussion im Kapitel \ref{sec:principle-rp1-rest} verwiesen. Wenn der Server eine REST-API zur Verfügung stellt, sollte diese vom Client auch konsumiert werden.