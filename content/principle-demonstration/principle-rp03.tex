\section{RP3 HTTP}
\label{sec:principle-rp3-http}

Das HTTP Prinzip von ROCA geht in die gleiche Richtung wie Kapitel \ref{sec:principle-rp1-rest} ``\nameref{sec:principle-rp1-rest}''.
Eine Applikation mit einer \gls{REST}-API ist die Grundvoraussetzung dafür, dass die Clientseite der Webapplikation mit dem Server \gls{RESTful} kommunizieren kann.

\subsection*{Geplante Umsetzung}
Die Kommunikation des Clients mit dem Server über \gls{REST} geschieht unter anderem über Backbone Models welche mittels den beiden Methoden ``save()'' und ``sync()'' bzw. ``fetch()'' (siehe \cite{BackboneSync}) mit dem Server kommunizieren.

Um auch normale HTML-Formulare \gls{RESTful} zu machen, soll ein Weg gefunden werden, mit dem nebst ``POST'' auch ``PUT'' und ``DELETE'' Abfragen ermöglicht werden.

\subsection*{Konkrete Umsetzung}
Die geplante Umsetzung ist so implementiert worden. Als Beispiel zeigt Quelltext \ref{lst:roomiesCommunityModel} ein \emph{Backbone.Model}, welches eine URL definiert hat und damit synchronisiert werden kann.

\begin{lstlisting}[language=JavaScript, caption=Community Model \cite{roomiesCommunityModel}, label=lst:roomiesCommunityModel]
/** Class: Models.Community
 * Community model as a subclass of <Barefoot.Model at
 * http://swissmanu.github.io/barefoot/docs/files/lib/model-js.html>
 */
var Barefoot = require('node-barefoot')()
	, Model = Barefoot.Model
	, CommunityModel = Model.extend({
		urlRoot: '/api/community'
		, idAttribute: 'id'
		, toString: function toString() {
			return 'CommunityModel';
		}
	});

module.exports = CommunityModel;
\end{lstlisting}

Um dieses Model mit Daten abzufüllen, kann wie in Quelltext \ref{lst:roomiesCommunitySync} gezeigt, eine Instanz vom ``DataStore'' geholt werden (Zeile \ref{lst:roomiesCommunitySync_dataStore}) und die ``fetch()''-Methode aufgerufen werden (Zeile \ref{lst:roomiesCommunitySync_fetch}).

\begin{lstlisting}[language=JavaScript, caption=Community Model Synchronisation \cite{roomiesCommunityJoinView}, label=lst:roomiesCommunitySync, firstnumber=10, escapeinside={@}{@}]
/** Function: initialize
 * Initializes the view
 */
, initialize: function initialize() {
	@\label{lst:roomiesCommunitySync_dataStore}@var community = this.options.dataStore.get('community');
	this.community = community;
}

/** Function: beforeRender
 * Before rendering it will fetch the community if not done yet.
 *
 * Parameters:
 *   (Promise.resolve) resolve - After successfully doing work, resolve
 *                               the promise.
 */
, beforeRender: function beforeRender(resolve) {
	/* jshint camelcase:false */
	var _super = this.constructor.__super__.beforeRender.bind(this)
		, resolver = function resolver() {
			_super(resolve);
		};

	if(!this.community.has('name')) {
		@\label{lst:roomiesCommunitySync_fetch}@this.community.fetch({
			success: resolver
			, error: resolver
		});
	} else {
		resolver();
	}
}
\end{lstlisting}

\subsubsection*{\gls{RESTful} Forms}
HTML-Formulare unterstützen nur zwei Arten von HTTP-Methoden, ``POST'' und ``GET'' \cite{FormMethodMDN}. Um eine Applikation \gls{RESTful} zu machen, sollten aber zumindest zumindest zusätzlich ``PUT'' und ``DELETE'' unterstützt werden.

Die Unterstützung dieser zusätzlichen Methoden erfordert einen Hilfskonstruktion:
\begin{itemize}
	\item Jedes Formular das nicht ``POST'' oder ``GET'' verwendet, erhält ein zusätzliches verstecktes Feld namens \emph{\_method} und dem Wert der gewünschten Methode
	\item Sobald der Benutzer das Formular abschickt, wird  vom Server dieses Feld gelesen
	\item Der Server ruft danach den entsprechenden Controller mit der entsprechenden Methode auf.
\end{itemize}

Quelltext \ref{lst:expressMethodOverride} zeigt die Anbindung der entsprechenden ``MethodOverride'' Middleware \cite{methodOverrideMiddleware} an eine Express-Applikation auf Zeile \ref{lst:expressMethodOverride_Override}.

\begin{lstlisting}[language=JavaScript, caption=HTTP Middleware \cite{roomiesHTTPMiddleware}, label=lst:expressMethodOverride, firstnumber=18, escapeinside={@}{@}]
/** Function: setupHttp
 * Adds described middlewares to the passed Express.JS application
 *
 * Parameters:
 *   (Object) app - Express.JS application
 *   (Object) config - Configuration
 */
function setupHttp(app, config) {
	var db = app.get('db');

	app.use(express.bodyParser());
	app.use(express.cookieParser());

	app.use(express.session({
		store: new SequelizeStore({
			db: db
		})
		, secret: config.sessionSecret
	}));

	@\label{lst:expressMethodOverride_Override}@app.use(express.methodOverride());
}

module.exports = setupHttp;
\end{lstlisting}

Als Beispiel zeigt Quelltext \ref{lst:taskCheckForm} das Formular für das markieren einer Aufgabe als erledigt. Auf Zeile \ref{lst:taskCheckForm_MethodField} wird das entsprechende Feld definiert.

\begin{lstlisting}[language=JavaScript, caption=Formular mit verstecktem \emph{\_method} Feld \cite{taskCheckForm}, label=lst:taskCheckForm, firstnumber=6, escapeinside={@}{@}]
<form class="reset-style" action="/community/{{community.slug}}/tasks/{{id}}" data-task-id="{{id}}" method="post">
	@\label{lst:taskCheckForm_MethodField}@<input type="hidden" name="_method" value="put"/>

	<input type="hidden" name="name" value="{{name}}"/>
	<input type="hidden" name="reward" value="{{reward}}"/>
	<input type="hidden" name="dueDate" value="{{formatDate dueDate}}"/>
	<input type="hidden" name="fulfillorId" value="{{resident.id}}"/>
	<input type="hidden" name="fulfilledAt" value="{{formatDate now}}"/>

	<button type="submit" class="reset-style">
		<i class="icon-check-empty"></i>
	</button>
</form>
\end{lstlisting}

Dieser Workaround kann nicht als wirklich schön bezeichnet werden, jedoch funktioniert er ohne Probleme und in allen Browsern.

\subsection*{Diskussion}

``RP3 - HTTP'' ist eine Richtlinie, welche nach Auffassung des Projektteams nur hinsichtlich der Formulare bereichernd für den Konzeptkatalog ist. Die restliche \gls{REST} Thematik ist schon mit \nameref{sec:principle-rp1-rest} abgedeckt und sollte somit hinlänglich Thematisiert worden sein.

Die Verwendung von ``PUT'', ``DELETE'' etc. für Formulare hat sowohl gute wie auch schlechte Seiten: Einerseits können die gleichen API-Routen (und somit der gleiche Code) wie für normale API-Aufrufe verwendet werden. Dafür ist allerdings ein relativ unschöner (aber doch relativ eleganter) Workaround zu verwenden.

Falls die eigene Applikation komplett mittels einer \gls{REST}-API aufgebaut wurde ist diese Hilfskonstruktion sicher ein gehbarer Weg, ohne zu viel Aufwand zu bringen.