\section{RP2 Application Logic}
\label{sec:principle-rp2-application-logic}

\subsection*{Geplante Umsetzung}
Die ``Application Logic'' ROCA-Richtlinie legt fest, dass jegliche Applikationslogik
auf dem Server passiert und geschrieben ist.
Entsprechend wird die Beispielapplikation bei kritischen Logiken wie Authentifizierung
etc. die Ausführung auf den Server beschränken.

\subsection*{Konkrete Umsetzung}
Geschäftskritische Applikationslogik wie Validierung und Bereinigung von Daten, sowie
Authentisierung und Authorisation von Usern muss immer auf dem Server passieren.

Um allerdings möglichst wenig Code duplizieren zu müssen, verwendet ``Roomies''
nahezu den gleichen Code auf Server-- wie auf Clientseite. Aus diesem Grund werden
Teile der Applikationslogik sowohl auf dem Client, wie auch auf dem Server sein.

\subsection*{Diskussion}
Die wichtige Applikationslogik muss, wie bereits erwähnt, sicher auf dem Server liegen.
Zudem müssen alle Daten validiert werden.

Heutige Websiten und insbesondere Web-Applikationen brechen dieses Paradigma aber
ganz bewusst, damit die Applikation insgesamt schneller ist. Wenn immer zuerst ein
API-call zum Server gemacht werden muss um zu überprüfen dass alles stimmt, sinkt
die Performance und der User wird schnell gelangweilt.

Für Entwickler ist es daher essenziell zu entscheiden, was unbedingt auf dem Server
validiert werden muss und was auch auf dem Client sein kann.