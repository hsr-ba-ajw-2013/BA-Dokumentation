\section{RP18 History API}
\label{sec:principle-rp18-history-api}

Mit dem Aufkommen der JavaScript History API \cite{HistoryAPI} ist es ohne Workarounds möglich, dynamische Seitenwechsel im Browser von JavaScript aus veranzulassen.

\subsection*{Geplante Umsetzung}
Mithilfe von ``Backbone.History'' \cite{BackbonejsHistory} soll jegliche URL-Änderung in modernen Browsern mit JavaScript gemacht werden.

Falls JavaScript oder die History API nicht unterstützt wird, soll es trotzdem möglich sein, die Webseite zu benutzen. Die User Experience soll dadurch nicht eingeschränkt werden. Eventuell soll dabei der \gls{Hashbang} als Fallback verwendet werden.

\subsection*{Konkrete Umsetzung}
Das Geplante konnte wie gewünscht umgesetzt werden. Wie im Abschnitt \ref{sec:principle-rp10-browser-controls} ``\nameref{sec:principle-rp10-browser-controls}'' beschrieben, wurde dafür ``Backbone.History'' verwendet.

In ``Backbone.History'' ist es möglich den Hashbang als Fallback zu verwenden. Da es aber im Internet Explorer mehr Probleme verursacht als löst, wurde für ``barefoot'' entschieden, diese Funktion zu deaktivieren. Dies sieht man im Quelltext \ref{lst:rp18barefootStartClientHistory} auf den Zeilen \ref{lst:rp18barefootStartClientHistory_useHistory} und \ref{lst:rp18barefootStartClientHistory_hashChange}.

\begin{lstlisting}[language=JavaScript, caption=Aktivierung des \emph{History}-API's in barefoot \cite{BarefootStartClient}, label=lst:rp18barefootStartClientHistory, firstnumber=29, escapeinside={@}{@}]
// from modernizr, MIT | BSD
// http://modernizr.com/
@\label{lst:rp18barefootStartClientHistory_useHistory}@var useHistory = !!(window.history && history.pushState);

Backbone.history.start({
	pushState: useHistory
	@\label{lst:rp18barefootStartClientHistory_hashChange}@, hashChange: useHistory
	, silent: true
});
\end{lstlisting}

\subsection*{Diskussion}
Vor einigen Jahren war die History API noch nicht verfügbar und mehrere Unternehmen verwendeten als Workaround das sogenannte \gls{Hashbang}. Diese Technik verwendet das setzen von sogenannten ``Anchors'' in der URL um von JavaScript aus die \gls{URL} gemäss aktueller Seite anzupassen. Das grosse Problem dieser Technik ist jedoch, dass der Browser dieses \gls{URL}-Fragment nicht dem Server mitschickt.
Infolgedessen mussten die Webapplikationen folgenden Workaround verwenden, wenn jemand eine URL mit dem Hashbang eintippte:

\begin{enumerate}
	\item Startseite ausliefern
	\item Mittels JavaScript auslesen, welche Seite eigentlich gefragt war
	\item Die eigentlich gewünschte Seite ausliefern
\end{enumerate}

Da es nur noch wenige Browser gibt, welche die History API nicht unterstützen (z.B. Internet Explorer 8), gibt es immer noch einige Seiten welche dies als Fallback einsetzen. Da dieser Fallback jedoch umständlich und nicht förderlich für die User Experience ist, wird immer wie mehr darauf verzichtet (siehe z.B. den Blogpost von Twitter über die Abschaffung des Hashbangs \cite{twitterAbandonsHashbangs}).

Die History API wird heutzutage von allen modernen Browsern unterstützt. Dementsprechend gibt es kein Herumkommen um diese API. Das Projektteam
empfiehlt zudem, Hashbangs ohne zwingenden Grund nicht zu verwenden.