\section{RP16 Know Structure}
\label{sec:principle-rp16-know-structure}

Gehen wir exemplarisch von einer modernen, entkoppelten Applikationsarchitektur aus, welche eine klare Trennung zwischen Front- und Backend vorsieht, so übernimmt der Frontendteil die Erzeugung des User Interfaces auf dem Clientrechner (Beispiele u.A. bei \emph{TodoMVC} \cite{TodoMVC}).

Das Backend liefert beim initialen Request ein HTML Grundgerüst, auf welchem die JavaScript Logik des Frontends das finale UI aufbaut.

\begin{lstlisting}[language=HTML, caption={Beispiel eines HTML Gerüsts zum Rendering eines User Interfaces}, label=lst:htmlSkeleton, escapeinside={@}{@}]
<html>
<head>
	<meta charset="utf-8">
	<title>HTML 5 Example App - Client Side UI Logic</title>
	<link href="/stylesheets/app.css" rel="stylesheet">
</head>
<body>
	@\label{lst:htmlSkeleton_maindiv}@<div id="main"></div>
	@\label{lst:htmlSkeleton_appjs}@<script src="/javascripts/app.js"></script>
</body>
</html>
\end{lstlisting}

Zeile \autoref{lst:htmlSkeleton_appjs} im Quelltext \ref{lst:htmlSkeleton} zeigt beispielhaft die Einbindung der JavaScript-Datei aus Quelltext \ref{lst:htmlSkeletonJavascriptFile}. Nach Beendigung des Ladevorgangs wird unter Verwendung des \gls{DOM}-Manipulators \emph{jQuery} \cite{jQuery} dynamisch ein Titel-Element in das \emph{<div>}-Element mit der ID \emph{main} eingefügt.

\begin{lstlisting}[language=JavaScript, caption={JavaScript-Datei \emph{app.js} zu Quelltext \ref{lst:htmlSkeleton}}, label=lst:htmlSkeletonJavascriptFile]
$(function() {
	$('div#main').html('<h1>Hello World</h1>');
});
\end{lstlisting}

Das ROCA Prinzip 17 \emph{Know Structure} beschreibt den oben aufgezeigten Aufbau und erachtet es als wichtig, dass die Backendkomponente keine Kenntnis über das vom Frontendteil gerenderten User Interface hat. Das Backend soll lediglich die initiale Struktur des Grundgerüsts aus Quelltext \ref{lst:htmlSkeleton} kennen und später nur noch als Datenlieferant via einer API dienen.


\subsection*{Geplante Umsetzung}

Unter Berücksichtigung des Prinzips \emph{\nameref{sec:principle-rp14-unobtrusive-javascript}}, näher beschrieben im Abschnitt \ref{sec:principle-rp14-unobtrusive-javascript}, wird nicht geplant, \emph{RP16 Know Structure} in seiner essentiellen Form innerhalb der Beispielapplikation \emph{Roomies} zur Anwendung zu bringen.


\subsection*{Konkrete Umsetzung}

Wie in Abschnitt \ref{sec:principle-rp14-unobtrusive-javascript} dokumentiert, wurde dem ROCA Prinzip \emph{\nameref{sec:principle-rp14-unobtrusive-javascript}} eine grössere Gewichtung zugestanden als \emph{Know Structure}. Aus diesem Grund wurde wie geplant darauf verzichtet User Interface Logik nur in der Frontendkomponente zu verwenden.

Die aus den Abschnitten \ref{sec:principle-rp15-no-duplication} und \ref{sec:principle-rp14-unobtrusive-javascript} bekannte geteilte Codebasis zwischen Front- und Backendkomponente zielt sogar absichtlich darauf ab, dass auch im Backend die komplette Struktur des User Interfaces bekannt ist.

Ein grundlegender Punkt wurde jedoch aus \emph{RP16 Know Structure} adaptiert: \emph{Roomies} verwendet wie vorgeschlagen ein HTML Grundgerüst \cite{roomiesHtmlSkeleton}, in welches sowohl im Front- als auch Backend die UI Elemente gerendert werden.


\subsection*{Diskussion}

\emph{Separation of concerns} \cite{SeparationOfConcerns} gehört nicht umsonst zu einem der grundlegendsten Prinzipien im Software Engineering. Darauf bezogen hat die eigentliche Intension von \emph{RP16 Know Structure} durchaus seine Daseinsberechtigung: Eine klare Auftrennung von UI Rendering und eigentlicher Geschäftslogik ist erstrebenswert.

Möchte man die Architekturrichtlinie \emph{\nameref{sec:principle-rp14-unobtrusive-javascript}} zwecks bestmöglicher Kompatibilität in die Entwicklung einer Applikation mit einfliessen lassen, kommt es unweigerlich zum Konflikt mit \emph{RP16}. Damit die Applikation auch ohne das User Interface Rendering direkt auf dem Client funktionieren kann, muss die Backendkomponente zwingend über Renderingfunktionalität und damit Wissen über die Struktur des resultierenden HTML Markups verfügen. Damit bricht dieses Vorgehen klar mit den Anforderungen von \emph{Know Structure}.

Kann auf \emph{\nameref{sec:principle-rp14-unobtrusive-javascript}} verzichtet werden, mag \emph{RP16 Know Structure} seine Stärken ausspielen können. Ist jedoch das clientunabhängige Rendering des User Interfaces eine Anforderung an die zu erstellende Lösung, empfiehlt das Projektteam von \emph{RP16} abzusehen.
