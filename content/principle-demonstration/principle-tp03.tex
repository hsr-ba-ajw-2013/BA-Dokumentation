\section{TP3 Eat your own API dog food}
\label{sec:principle-tp3-eat-your-own-api}

Eine Applikation mit einer verteilten, entkoppelten Architektur kommt unweigerlich zu einem Punkt, an welchem die einzelnen Komponenten Schnittstellen definieren müssen.

Die durch diesen Prozess entstehenden API's sind klassischerweise auf spezifische Anwendungsfälle zugeschnitten da diese schnellstmöglich die applikationseigenen Anforderungen umsetzen sollen.

Als weitere Konsequenz werden ``unschöne'' Interfacemethoden gerne auch gar nicht erst an externe Anwendungen resp. Konsumenten veröffentlicht.

Auf lange Dauer gesehen entsteht so ein Flickwerk mit API Methoden für jeden einzelnen Anwendungsfall.

Mit \emph{Eat your own API dog food} forciert Tilkov gezielt die Konzipierung und Umsetzung guter und generischer Schnittstellen für Applikationskomponenten. Dazu gehört, dass keine privaten Methoden existieren sollen, frei nach dem Credo ``\emph{Wir haben nichts zu verstecken}''.


\subsection*{Geplante Umsetzung}

Für die Beispielapplikation \emph{Roomies} soll ein Servicelayer auf Basis einer HTTP \gls{REST} Architektur entwickelt werden. Als Datenformat soll \gls{JSON} verwendet werden.

Entsprechend der \gls{REST} Richtlinien (siehe \ref{sec:principle-rp1-rest} ``\nameref{sec:principle-rp1-rest}'') soll jedes Objekt aus der Problemdomäne gezielt abgefragt und manipuliert werden können.

Es sind keine privaten Methoden geplant. Soll ein Objekt vor Zugriffen unbefugter Konsumenten geschützt werden, sind entsprechende Sicherheitsmechanismen umzusetzen.


\subsection*{Konkrete Umsetzung}

Wie bereits im Abschnitt \ref{sec:principle-rp1-rest-concrete-solution} des Kapitels ``\nameref{sec:principle-demonstration}'' erläutert, konnte die generische Serviceschnittstelle für alle Objekte aus der \emph{Roomies} Problemdomäne (siehe \ref{sec:sad-domain-model} ``\nameref{sec:sad-domain-model}'') umgesetzt werden.

Es wurde komplett auf private Methoden verzichtet. Zum Schutz sensibler Daten wurde wie in den Abschnitten \ref{sec:principle-rp7-auth}, \ref{sec:principle-rp8-cookies} sowie \ref{sec:principle-rp9-session} beschrieben ein Session-basierter Authentifizierungsmechanismus via Facebook (siehe \ref{sec:principle-tp4-seperate-user-identity} ``\nameref{sec:principle-tp4-seperate-user-identity}'') implementiert.


\subsection*{Diskussion}
