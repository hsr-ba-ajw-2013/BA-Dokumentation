\section{RP10 Browser-Controls}
\label{sec:principle-rp10-browser-controls}

\begin{quotation}
Man befindet sich auf einem Online-Shop und sucht mit Hilfe der eingebauten Suche Zubehör für seine neu ergatterte Fotokamera. Das Suchresultat ist aber noch zu grob. Deshalb klickt man auf den Zurück-Knopf des Browsers um auf die Suche zurück zu gelangen. Aber was uns hier erwartet, ist nicht etwa das Suchformular, welches man erwartet hätte, sondern die Startseite.
\end{quotation}
Dies ist kein fiktives Beispiel, sondern ein reelles Szenario aus dem Online-Shop des Elektronikgrosshändlers Digitec \cite{Digitec}.

Das ROCA-Prinzip ``Browser-Controls'' will genau dieses unangenehme Erleben verhindern. Dabei ist es eng mit dem Prinzip ``\nameref{sec:principle-rp4-link}'' verbunden. Indem jede Seite ihre eigene \gls{URI} hat und diese beim Navigieren in die ``History'' des Browsers hinzugefügt wird, funktionieren die Standardbedienelemente (Zurück, Vorwärts und Aktualisieren) eines Browsers weiterhin erwartungsgemäss.


\subsection*{Geplante Umsetzung}

Jeder Seitenwechsel innerhalb der Beispielapplikation \emph{Roomies} soll über den Verlauf im Browser nachvollziehbar sein und so dem Benutzer erlauben, die im Browser eingebauten Funktionen zur Navigation zu benutzen.

\subsection*{Konkrete Umsetzung}

Mit der Funktion ``Backbone.history.start()'' \cite{BackbonejsHistory} bietet \emph{Backbone.js} resp. \emph{barefoot} eine Anbindung an die History API \cite{HistoryAPI} und ermöglicht damit JavaScript Applikationen das Navigieren inkl. einer Aktualisierung der URL bei Seitenwechseln.

\begin{lstlisting}[language=JavaScript, caption={Aktivierung der History in Barefoot \cite{BarefootStartClient}}, label=lst:barefootStartClientHistory, firstnumber=29]
// from modernizr, MIT | BSD
// http://modernizr.com/
var useHistory = !!(window.history && history.pushState);

Backbone.history.start({
	pushState: useHistory
	, hashChange: useHistory
	, silent: true
});
\end{lstlisting}

\subsection*{Diskussion}

Das einleitende Beispiel \emph{Digitec} ist keine Seltenheit: Oft leidet die User Experience auf entsprechenden Angeboten extremst. Diese Internetseiten verletzen zudem meist absichtlich \nameref{sec:principle-rp4-link} um Funktionalitäten wie \emph{\gls{Frames}} einbinden zu können.

In der Zeit der Social Networks gehört es zum Alltag, interessante Links mit Freunden zu teilen. Gerade hier wird dem Endbenutzer oft bewusst, dass das geteilte Angebot hintergründig nicht \emph{RP4} und somit \emph{RP10} befolgt. Dem Projektteam ist es unverständlich, warum nicht mehr Gebrauch von neuen Features wie dem \emph{History API} gemacht wird. Der Mehrwert für den Benutzer bzw. die Verbesserung der User Experience ist oft immens.

Abschliessend gehört für das Projektteam die Umsetzung von \emph{RP10 Browser-Controls} zu einem Muss.