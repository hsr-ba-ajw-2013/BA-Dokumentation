\section{RP10 Browser-Controls}
\label{sec:principle-rp10-browser-controls}
Man befindet sich auf einem Online-Shop und sucht mit Hilfe der eingebauten Suche, Zubehör für seine neu ergatterte Fotokamera. Das Suchresultat ist aber noch zu grob. Deshalb klick man auf dem Zurückknopf des Browsers um auf die Suche zurück zu gelangen. Aber was uns hier erwartet, ist nicht etwa das Suchformular, welches man erwartet hätte, sondern die Startseite.
Dies ist nicht etwa ein fiktives Beispiel, sondern ein reeller Szenario aus dem Online-Shop des Elektronikgrosshändlers Digitec \cite{Digitec}.

Das ROCA-Prinzip ``Browser-Controls'' will genau dieses unangenehme Erleben verhindern.

\subsection*{Geplante Umsetzung}

``Browser-Controls'' ist sehr eng mit dem Prinzip ``\nameref{sec:principle-rp4-link}'' verbunden. Indem jede Seite sein eigenes Link hat und dieser in die ``History'' des Browsers hinzugefügt wird, können die Standardbedienelementen (Zurück, Vorwärts und Aktualisieren) eines Browsers erwartungsgemäss funktionieren.

Dies soll auch in der Beispielanwendung ``Roomies'' der Fall sein. Jedes Seitenwechsel soll über die Browserhistory nachvollziehbar sein und somit die Browserbedienelementen unterstützen.

\subsection*{Konkrete Umsetzung}

Backbone bietet mit der Function ``Backbone.history.start()'' \cite{BackbonejsHistory} genau diese erwünschte Funktionalität und wurde in Barefoot direkt umgesetzt.

\begin{lstlisting}[language=JavaScript, caption=Activierung der History in Barefoot \cite{BarefootStartClient}, label=lst:barefootStartClientHistory, firstnumber=29]
// from modernizr, MIT | BSD
// http://modernizr.com/
var useHistory = !!(window.history && history.pushState);

Backbone.history.start({
	pushState: useHistory
	, hashChange: useHistory
	, silent: true
});
\end{lstlisting}

\subsection*{Diskussion}
