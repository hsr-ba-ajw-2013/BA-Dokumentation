\section{TP7 Apply the Web instead of working around}
\label{sec:principle-tp7-apply-the-web}

Der moderne Internetbrowser kapselt eine Vielzahl von leistungsfähigen Funktionen des jeweiligen Hostrechner in ein für den Software Entwickler leicht zu verwendendes Interface. Dazu gehört die Integration von systemnahen Komponenten wie \gls{GPU}'s \cite{webgl} genauso wie die Möglichkeit, gleichzeitig mehrere Fenster oder Tabs für die selbe oder verschiedenen Applikationen resp. Internetseite offen zu halten.

Das Hin- und Herspringen zwischen besuchten Seiten mittels Vorwärts- und Zurück-Schaltflächen gehört seit Beginn der Webära zum festen Bestandteil der User Experience im Internet.

In der Vergangenheit gehörten wiederkehrende Umsetzungen von Funktionen wie der Validierung von Formularinhalten zu lästigen, aber nötigen Ärgernissen. Mit der Einführung der neusten Revision 5 des HTML Standards können gerade solche Aufgaben bequem dem Browser \cite{HTML5Forms} überlassen werden

Mit der immer mächtiger werdenden Formatierungssprache CSS und dessen neuster Version 3 sind heute gestalterische Effekte möglich, welche bis vor Kurzem nur mittels umständlicher Einbindung von Grafikdateien (Stichwort Schlagschatten \cite{css-box-shadow} oder Farbverlauf \cite{css-gradient}) möglich waren.

Mit der Richtlinie \emph{TP7} hält Stefan Tilkov Software Entwickler dazu an, die Werkzeuge welche vom Internetbrowser angeboten werden, gewinnbringend zu nutzen.


\subsection*{Geplante Umsetzung}

In der Beispielapplikation sollen gezielt HTML 5 Features verwendet werden:

\begin{itemize}
	\item Semantisch korrekte Tags (\emph{<header>}, \emph{<section>} etc.) \cite{SemanticHTML}
	\item Formularvalidierung \cite{HTML5Forms}
\end{itemize}

Das entstehende, semantisch korrekte HTML Markup soll mit CSS 3 gestaltet werden. Neue Möglichkeiten zur grafischen Darstellung sollen ausgenutzt werden.

Bei der Entwicklung der Front- als auch Backend-Komponente muss zwingend darauf geachtet werden, dass die Browser-Funktionen \emph{Vorwärts}, \emph{Zurück} und \emph{Aktualisieren} zu keinem ungewünschten resp. unerwarteten Verhalten führen (siehe hierzu auch Abschnitt \ref{sec:principle-rp10-browser-controls} ``\nameref{sec:principle-rp10-browser-controls}'').


\subsection*{Konkrete Umsetzung}



\subsection*{Diskussion}


