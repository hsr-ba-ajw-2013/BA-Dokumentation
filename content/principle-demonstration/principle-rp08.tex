\section{RP8 Cookies}
\label{sec:principle-rp8-cookies}
Die ROCA Richtlinie RP8 legt fest, dass Cookies lediglich zur Authentifizierung oder zur statistischen Analyse (Tracking) eines Benutzers verwendet werden soll.

\subsection*{Geplante Umsetzung}
Die Beispielapplikation soll lediglich im Bereich der Anmeldung des Benutzers auf Cookies zurückgreifen.

\subsection*{Konkrete Umsetzung}
Das Geplante konnte mit Erfolg umgesetzt werden.
Um einem User eine Session zuzuweisen zu können verwendet Express.js \cite{Expressjs} die \gls{Middleware} ``\emph{connect-session}'' \cite{ConnectSession}.

Konnte ein Benutzer erfolgreich authentifiziert werden, sendet der Server dem Client ein Cookie mit einer eindeutigen Session-ID. Diese ID muss mit allen künftigen Requests mitgeschickt werden.

In der Beispielapplikation wird die Session-Middleware wie in Quelltext \ref{lst:connect-session-middleware} initialisiert:

\begin{lstlisting}[language=JavaScript, caption=Connect Session Middleware \cite{RoomiesMiddlewareHttp}, label=lst:connect-session-middleware, firstnumber=31]
app.use(express.session({
	store: new SequelizeStore({
		db: db
	})
	, secret: config.sessionSecret
}));
\end{lstlisting}

Der ``SequelizeStore'' \cite{SequelizeStore} übernimmt die Speicherung der Session-Daten innerhalb der Datenbank.

\subsection*{Diskussion}
Das Einbinden von mehr Informationen in Cookies hat mehrere Nachteile:
\begin{itemize}
	\item Sicherheit \\
		Wenn Benutzername und Passwort in ein Cookie gespeichert werden kann dies ohne Probleme von Drittpersonen mitgelesen werden. Falls HTTPS eingesetzt (auch wenn dies nur direkt im Browser geschieht) bildet dies ein nicht tragbares Risiko. \\
		Ausserdem ist es einem Benutzer möglich seine Cookies abzuändern. Dies könnte nur mithilfe von Validierung und dem Einsetzen eines \glspl{MAC} erkennbar gemacht werden.
	\item Datenmenge \\
		Cookies werden mit jeder Anfrage und jeder Antwort zwischen Client und Server mitgeschickt. Zwar ist die Grösse eines Cookies beschränkt, trotzdem wird die Datenmenge unnötig grösser.
\end{itemize}

Aus diesen Gründen sollte unbedingt darauf verzichtet werden, Cookies für andere Zwecke als für die wiederkehrende Authentifizierung über eine Session-ID oder für das Tracking zu verwenden.