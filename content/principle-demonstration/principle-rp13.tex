\section{RP13 Progressive Enhancement}
\label{sec:principle-rp13-progressive-enhancement}

Die Client-Technologien HTML und CSS haben sich über Jahre weiterentwickelt. Dies auch meistens mit dem Hintergedanken ``Progressive Enhancement'' oder Rückwärtskompatibilität. Das bedeutet, dass die Entwicklung vielfach versucht hat, Rücksicht auf die älteren Browserversionen zu nehmen.
Deutlich sieht man diese Rückwärtskompatibilität beim neuen HTML5 Standard. Neue Formularfeldtypen wie ``tel'' oder ``email''wurden eingeführt. Browser die HTML5 nicht unterstützen, wechseln bei einem für sie unbekannten Typ zum normalen Textfeld zurück.

\begin{lstlisting}[language=HTML, caption={Formularfeld mit HTML5, welches eine Telefonnummer erwartet}, label={lst:html5TelInput}]
<input type="tel" name="telefon">
\end{lstlisting}

Natürlich gibt es keine Regel ohne Ausnahme. Manche Tags, die im HTML 5 Standard eingeführt wurden, werden überhaupt nicht beachtet. Dies kann vor allem bei Versionen \emph{kleiner Neun} des Internet Explorers beobachtet werden.

\subsection*{Geplante Umsetzung}
Die Beispielapplikation soll, wie bereits in Kapitel \ref{sec:requirments-engineering-nonfunctionals} erwähnt, folgende Versionen unterstützen: Internet Explorer 8 und höher, Chrome 25 und höher, Firefox 19 und höher und Safari 6 und höher.

Auch Browser auf Smartphones sollten nicht vernachlässigt werden. Hierfür sollen Safari 6 und höher und Android Browser 4.0 und höher unterstützt werden.

\subsection*{Konkrete Umsetzung}
Da nicht alle geplanten und zu unterstützenden Browser HTML5 interpretieren können, musste ein kleiner Trick angewendet werden, um so die Rückwärtskompatibilität zu erhöhen. Dieser Trick heisst ``modernizr'' \cite{modernizr}, eine JavaScript-Bibliothek, welche gezielt den verwendeten Browser auf seinen Funktions- und Unterstützungsumfang von HTML und CSS überprüft. Das Resultat wird dann in einem JavaScript-Objekt gespeichert und zur Verfügung gestellt.  Zusätzlich läuft modernizr in eine kleine Schleife, um die neuen Tags (head, section, article, nav, u.w.) zu aktivieren. Das bedeutet, dass diese Tags nicht mit einem ``div''-Tag ersetzt werden müssen, wie dies sonst der Fall wäre.

Um diese Bibliothek einzubinden, hat man nichts anderes zu tun als das Skript im Header hinzuzufügen.

\begin{lstlisting}[language=HTML, caption=Einbinden von modernizr \cite{roomiesLayout}, firstnumber=12, label=lst:mdernizrLayoutServer]
<script src="/javascripts/lib/custom.modernizr.js"></script>
\end{lstlisting}

Mit ``modernizr'' konnte erreicht werden, dass in allen geplanten Browser alle Elemente erscheinen, wenn auch nicht immer korrekt. Nicht immer korrekt, weil im Internet Explorer 8 die Darstellung des Bildes im oberen linken Rand nicht dem entspricht, was erwartet wurde.

\begin{figure}[H]
	\centering
	\includegraphics[width=12cm]{content/principle-demonstration/images/progressive-enhancement-ie8.png}
	\caption{Fehlerhafte Darstellung im Internet Explorer 8}
	\label{fig:iossafari-datepicker}
\end{figure}

\subsection*{Diskussion}
Abhängig vom Zielpublikum der zu erstellenden Webapplikation gewinnt oder verliert die Unterstützung älterer Browser und somit ``Progressive Enhancement'' an Wichtigkeit. Mithilfe von Tools wie \emph{modernizr} kann Kompatibilität einfacher gewährleistet werden. Die Implementation der Beispielapplikation hat jedoch bewiesen, dass diese Werkzeuge kein Allheilmittel darstellen.

Manuelles Testen und insbesondere eventuelle manuelle Korrekturen sind und bleiben wichtig.
\\ \\
Das Projektteam ist sich einig: \emph{Progressive Enhancement} ist wichtig. Insbesondere die Sicherstellung der Kompatibilität zu älteren Versionen von Internet Explorer stellt immer noch eine Herausforderung dar und verursacht meist immense Kosten.

Weiter ist das Projektteam der Meinung, dass diese Probleme in Zukunft zwar nicht komplett wegfallen werden, aber immerhin reduziert werden könnten.

Microsoft verfolgt mit Internet Explorer 10 eine ähnliche Updatestrategie wie Mozilla und Google mit ihren Browsern: Häufige Veröffentlichung von Updates in kurzen Abständen \cite{MicrosoftQuickensIEReleaseCycle} ermöglichen Microsoft künftig die vermehrte Partizipation bei der Entwicklung von Web-Standards. In absehbarer Zeit wird dies Webapplikationsentwicklern helfen, bestmögliche Kompatibilität zwischen verschiedenen Browsern innerhalb kürzester Zeit zu gewährleisten.