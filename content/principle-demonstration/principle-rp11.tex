\section{RP11 POSH}
\label{sec:principle-rp11-posh}

``POSH'', ausgeschrieben ``Plain Old Semantic HTML'', bezeichnet das Erstellen von HTML-Seiten mithilfe von semantischen Elementen. \cite{SemanticHTML}

Semantisches HTML bietet den Vorteil, dass ``\glspl{SearchEngineSpider}'' die Seite besser verstehen, interpretieren und kategorisieren können. Wird eine Seite gut kategorisiert erscheint sie eher bei einer Suche in einer Suchmaschine wie \emph{Google} oder \emph{Bing}.

\subsection*{Geplante Umsetzung}
Das HTML Markup der Beispielapplikation sollen eine klare und logische Struktur aufweisen.

Alle Seiten haben einen Titel, eine Überschrift und entsprechenden Inhalt.

Für Überschriften werden die Tags \emph{h1}, für die grösste und wichtigste Überschrift, bis \emph{h6} für untergeordnete.

Tabellen sollen für tabellarische Daten verwendet werden und nicht zur gestalterischen Strukturierung einer Seite.


\subsection*{Konkrete Umsetzung}
Um semantische Inkorrektheiten durch Wiederholungsfehler zu reduzieren, wurde in \emph{Roomies} mit Templates gearbeitet. So können bspw. Menüs oder Fusszeilen einmalig definiert und wiederverwendet werden. Dies geht sehr stark nach dem ``\gls{DRY}'' Prinzip. Fehler werden verhindert, indem man sie gar nicht zu schreiben hat.

Quelltext \ref{lst:layoutDefinition} zeigt einen Auschnitt des HTML-Markups vom Haupttemplate von \emph{Roomies}''. Die Seitenstruktur ist klar zu erkennen: Oben beginnt der Header mit dem Menü, gefolgt von einem Platzhalter für allfällige Fehler-, Warnungs- oder Informationsmeldungen. Weiter ist der Hauptbereich mit der ID ``main'' ersichtlich, welcher zur Laufzeit mit eigentlichen Applikationsinhalten gefüllt wird. Abschliessend steht ein semantisch korrektes \emph{footer}-Element, welches final den Link zur Abmeldung des Benutzers enthalten wird.

\begin{lstlisting}[language=HTML, caption=Layout Definition \cite{roomiesHtmlSkeleton}, label=lst:layoutDefinition, firstnumber=27]
<body>
	<header id="menu"></header>
	<div id="flash-messages" class="row flash-messages"></div>
	<section id="main"></section>
	<footer id="footer"></footer>
</body>
\end{lstlisting}

\subsection*{Diskussion}
Semantisches HTML ist aus zwei Gründen eine gute Idee:
\begin{itemize}
	\item Suchmaschinenoptimierung
	\item Barrierefreiheit
\end{itemize}

\subsubsection*{Suchmaschinenoptimierung}
Suchmaschinen wenden komplexe Algorithmen an, um Webseiten zu analysieren und in verwertbare Suchresultate zu transformieren. Ein Entwickler von Webseiten und -applikationen kann durch die Verwendung von semantisch korrekten Tag-Elementen viel zur Optimierung des Suchprozesses beitragen. Diese ermöglichen den Suchmaschinen ein fehlerfreies Interpretieren der untersuchten Informationen.
\\ \\
Ergänzend dazu helfen Standards wie \emph{schema.org} \cite{SchemaOrg}. Unterstützt von Google, Yahoo und Microsoft, hat dieser zum Ziel, die Maschinenlesbarkeit von Webseiten zu erhöhen und spezifischen Teilen einer Seite analyisierbaren \emph{Sinn} zu geben.

Mithilfe von \emph{schema.org} können bspw. Attribute von Produkten (Name, Preis etc.) gekennzeichnet und gezielt für Suchmaschinen verwertbar gemacht werden. Sucht ein Benutzer nun nach Produkten, kann die Suchmaschine die nun strukturierten Informationen besser durchsuchen und massgeschneiderte Suchergebnisse dem Benutzer präsentieren.

Für Dienstanbieter ergibt sich so ein enormes Potential zur Monetarisierung.

\subsubsection*{Barrierefreiheit}
Nebst der Verbesserung der Sichtbarkeit bei Suchmaschinen ist ein semantisch korrektes HTML-Markup auch aus Sicht der Barrierefreiheit sinnvoll.

Ein signifikanter Teil der Bevölkerung ist auf barrierefreie Webseiten angewiesen \cite{BarrierefreiesInternet}. Das Erstellen von barrierefreien Webseiten unterstützt Geräte wie \emph{Screen Reader} Inhalte zu analysieren und in einem für den gehandicapten Benutzer gerechten Format zugänglich zu machen.

Semantische Tags sind einer der wichtigsten Schritte auf dem Weg zu einem barrierefreien Webangebot.
\\ \\
Das Projektteam ist aufgrund der genannten Gründe der Meinung, dass semantische Tags und somit \emph{RP11 POSH} sehr wichtig ist und auch für eine Webapplikation umgesetzt werden sollte.