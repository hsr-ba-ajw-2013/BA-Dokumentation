\section{RP17 Static Assets}
\label{sec:principle-rp17-static-assets}

Die ``Static Assets'' ROCA Richtlinie will dass jeglicher JavaScript Code oder CSS Stylesheets für den Client von statischer Natur ist. Dies bedeutet, dass der Server keine dynamische Generierung von eben diesem Code vornimmt.

\subsection*{Geplante Umsetzung}
\subsubsection*{CSS Stylesheets}
Die verwendeten CSS Stylesheets sollen mit dem SASS Präprozessor \cite{SASS} erstellt werden. Zwar muss der eigentliche Stylesheet Code vorneweg einmalig übersetzt werden, die dadurch entstehenden Vorteile beim Entwickeln der Formatierungsinformationen sind jedoch bei Weitem grösser.

Da zudem statische SASS-Quelldateien übersetzt werden, kann der Anspruch keine dynamischen CSS Formatierungen zu generieren befriedigt werden.

\subsubsection*{Clientside JavaScript}
Die JavaScript Applikation für den Client soll auf verschiedene Dateien aufgeteilt werden. Für den späteren produktiven Betrieb ist das Übertragen vieler kleinen Quellcode-Dateien jedoch nicht effektiv.

Aus diesem Grund sollen auf dem Backend alle Client-Quellcode-Dateien zusammengefasst werden können und mit gängigen Methoden zur schnellstmöglichen Übertragung über das Internet optimiert werden.

Ähnlich wie beim SASS-Präprozessor soll auch hier kein dynamischer Code entstehen. Es wird lediglich eine Optimierung der zu übertragenden Informationen vorgenommen.


\subsection*{Konkrete Umsetzung}
Beide geplanten Umsetzungen konnten erfolgreich implementiert werden.

\subsubsection*{CSS Stylesheets}
Zur Entwicklungszeit werden die SASS-Quellcodes bei jedem Start der Beispielapplikation neu umgewandelt. Möchte man die Applikation produktiv verwenden, kann die initial beim Ausführen des \emph{install.sh} Skripts durchgeführt werden, oder gezielt über den Kommandozeilenbefehl \emph{make precompile-sass}.

Die daraus entstehende Datei kann anschliessend ohne weitere Veränderungen vom Webserver an den Browser des Benutzers übertragen werden.

\subsubsection*{Clientside JavaScript}
Mit \emph{barefoot} \cite{Barefoot} kann zwar der Server-Quellcode auch für die Client-Applikation verwendet werden. Jedoch stellt sich auch hier die Herausforderung, über verschiedenen Dateien verteilte Programmlogik in eine einzige, grössenoptimierte JavaScript-Date zusammenzufassen.

Zu diesem Zweck verwendet \emph{barefoot} fix die \emph{browserify-middleware} für Express.JS \cite{browserifymiddleware}. Diese Komponente ermittelt anhand vorhandener \emph{require}-Statements im Quelltext, welche CommonJS Module \cite{commonjsmodules} zusammengefasst und bereitgestellt werden müssen.

Zusammen mit einigen Zeilen \gls{Boilerplate}-Code entsteht so eine eigenständige JavaScript-Datei welche als Ganzes an den Client ausgeliefert werden kann. Falls konfiguriert, wird diese zudem von Kommentaren und unnötigen Füllzeichen befreit und mittels Gzip \cite{gzip} komprimiert. Dies bringt insbesondere im produktiven Betrieb imense Vorteile.

\begin{lstlisting}[language=JavaScript, firstnumber=95, caption=Konfiguration der browserify Middleware \cite{RoomiesExampleConfig}, label=lst:configBrowserifyMiddleware]
/* Browserify: */
// Roomies uses browserify to package all neceessary CommonJS modules into
// one big JavaScript file when delivering them to the client.
// Use these settings to customize how that app.js file is created.
//
// More information about these settings is available here:
// https://github.com/ForbesLindesay/browserify-middleware
, clientsideJavaScriptOptimizations: {
	debug: false
	, gzip: true
	, minify: true
}
\end{lstlisting}


\subsection*{Diskussion}
Das Projektteam ist davon überzeugt, dass die umgesetzten und aufgezeigten Methoden und Mechanismen für moderne Webapplikationen ein extrem hilfreiches Werkzeug sind.

Im Bereich der clientseitigen JavaScript Entwicklung ermöglichen das erwähnte \emph{browserify-middleware} oder andere Bibliotheken wie RequireJS \cite{requirejs} erst das effiziente verteilen von Quellcode auf verschiedene Dateien.

In die gleiche Richtung strebt SASS: Es ergänzt CSS um viele nützliche Features wie Variablen und Mixins. Daneben ermöglicht es aber, entsprechende Bibliotheken vorausgesetzt, eine extreme Vereinfachung der Entwicklung von CSS-Stylesheets, welche mit verschiedenen Browsern kompatibel ist.

Sollten die vorgestellten Techniken für jede Webapplikation verwendet werden?

Das Projektteam ist der Meinung, dass ab einem gewissen Projektumfang uneingeschränkt auf JavaScript-Modularisierungsmethoden und CSS-Präprozessoren gesetzt werden sollte.