\section{RP12 Accessibility}
\label{sec:principle-rp12-accessibility}
Aufbauend auf dem Prinzip ``\nameref{sec:principle-rp11-posh}'' besagt das Prinzip ``Accessibility'', dass alle Seiten für Hilfesoftware wie Screen Reader zugänglich sein müssen.

Der Begriff Barrierefreiheit schliesst Benutzer mit und ohne Behinderungen mit ein. Alle sollen auf die Seiten Zugang haben. Sei es Sehbehinderte, Farbenblinde, Leuten mit altersbedingten oder mit technischen Einschränkungen, Menschen mit kognitiven Behinderungen oder Gehörlose \cite{BarrierefreiesInternet}. Das ROCA-Prinzip befasst sich jedoch nur mit der Zugänglichkeit über Hilfeanwendungen. Die Problematik Farbenblinde mit der richtigen Farbenwahl zu unterstützen wird nicht behandelt. Ebenfalls das Navigieren durch die Seiten mit der Tastatur wird nicht speziell hervorgehoben.

\subsection*{Geplante Umsetzung}
Damit andere User Agents bzw. Screen Readers die Seite interpretieren können, ist es wichtig dass die Seiten semantisch korrekt sind. Deswegen ist es umso wichtiger das Prinzip ``\nameref{sec:principle-rp11-posh}'' sorgfältig umzusetzen.

\subsection*{Konkrete Umsetzung}
Wie in ``\nameref{sec:principle-rp11-posh}'' erwähnt, wurden semantisch korrekte Tags für die Beispielapplikation ``Roomies'' verwendet.
Weitere Implementationen bezüglich Access Keys, Akronyme etc. wurden wie Geplant nicht implementiert. Dies einerseits aus Zeitgründen, andererseits (z.B. bei Akronymen) auch weil es nicht benötigt wurde.

\subsection*{Diskussion}
In ``\nameref{sec:principle-rp11-posh}'' wurde diskutiert, wie wichtig die Barrierefreiheit ist. Das Prinzip ``\nameref{sec:principle-rp12-accessibility}'' verstärkt jenes noch mehr, geht jedoch in eine ähnliche Richtung.

Wie bereits in der Einleitung zu diesem Prinzip erwähnt, gehören barrierefreie Richtlinien wie die Beachtung von Farbenblindheit nicht zu diesem Prinzip. Dies ist wahrscheinlich so gewollt, trotzdem ist sich das Projektteam einig, dass dies ebenfalls sehr wichtig ist.
