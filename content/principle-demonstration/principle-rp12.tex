\section{RP12 Accessibility}
\label{sec:principle-rp12-accessibility}
Aufbauend auf das Prinzip \nameref{sec:principle-rp11-posh} besagt das Prinzip ``Accessibility'', dass alle Seiten für Hilfesoftware wie Screen Readers zugänglich sein müssen.

Der Begriff Barrierefreiheit schliesst Benutzer mit und ohne Behinderungen mit ein. Alle sollen auf die Seiten Zugang haben. Sei es Sehbehinderten, Farbenblinden, Leuten mit altersbedingten oder mit technischen Einschränkungen, Menschen mit kognitiven Behinderungen oder Gehörlose. Das ROCA-Prinzip befasst sich, jedoch nur mit der Zugänglichkeit über Hilfeanwendungen. Die Problematik Farbenblinden mir der richtigen Farbenwahl zu unterstützen wird nicht behandelt. Ebenfalls das Navigieren durch die Seiten mit der Tastatur wird nicht speziell hervorgehoben.

\subsection*{Geplante Umsetzung}
Damit andere User Agents bzw. Screen Readers die Seite interpretieren können, ist es wichtig dass die Seiten semantisch korrekt sind. Deswegen ist es umso wichtiger das Prinzip \nameref{sec:principle-rp11-posh} sorgfältig umzusetzen.

\subsection*{Konkrete Umsetzung}


\subsection*{Diskussion}
-
