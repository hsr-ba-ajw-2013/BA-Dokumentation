\section{RP12 Accessibility}
\label{sec:principle-rp12-accessibility}
Aufbauend auf dem Prinzip ``\nameref{sec:principle-rp11-posh}'' besagt das Prinzip \emph{Accessibility}, dass alle Seiten für Hilfesoftware/-geräte wie \emph{Screen Reader} oder \emph{Braillezeile} zugänglich sein müssen.


``Barrierefreiheit schliesst sowohl Menschen mit und ohne Behinderungen als auch Benutzer mit technischen (Textbrowser oder PDA) oder altersbedingten Einschränkungen (Sehschwächen) ... ein.'' \cite{BarrierefreiesInternet}

\emph{RP12 Accessibility} beschränkt sich dabei auf Zugänglichkeit über Hilfeanwendungen. So wird die in der Definition erwähnte Problematik \emph{Farbblindheit} (richtige Farbenwahl etc.) oder die Navigation ohne Maus nicht abgedeckt.

\subsection*{Geplante Umsetzung}
Damit Hilfesoft- und Hilfehardware das User Interface der Beispielapplikation interpretieren kann, soll das HTML Markup, wie unter \ref{sec:principle-rp11-posh} ``\nameref{sec:principle-rp11-posh}'' erklärt, eine semantisch korrekte Struktur aufweisen.

\subsection*{Konkrete Umsetzung}

\emph{Roomies} verwendet, wo nötig, semantisch korrekte HTML-Elemente.

\subsection*{Diskussion}
In ``\nameref{sec:principle-rp11-posh}'' wird ausführlich über die Wichtigkeit eines barrierefreien Internets diskutiert. Das Prinzip \emph{RP12 Accessibility} strebt durch die notwendigen Vorkehrungen in die selbe Richtung, hat aber die Unterstützung gehandicapter Personen zum Ziel.

Leider gehört die Berücksichtigung von Farbenblindheit o.Ä. nicht zum Repertoire von \emph{RP12}. Für das Projektteam gehören aber auch diese Themen klar zum Aufgabenbereich der Richtlinie \emph{Accessibility}.

Das Projektteam schlägt darum die Ergänzung von \emph{RP12 Accessibility} um die erwähnten Punkte vor. Machen es die Anforderung nötig, empfiehlt das Projektteam die Umsetzung von \emph{RP12}.