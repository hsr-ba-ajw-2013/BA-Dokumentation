\section{RP4 Link}
\label{sec:principle-rp4-link}

\subsection*{Geplante Umsetzung}
Jede Seite innerhalb einer Web-Applikation muss mit einer eindeutigen URL adressierbar
sein.

Dies ist eines der ersten Paradigmas des Webs und auch eines der wichtigstens. In der Vergangenheit
ist es öfters vorgekommen, dass neuere Applikationen auf diesen ``Komfort'' verzichtet haben.
Heutzutage ist es aber u.a. dank der History API \cite{HistoryAPI} einiges
einfacher geworden, eindeutige URLs auch in JavaScript-lastigen Applikationen zu
verwenden.


\subsection*{Konkrete Umsetzung}
Weil die Beispielapplikation so oder so den REST \cite{REST} Prinzipien entspricht,
ist diese Richtlinie ein muss.

Jede URL entspricht einer eindeutigen Ressource und kann angesprochen werden.
Mithilfe der History API \cite{HistoryAPI} wird auch auf dem Browser die URL
geändert, obwohl u.U. nur ein AJAX-Request gemacht wird.

\subsection*{Diskussion}
Schon Tim Berners-Lee hat vor Jahren geschrieben: ``Cool URIs don't change'' \cite{CoolURIsTBL}.
Damit eine URL ``cool'' ist, muss sie zuerst mal vorhanden sein.

Auch aus einem weiteren Grund sind URLs nur dann ``cool'', wenn sie vorhanden sind
und niemals ändern: Mit den heuten Möglichkeiten des ``Sharing'' auf diversen Sozialen
Netzwerken, muss es möglich sein, direkt auf die momentane Seite zu verlinken.