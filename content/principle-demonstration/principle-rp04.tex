\section{RP4 Link}
\label{sec:principle-rp4-link}

Das ``Link''-Prinzip ist eng verbunden mit Kapitel \ref{sec:principle-rp10-browser-controls}, ``\nameref{sec:principle-rp10-browser-controls}'' und stellt die Anforderung, dass jede Seite eindeutig per Link identifizierbar sein muss.

Wenn Web-Applikationen diesem Prinzip nicht folgen (bspw. Digitec \cite{Digitec}), dann können Seiten nicht per Link mit Freunden geteilt werden.
Dies ist für einen Nutzer häufig nicht nachvollziehbar und die User Expierence leidet damit.

Bevor Browser die History API \cite{HistoryAPI} implementiert haben, konnte das für JavaScript-lastige Webseiten nur schwer eingehalten werden.

\subsection*{Geplante Umsetzung}
Die Beispielapplikation soll eindeutige URLs für Ressourcen aufweisen. Dies soll sowohl für die REST-API gelten, wie auch für die Webseite selber.

\subsection*{Konkrete Umsetzung}
Weil die Beispielapplikation so oder so den REST \cite{REST} Prinzipien entspricht, ist diese Richtlinie ein Muss.

Jede URL entspricht einer eindeutigen Ressource und kann angesprochen werden.
Mithilfe der History API \cite{HistoryAPI} wird auch auf dem Browser die URL
geändert, obwohl u.U. nur ein AJAX-Request gemacht wird.

\subsection*{Diskussion}
Schon Tim Berners-Lee hat vor Jahren geschrieben: ``Cool URIs don't change'' \cite{CoolURIsTBL}.
Damit eine URL ``cool'' ist, muss sie zuerst mal vorhanden und funktional sein.

Auch aus einem weiteren Grund sind URLs nur dann ``cool'', wenn sie vorhanden sind
und niemals ändern: Mit den heuten Möglichkeiten des ``Sharing'' auf diversen Sozialen
Netzwerken muss es möglich sein, direkt auf die momentane Seite verlinken zu können.