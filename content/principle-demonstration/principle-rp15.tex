\section{RP15 No Duplication}
\label{sec:principle-rp15-no-duplication}

Um \emph{RP15} einfach erklären zu können, soll folgendes Beispiel dienen:

\begin{quotation}
In einer Webapplikation sollen die Benutzereingaben aus einem Formular auf formale Korrektheit hin geprüft werden. Beim Versenden des Formulars werden dazu die übertragenen Informationen in der Backendkomponente überprüft und ggf. mit einer Fehlermeldung zurückgewiesen.

Für eine Verbesserung der User Experience soll nun bereits vor dem Versenden des Formulars im Frontend eine Prüfung der Eingaben gemacht werden. Da die Backendkomponente mit PHP implementiert wurde, entscheidet der zuständige Entwickler den bestehenden Code mit JavaScript auf den Client zu portieren.
\end{quotation}

Das Beispiel verdeutlicht, welche Stellen einer Webapplikation tendenziell besonders anfällig für duplizierten Quelltext sein können.

Die Richtlinie 15 \emph{No Duplication} soll die Erstellung von doppelten Codefragmenten minimieren resp. komplett verhindern.


\subsection*{Geplante Umsetzung}

Die Aufhebung der Sprachbarriere, welche durch Verwendung von JavaScript sowohl auf Client- als auch auf Serverseite resultiert, soll bereits zu einem grossen Teil zur Vermeidung von doppelten Codefragmenten beitragen.

Das Projektteam will zudem durch geschickte Erstellung von Modulen die Wiederverwendbarkeit des enthaltenen Quelltexts erleichtern.


\subsection*{Konkrete Umsetzung}

Mit der durchgängigen Verwendung von \emph{barefoot} \cite{Barefoot} für die Implementation der Beispielapplikation konnte der Anspruch von \emph{RP15 No Duplication} besser als erwartet umgesetzt werden.

Wie unter ``Konkrete Umsetzung'' im Abschnitt \ref{sec:principle-rp14-unobtrusive-javascript} ``\nameref{sec:principle-rp14-unobtrusive-javascript}'' bereits ausführlich beschrieben wurde, konnte eine durchgängige und duplikatfreie Codebasis umgesetzt werden.


\subsection*{Diskussion}

Unabhängig von der Entwicklung von Webapplikationen kennt der Software Engineer das Prinzip von \emph{Don't repeat yourself}. Dementsprechend bietet \emph{RP15 No Duplication} eigentlich keine grundlegenden Neuerungen. Wie in der Beispielapplikation aufgezeigt werden konnte, erleichtert die Verwendung der gleichen Programmiersprache in Front- und Backend die Umsetzung von \emph{RP15} zudem zusätzlich.

Lassen es daher die Umstände zu, empfiehlt das Projektteam aufgrund des besser wartbaren Codes die Umsetzung von \emph{RP15 No Duplication} uneingeschränkt.