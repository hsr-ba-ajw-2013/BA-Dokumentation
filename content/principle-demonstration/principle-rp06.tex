\section{RP6 Should-Formats}
\label{sec:principle-rp6-should-formats}
Wie bereits der Name ROCA sagt, sollen Anwendungen ressourcenorientiert sein. Damit diese Ressourcen auch in anderen Anwendungen als in einem Browser verwendet werden können, muss das erzeugte HTML entweder maschinenlesbar sein oder es müssen alternative Formate (wie z.B. JSON, XML usw.) verfügbar gemacht werden.

\subsection*{Geplante Umsetzung}
Im Abschnitt \ref{sec:principle-rp1-rest} \nameref{sec:principle-rp1-rest} wird darauf hingewiesen, dass eine REST API mit JSON als Datenformat verwendet werden soll.

\subsection*{Konkrete Umsetzung}
Das Abschnitt \ref{sec:principle-rp1-rest} \nameref{sec:principle-rp1-rest} zeigt, dass eine API umgesetzt werden konnte.
Die Schnittstelle liefert JSON als Ausgabeformat. Dies wird garantiert, indem ein Objekt an die Express.js Response \cite{ExpressjsResSend} zurückgeliefert wird und Express.js automatisch JavaScript Objekte in JSON umwandelt.

\begin{lstlisting}[language=JavaScript, caption=Community API getCommunityWithSlug \cite{roomiesCommunityApiExample}, label=lst:restApiCommunitySlug, firstnumber=333, escapeinside={@}{@}]
/** Function: getCommunityWithSlug
 * Looks up a community with a specific slug.
 *
 * Parameters:
 *   (Function) success - Callback on success. Will pass the community data as
 *                        first argument.
 *   (Function) error - Callback in case of an error
 *   (String) slug - The slug of the community to look for.
 */
function getCommunityWithSlug(success, error, slug) {
	debug('get community with slug');

	var communityDao = getCommunityDao.call(this);

	communityDao.find({ where: { slug: slug, enabled: true }})
		.success(function findResult(community) {
			if(!_.isNull(community)) {
				success(community.dataValues);
			} else {
				error(new errors.NotFoundError('Community with slug ' + slug +
					'does not exist.'));
			}
		})
		.error(function daoError(err) {
			error(err);
		});
}
\end{lstlisting}

Quelltext \ref{lst:restApiCommunitySlug} zeigt, wie das Community Objekt an den Success-Handler weitergegeben wird. Der Success-Handler ist von ``barefoot'' definiert und reicht den übergebenen Parameter an Express.js weiter, welcher daraus einen JSON String generiert.

\subsection*{Diskussion}
