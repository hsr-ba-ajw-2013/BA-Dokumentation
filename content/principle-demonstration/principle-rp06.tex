\section{RP6 Should-Formats}
\label{sec:principle-rp6-should-formats}
Wie es bereits im Name ROCA (Resource-oriented Client Architecture) steht, die Anwendung soll Ressourcen-orientiert sein. Damit diese Ressourcen auch in anderen Anwendungen als in einem Browser verwendet werden können, muss das erzeugte HTML entweder maschinenlesbar sein oder es müssen alternative Formate (wie z.B. JSON und/oder XML) verfügbar gemacht werden.

\subsection*{Geplante Umsetzung}
Die ``öffentlichen'' Daten sollen über eine API als JSON verfügbar sein. Diese können über eine einheitliche URL abgerufen werden.


\subsection*{Konkrete Umsetzung}
Der Server bietet eine API für jegliche Ressource, welche in einer View benötigt wird. Diese API entspricht einer RESTful-Architektur. Die Daten können über die bekannte GET HTTP-Request-Methode abgefragt werden, mit POST können neue Datensätze hinzugefügt werden, mit PUT verändert und natürlich mit DELETE gelöscht werden.



\subsection*{Diskussion}
