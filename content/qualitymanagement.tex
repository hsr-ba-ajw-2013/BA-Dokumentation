\chapter{Qualitätsmanagement}


\section{Coding Guideline}
Die Coding Guidelines werden im Kapitel \ref{sec:coding-guidelines} \nameref{sec:coding-guidelines} behandelt und beschrieben.


\section{Test Guideline}
Die Arbeit soll nach der \gls{TDD}-Methodik entwickelt werden.

\subsection*{Testing Framework}
Als Testing Framework soll Mocha \cite{Mocha} dienen. Mocha ermöglicht asynchrone Tests einfach zu schreiben und bittet flexible Raporte.
\newline

Als kleine Hilfe wird Chai.JS \cite{ChaiJS} eingesetzt. Chai bittet drei angenehme Varianten Asserts zu schreiben. Die ``Should''-Variante soll für die Tests verwendet werden. Diese Variante ist einfach zu lesen und zu interpretieren und soll so die verschiedenen Tests vereinfachen. \newline

//TODO: einfacherer code beispiel, maybe?
\begin{lstlisting}[language=JavaScript, caption=Zusammenspiel aus Mocha und Chai.js Should]
var chai = require('chai')
	,config = require('../../config_test')
	,db = require('../../lib/db');

	chai.should();

	var sequelize = db(config);
	describe('Get all communities', function(){
		it('should not have any communities', function(done){
			(function() {
				sequelize.daoFactoryManager.getDAO('Community').all().success(function(communities) {
					communities.should.have.length(0);
					done();
				}).error(function(error) {
					throw(error);
				})
			}).should.not.throw();
		});
	});
\end{lstlisting}


\section{Continuous Integration}
Dieses Projekt verwendet Travis CI als Continuous Integration Lösung.

Beide Git Repositories (Code \& Thesis) verfügen über einen \gls{PushHook} welcher automatisch einen Build im CI-System auslöst.


\section{Code Reviews}
Laufend sollen Code Reviews durchgeführt werden. Deshalb wurde pro Projektphase ein Arbeitspacket ``Code Review'' dem Team zugeordnet.


\section{Code Coverage Analysis}
Die Testabdeckung der Software wird von JSCover \cite{JSCover} unterstüzt.
\newline TODO: more infos about JSCover???
