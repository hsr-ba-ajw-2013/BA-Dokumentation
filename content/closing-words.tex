\chapter{Schlussfolgerung}

\section*{Roll Up}
In den vorangegangenen drei Kapiteln hat sich das Projektteam ausführlich mit verschiedenen Aspekten modernen Webapplikationen befasst. Es wurde eine Zusammenstellung von Architekturrichtlinien analysiert und dabei jedes Konzept aus unterschiedlichen Perspektiven sowohl theoretisch als auch praktisch bewertet.

Nach der Evaluation dreier Technologiekandidaten und mehrerer Bibliotheken wurde unter Verwendung von \emph{JavaScript} und \emph{Express.js} mit \emph{Roomies} eine Applikation zur Veranschaulichung der ausgewählten Konzepte entwickelt.

Mit den ersten Arbeiten an der Beispielapplikation wurde dem Projektteam schnell klar, dass die Anforderungen der \emph{Unobtrusive JavaScript}-Richtlinie einiges mehr an Aufwand verschlingen würden als zuerst angenommen wurde. Die Anstrengungen in diesem Gebiet wurden kurz vor Schluss aber mit dem wiederverwendbaren Framework \emph{barefoot} belohnt: Unter dessen Integration kann \emph{Roomies}' Quellcode ohne mehrfache Implementation sowohl auf dem Server als auch auf dem Client verwendet werden.


\section*{Fazit}
\subsection*{Kontroverse ``JavaScript''}
Während der Technologieevaluation war umstritten, ob \emph{JavaScript} wirklich für den geplanten Einsatz geeignet sei. Zwar wurden bereits Lösungen mit \emph{Node.js} in der Industrie umgesetzt, altbekannten Vorbehalte waren aber sowohl im Projekt- als auch Betreuerteam klar spürbar.

Die gemachten Erfahrungen mit der Umsetzung von \emph{Roomies} zeigen, dass sich \emph{JavaScript} sehr wohl mit bekannten Technologien wie \emph{Ruby} messen kann.

\subsection*{Architekturrichtlinien}



\section*{Ausblick}


