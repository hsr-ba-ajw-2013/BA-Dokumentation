\chapter{Schlussfolgerung}

\section*{Roll Up}
In den vorangegangenen drei Kapiteln hat sich das Projektteam ausführlich mit verschiedenen Aspekten moderner Webapplikationen befasst. Es wurde eine Zusammenstellung von Architekturrichtlinien analysiert und dabei jedes Konzept aus unterschiedlichen Perspektiven sowohl theoretisch als auch praktisch bewertet.

Nach der Evaluation dreier Technologiekandidaten und mehrerer Bibliotheken wurde unter Verwendung von \emph{JavaScript} und \emph{Express.js} mit \emph{Roomies} eine Applikation zur Veranschaulichung der ausgewählten Konzepte entwickelt.

Mit den ersten Arbeiten an der Beispielapplikation wurde dem Projektteam schnell klar, dass die Anforderungen der \emph{Unobtrusive JavaScript}-Richtlinie einiges mehr an Aufwand verschlingen würden als zuerst angenommen wurde. Die Anstrengungen in diesem Gebiet wurden mit dem wiederverwendbaren Framework \emph{barefoot} belohnt: Unter dessen Integration kann \emph{Roomies}' Quellcode ohne mehrfache Implementation sowohl auf dem Server als auch auf dem Client verwendet werden.


\section*{Fazit}

\subsection*{Architekturrichtlinien}

Aus über 25 Richtlinien, Architekturkonzepten und Prinzipien wurden während der Vorselektion insgesamt 22 ausgewählt, um in einem weiteren Schritt an dem praktischen Beispiel \emph{Roomies} demonstriert zu werden.

Bis auf einige wenige Ausnahmen konnten alle Konzepte befriedigend implementiert werden. Dabei stechen \emph{No Duplication} und \emph{Unobtrusive JavaScript} aus dem ROCA-Katalog besonders hervor:

Während der Implementation von \emph{Roomies} flossen über 40 Prozent des Gesamtaufwandes in die Umsetzung dieser Richtlinien. Insbesondere die Integration bestehender Codefragmente mit \emph{barefoot} verschlang immens Zeit.

Die Qualität von \emph{barefoot} hat von diesem Prozess hingegen sehr profitiert. In dessen aktueller Version \emph{0.0.11} sind viele Fehler behoben und wichtige Features wurden ergänzt oder neu konzipiert.

Negativ aufgefallen sind lediglich die Konzepte \emph{Auth} und \emph{Know Structure}: Letzteres war mit der Umsetzung von \emph{Unobtrusive JavaScript} nicht vereinbar, da diese beiden Prinzipien sich widersprechen. Ersteres wurde gezielt nicht umgesetzt, um den Workload auf das Team zu optimieren.

Gesamtheitlich schätzt das Projektteam die bearbeiteten Architekturrichtlinien als zweckmässig und sehr hilfreich ein. Vielerorts muss jedoch wie dokumentiert von Situation zu Situation entschieden, welchen Konzepten aufgrund der vorliegenden Anforderungen der Vorzug gewährt wird.


\subsection*{Tauglichkeit das für Unterrichtsmodul Internettechnologien}



\subsection*{Kontroverse: JavaScript}
Während der Technologieevaluation war umstritten, ob \emph{JavaScript} wirklich für den geplanten Einsatz geeignet sei. Zwar wurden bereits Lösungen mit \emph{Node.js} für den produktiven Einsatz umgesetzt, altbekannten Vorbehalte waren aber sowohl im Projekt- als auch Betreuerteam klar spürbar:

\begin{itemize}
	\item Seriöses Software Engineering (\emph{Separation Of Concerns}, \gls{TDD}, CI etc.) ist mit \emph{JavaScript} nicht möglich
	\item Schlechte Performance
	\item Keine Strukturierung von Quellcode möglich
\end{itemize}

Die gemachten Erfahrungen mit der Umsetzung von \emph{Roomies} beweisen, dass \emph{JavaScript} sich sehr wohl mit bekannten Technologien wie \emph{Ruby} messen kann. Gerade in Belangen wie Flexibilität oder Skalierbarkeit braucht es sich keinesfalls zu verstecken.

Werkzeuge wie \emph{JSHint}, \emph{Mocha}, \emph{Travis CI} oder \emph{JSCoverage} ermöglichen die Sicherstellung der Codequalität von \emph{JavaScript} Quelltext-Artefakten.

\emph{JavaScript} ist definitiv seinen Kinderschuhen entwachsen und wird Heute über die \emph{ECMAScript} Spezifikation rege weiterentwickelt.

Das Projektteam hat die Entscheidung für \emph{JavaScript} keine Sekunde bereut und kann auch nach erfolgreicher Durchführung dieser Arbeit voll und ganz hinter dieser Entscheidung stehen.


\section*{Ausblick}


