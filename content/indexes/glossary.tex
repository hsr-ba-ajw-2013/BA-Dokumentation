\newglossaryentry{nodejs}{
	name=Node.js,
	description={Node.js ist ein Framework um Javascript auf dem Server laufen zu lassen. Es ist auf der V8 Engine von Google Chrome aufgebaut}
}
\newglossaryentry{REST}{
	name=REST,
	description={Representational State Transfer, definiert von Roy Fielding in seiner Dissertation \cite{REST}}
}
\newglossaryentry{PushHook}{
	name={Push-Hook},
	description={In \emph{git} bezeichnet ein Push-Hook ein Script, welches nach jedem Commit ausgeführt wird.}
}
\newglossaryentry{Benutzer}{
	name=Benutzer,
	% FIXME
	description={Ein Benutzer .... TOOODOOO}
}
\newglossaryentry{Bewohner}{
	name=Bewohner,
	% FIXME
	description={Ein Bewohner .... TOOODOOO}
}
\newglossaryentry{Administrator}{
	name=Administrator,
	% FIXME
	description={Ein Administrator .... TOOODOOO}
}

\newglossaryentry{WG}{
	name=WG,
	% FIXME
	description={Eine WG .... TOOODOOO}
}

% TODO: Das ist ein Akronym, kein Glossary
\newglossaryentry{CI}{
	name=CI,
	description={Continuous Integration}
}

% TODO: Das ist ein Akronym, kein Glossary
\newglossaryentry{IDE}{
	name=IDE,
	description={Integrated Development Environment}
}

% TODO: Das ist ein Akronym, kein Glossary
\newglossaryentry{RUP}{
	name=RUP,
	description={Rational Unified Process; Iteratives Projektvorgehen}
}

\newglossaryentry{VM}{
	name=VM,
	description={Virtual Machine}
}

% TODO: Das ist ein Akronym, kein Glossary
\newglossaryentry{SAD}{
	name=SAD,
	description={Software Architektur Dokument}
}

\newglossaryentry{Gamification}{
	name=Gamification,
	description={Als Gamification oder Gamifizierung (seltener auch Spielifizierung) bezeichnet man die Anwendung spieltypischer Elemente und Prozesse in spielfremdem Kontext \cite{Gamification}}
}