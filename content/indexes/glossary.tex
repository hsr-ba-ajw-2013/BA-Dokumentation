\newglossaryentry{nodejs}{
	name=Node.js,
	description={Node.js ist ein Framework um Javascript auf dem Server laufen zu lassen. Es ist auf der V8 Engine von Google Chrome aufgebaut}
}
\newglossaryentry{REST}{
	name=REST,
	description={Representational State Transfer, definiert von Roy Fielding in seiner Dissertation \cite{REST}}
}
\newglossaryentry{PushHook}{
	name={Push-Hook},
	description={In \emph{git} bezeichnet ein Push-Hook ein Script, welches nach jedem Commit ausgeführt wird}
}
\newglossaryentry{Benutzer}{
	name=Benutzer,
	% FIXME
	description={Ein Benutzer .... TOOODOOO}
}
\newglossaryentry{Bewohner}{
	name=Bewohner,
	% FIXME
	description={Ein Bewohner .... TOOODOOO}
}
\newglossaryentry{Administrator}{
	name=Administrator,
	% FIXME
	description={Ein Administrator .... TOOODOOO}
}

\newglossaryentry{WG}{
	name=WG,
	% FIXME
	description={Eine WG .... TOOODOOO}
}

% TODO: Das ist ein Akronym, kein Glossary
\newglossaryentry{CI}{
	name=CI,
	description={Continuous Integration}
}

% TODO: Das ist ein Akronym, kein Glossary
\newglossaryentry{IDE}{
	name=IDE,
	description={Integrated Development Environment}
}

% TODO: Das ist ein Akronym, kein Glossary
\newglossaryentry{RUP}{
	name=RUP,
	description={Rational Unified Process; Iteratives Projektvorgehen}
}

% TODO: Das ist ein Akronym, kein Glossary
\newglossaryentry{DSL}{
	name=DSL,
	description={Domain Specific Language; Formale Sprache für Lösung von Problemen in einem bestimmten Umfeld}
}

\newglossaryentry{VM}{
	name=VM,
	description={Virtual Machine}
}

% TODO: Das ist ein Akronym, kein Glossary
\newglossaryentry{SAD}{
	name=SAD,
	description={Software Architektur Dokument}
}

% TODO: Das ist ein Akronym, kein Glossary
\newglossaryentry{CRUD}{
	name=CRUD,
	description={Abkürzung für ``Create, Read, Update and Delete''; Die Zusammenfassung der Datenmanipulationsoperationen}
}

\newglossaryentry{Gamification}{
	name=Gamification,
	description={Als Gamification oder Gamifizierung (seltener auch Spielifizierung) bezeichnet man die Anwendung spieltypischer Elemente und Prozesse in spielfremdem Kontext \cite{Gamification}}
}

\newglossaryentry{Scaffolding}{
	name=Scaffolding,
	description={Scaffolding bezeichnet das toolunterstützte Generieren von Quellcodefragmenten. Beispiel: Erstellung einer Model-Klasse inkl. zugehörigen \gls{CRUD}-Controller}
}

\newglossaryentry{Multiple Inheritance}{
	name=Multiple Inheritance,
	description={Multiple Inheritance bezeichnet ein Konzept, welches es erlaubt, eine Klassen von mehr als einer Oberklasse erben zu lassen (vgl. Single Inheritance)}
}

\newglossaryentry{ORM}{
	name=ORM,
	description={Ein ``Object Relational Mapper'' wird verwendet um Entitäten auf einer Relationalen Datenbank abzubilden und verwenden zu können}
}

\newglossaryentry{Websocket}{
	name=Websocket,
	description={Websockets ermöglichen das Herstellen von Verbindungen zwischen einem Browser und dem Webserver ausserhalb des eigentlichen HTTP Kontextes. Auf diese Weise werden ``Serverside-Pushes'' auf einfache Art und weise möglich: Über die persistente Verbindung kann der Server jederzeit ohne erneuten Request des Clients Daten an diesen senden. Websockets werden heute von praktisch allen modernen Browsern unterstützt \cite{CanIUseWebsockets}}
}

\newglossaryentry{RealTime}{
	name={Real-Time},
	description={Mit ``Real-Time'' ist in Web-Applikationen meistens ``Soft-Real-Time'' gemeint. Antwortzeiten sind somit nicht garantiert, sind aber möglichst klein gehalten (im Milisekunden-Bereich). Mittels Websockets oder BOSH \cite{BOSH} sind solche Applikationen im Web realisierbar}
}

\newglossaryentry{NoSQL}{
	name={NoSQL},
	description={NoSQL Datenbanken ist eine relativ neue Art um Datenbanken ohne die SQL-Sprache zu entwickeln. Vielfach haben solche Datenbanken keine Relationen im Sinne der Relationalen Datenbanken implementiert.}
}