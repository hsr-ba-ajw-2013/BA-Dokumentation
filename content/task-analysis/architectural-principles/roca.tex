\subsection{Resource-oriented Client Architecture (ROCA)}

\subsubsection*{Backend-Layer}
\begin{table}[H]
\tablestyle
\tablealtcolored
\begin{tabularx}{\textwidth}{l l X}
\tableheadcolor
	\tablehead ID &
	\tablehead Prinzip &
	\tablehead Erläuterung\tabularnewline
\tablebody
	\textit{RP1} & REST &
	Kommunikation mit Serverressourcen folgt dem REST-Prinzip \cite{REST}
	\tabularnewline

	\textit{RP2} & Application Logic &
	Die Businesslogik der Applikation soll im Backend bleiben.
	\tabularnewline

	\textit{RP3} & HTTP &
	Ergänzend zu \emph{RP1} findet die Kommunikation mit Serverressourcen über wohldefinierte RESTful HTTP Requests \cite{HTTPRequest} statt 
	\tabularnewline

	\textit{RP4} & Link &
	Alle URI's weisen zu einer eindeutigen Ressource.
	\tabularnewline
	
	\textit{RP5} & Non-Browser &
	Die Serverkomponente kann ohne Browser resp. Frontendkomponente (z.B. mit \emph{wget} \cite{wget} oder \emph{curl} \cite{curl}) verwendet werden.
	\tabularnewline
	
	\textit{RP6} & Should-Formats &
	Serverressourcen können ihre Daten in verschiedenen Formaten (JSON, XML) ausliefern.
	\tabularnewline
	
	\textit{RP7} & Auth &
	\emph{HTTP Basic Authentication over SSL} \cite{HTTPBasicAuth} wird als grundlegender Sicherheitsmechanismus eingesetzt. Um ältere Clients abzudecken, können formularbasierte Logins in Verbindung mit Cookies eingesetzt werden. Cookies sollen dabei jegliche zustandsbezogene Informationen enthalten.
	\tabularnewline
	
	\textit{RP8} & Cookies &
	Cookies werden nur zur Authentifizierung oder zum Tracking des Benutzers verwendet.
	\tabularnewline
	
	\textit{RP9} & Session &
	Wo möglich soll auf Sessions verzichtet werden.
	\tabularnewline
\tableend
\end{tabularx}
\caption{Die ROCA Architekturprinzipien: Backend}
\end{table}

\subsubsection*{Frontend-Layer}
\begin{table}[H]
\tablestyle
\tablealtcolored
\begin{tabularx}{\textwidth}{l l X}
\tableheadcolor
	\tablehead ID &
	\tablehead Prinzip &
	\tablehead Erläuterung\tabularnewline
\tablebody
	\textit{RP10} & Browser-Controls &
	Die Verwendung von Browser-Steuerelementen (Zurück, Aktualisieren usw.) müssen wie erwartet funktionieren und die Applikation nicht unerwartet beeinflussen.
	\tabularnewline
	
	\textit{RP11} & POSH &
	Vom Backend generiertes HTML ist semantisch korrekt \cite{SemanticHTML} und ist frei von Darstellungsinformationen
	\tabularnewline
	
	\textit{RP12} & Accessibility &
	Alle Ansichten können von Accessibility Tools (z.B. Screen Reader für Sehbehinderte) verarbeitet werden.
	\tabularnewline
	
	\textit{RP13} & Progressive Enhancement &
	Die Darstellung des Frontends soll auf aktuellsten Browsern top aussehen, aber auch auf älteren mit weniger Features verwendbar sein.
	\tabularnewline
	
	\textit{RP14} & Unobtrusive JavaScript &
	Die grundlegenden Funktionalitäten des Frontends müssen auch ohne JavaScript verwendbar sein.
	\tabularnewline
	
	\textit{RP15} & No Duplication &
	Eine Duplizierung von Businesslogik auf dem Frontend-Layer soll vermieden werden (vgl. \emph{RP2})
	\tabularnewline
	
	\textit{RP16} & Know Structure &
	Der Backendlayer soll so wenig wie möglich über die finale Struktur des HTML-Markups auf dem Frontend ``kennen''.
	\tabularnewline
	
	\textit{RP17} & Static Assets &
	Jeglicher JavaScript oder CSS Quellcode soll nicht dynamisch auf dem Backend generiert werden. Die Verwendung von Präprozessoren (SASS \cite{SASS}, LESS \cite{LESS} oder CoffeeScript \cite{CoffeeScript}) sind erlaubt und sollen sogar genutzt werden.
	\tabularnewline
	
	\textit{RP18} & History API &
	Von JavaScript ausgelöste Navigation soll über das HTML 5 History API \cite{HTML5HistoryAPI} im Browser abgebildet werden.
	\tabularnewline
\tableend
\end{tabularx}
\caption{Die ROCA Architekturprinzipien: Frontend}
\end{table}

\subsubsection*{Bewertung \& Einschätzung}
Die 18 Richtlinien des ROCA Manifests \cite{ROCA} propagieren eine verteilte Systemarchitektur.

Dabei wird die eigentliche Applikationslogik klar auf dem Backend-Layer implementiert. Dieser wird über eine wohldefinierte REST Serviceschnittstelle  angesprochen und gesteuert.

Im Frontend-Layer werden zwar die neusten Browserfeatures wie CSS 3 oder verschiedenste HTML 5 Features verwendet, es wird aber auch darauf geachtet dass das User Interface zu älteren Browsern kompatibel bleibt.

Das Projektteam kann alle ROCA Richtlinien unterstützen. Einige Bedenken sind jedoch bezüglich der Prinzipien ``RP13: Progressive Enhancement'' und ``RP14: Unobtrusive JavaScript'' anzubringen:

\begin{itemize}
	\item Ein blindes Umsetzen dieser beiden Richtlinien führt unweigerlich zu Trade-Offs in der User Experience und/oder bedeutet einen Mehraufwand in der Umsetzung.
	\item Situationsabhängig muss entschieden werden, wie wichtig die Unterstützung von alten Browsern wirklich ist
	\item Sollen lediglich aktuelle Browser auf unterschiedlichen Geräten (Smartphones, Tablets etc.) bedient werden, kann durch Verwendung von Responsive Design Techniken \cite{ResponsiveDesign} problemlos die Progressive Enhancement Anforderung umgesetzt werden.
\end{itemize}