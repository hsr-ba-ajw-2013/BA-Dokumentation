\subsection{Prototyp mit Sails.js}
Um sich ein Bild von Sails.js zu machen wurde ein einfacher Prototyp \cite{SailsPrototyp} erstellt.

Sails.js verwendet \gls{Scaffolding} um einerseits ein neues Projekt zu erstellen, andererseits auch um verschiedene Komponenten wie Model oder Controller zu erstellen. Wie im \nameref{sec:erdiagramm} beschrieben, werden u.a. ein Task und ein Resident Model (sowie entsprechende Tabelle) benötigt.

Um das \gls{ORM} ``Waterline'' \cite{Waterline} zu testen, wurden diese beiden Models implementiert. Das Task-Model mittels Sails.js definiert sieht so aus:

\begin{lstlisting}[language=JavaScript, caption=Task Model in Sails.js]
var Task = {
	attributes: {
		name: "string"
		, description: "string"
		, points: "int"
		, userId: "int"
		, communityId: "int"
	}
};
module.exports = Task;
\end{lstlisting}

Mit einer Definition eines Models wird automatisch eine \gls{REST}-API für dieses erstellt.

Damit lassen sich einerseits CRUD-Operationen direkt über HTTP ausführen, andererseits existiert auch die Möglichkeit Socket.IO \cite{SocketIO} zu aktivieren, um Models direkt von einem offenen \gls{Websocket} zu verwenden.

Dieses Feature macht Sails.js sehr nützlich für \gls{Realtime}-Applikationen.

Um eine effektive HTML-Seite darstellen zu können wird ein Controller sowie eine View benötigt. Dies wurde im Prototyp für Aufgaben (Tasks) implementiert.

\newpage
\begin{lstlisting}[language=JavaScript, caption=Task Controller in Sails.js, label=lst:sailsjstaskcontroller]
function findTaskAndUser(id, next) {
	Task.find(id).done(function(err, task) {
		User.find(task.userId).done(function(err, user) {
			next(task, user);
		});
	});
}

function renderResponse(req, res, response) {
	if (req.acceptJson) {
		res.json(response);
	} else if(req.isAjax && req.param('partial')) {
		response['layout'] = false;
		res.view(response);
	} else {
		res.view(response);
	}
}

var TaskController = {
	get: function(req, res) {
		var id = req.param('id');
		findTaskAndUser(id, function(task, user) {
			var response = {
				'task': task,
				'user': user,
				'title': task.name
			};
			renderResponse(req, res, response);
		});
	}
};
module.exports = TaskController;
\end{lstlisting}

In diesem Controller wird ein Task aufgrund des GET-Parameters ``id'' (Linie 52) geladen. Das Code-Stück zeigt die grosse Schwäche des ORMs Waterline \cite{Waterline}. Statt dass man auf dem ``task''-Objekt direkt ``.user'' aufrufen könnte, muss man den Umweg über ``User.find()'' gehen.

Bei einem ausgereiften ORM würden solche Methoden wegen der Definition von Relationen direkt zur Verfügung stehen.

Der Controller verwendet das folgende Template, um das HTML zu rendern:

\begin{lstlisting}[language=HTML, caption=Task Template]
<div id="task-display">
	<h1>Task: <%= task.name %></h1>
	<ul>
		<li>Points: <%= task.points %></li>
		<li>Created At: <%= task.createdAt %></li>
	</ul>
	<h2>User: <%= user.name %></h2>
	<ul>
		<li>Created At: <%= user.createdAt %></li>
	</ul>
</div>
<a href="#" id="reload">Reload!</a>
<script>
	$('#reload').on('click', function() {
		$.ajax('/task/get/?id=<%= task.id %>&partial=true', {
			success: function(response) {
				var $response = $('<div class="body-mock">' + response + '</div>');
				html = $response.find('#task-display');
				$('#task-display').replaceWith(html);
			}
		});
	});
</script>
\end{lstlisting}

Mit dem einfachen Script am Schluss des Task Templates wird aufgezeigt, wie man ohne Neuladen der Seite direkt HTML ersetzen kann. Dies ist grundsätzlich Framework-unabhängig und es wurde dabei auch nicht auf ``RP14'' (Unobtrusive JavaScript) Wert gelegt.

\subsubsection*{Schlussfolgerung}

Wie bereits vorher angemerkt, ist das ORM von Sails.js nicht ausgereift. Weder Assoziationen zwischen Models \cite{SailsjsModelAssociations} noch das setzen von Indizes ist möglich.

Für das Resident-Model ist es u.a. auch nötig, Facebook IDs zu speichern. Diese sind 64 Bit gross und wegen mangelhafter Unterstützung des ORMs wäre das gar nicht möglich.

Nebst dem ORM ist auch das Framework und die zugehörige Dokumentation wenig umfangreich. Auch die Community war zum Zeitpunkt der Evaluation (siehe Anhang \ref{sec:appendix-technology-evaluation} zur \nameref{sec:appendix-technology-evaluation}) sehr klein. Die Fakten deuten sehr darauf hin, Sails.js nicht für die konkrete Beispielapplokation zu verwenden. Aus diesem Grund soll ein zweiter Prototyp unter Verwendung von Express.js erstellt werden.