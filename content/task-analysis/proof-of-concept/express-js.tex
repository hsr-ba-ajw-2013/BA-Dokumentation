\subsection{Prototyp B: Express.js}

Express.js \cite{Expressjs} ist ein leichtgewichtiges Framework, welches mittels Connect-Middlewares \cite{connect} erweitert werden kann.

Der initiale Startpunkt des Express.js Prototyps ist die Datei ``app.js'' \cite{ExpressjsPrototypAppjs}. Dort werden alle benutzten Middlewares registriert, die Datenbank aufgesetzt und Controller registriert.

Ein Beispielhafter Controller ist im Quelltext \ref{lst:controllerInExpressjs} zu sehen.

\begin{lstlisting}[language=JavaScript, caption=Beispiel eines Controllers in Express.js, label=lst:controllerInExpressjs]
exports.index = function(req, res){
	// first Parameter: Template File to use
	// 2nd Parameter: Context to pass to the template
	res.render('index', { title: 'Express' });
};
\end{lstlisting}

Ein zugehöriges Template kann folgendermassen aussehen:

\begin{lstlisting}[language=HTML, caption=Template in Express.js, label=lst:templateInExpressjs]
<!DOCTYPE html>
<html>
	<head>
		<title><%= title %></title>
		<link rel='stylesheet' href='/stylesheets/style.css' />
		<%- LRScript %>
	</head>
	<body>
		<h1><%= title %></h1>
		<p>Welcome to <%= title %></p>
	</body>
</html>
\end{lstlisting}

In den vorangegangenen zwei Quelltexten \ref{lst:controllerInExpressjs} und \ref{lst:templateInExpressjs} ist ersichtlich, dass der Applikationsentwickler sehr grosse Kontrolle über Express.js hat.

Die Flexibilität von Express.js bietet sowohl Vor- als auch Nachteile für die Erstellung von Webapplikationen. Im Bezug auf die Veranschaulichung der \nameref{sec:architekturrichtlinien} ist es jedoch ein grosser Vorteil, da wenig Logik fix in Express.js eingebaut ist.