\subsubsection{\gls{ORM}}
Im Gegensatz zu den anderen evaluierten Frameworks ist in Express.js kein ORM enthalten. Aus diesem Grund wurde in einer weiteren Evaluation drei ORMs anhand der Kriterien in Tabelle \ref{tab:bewertungskriterienORM} bewertet.

\begin{table}[H]
\tablestyle
\tablealtcolored
\begin{tabularx}{\textwidth}{l l X c}
\tableheadcolor
	\tablehead ID &
	\tablehead Kriterium &
	\tablehead Erläuterung &
	\tablehead Gewichtung \tabularnewline
\tablebody
\textit{OK1} &
	Unterstützte DBs &
	Wieviele unterschiedliche Datenbanken unterstützt das ORM? Werden auch \gls{NoSQL}-Datenbanken unterstützt? \newline \emph{Hohe Bewertung = Grosse Anzahl an Datenbanken}&
	\faStar \tabularnewline
\textit{OK2} &
	Relationen &
	Sind Relationen zwischen Tabellen definierbar? Verwenden diese die datenbankspezifischen Foreign Keys dafür (falls möglich)? \newline \emph{Hohe Bewertung = Relationen möglich und verwendet Datenbank-spezifische Datentypen}&
	\faStar\faStar\faStar \tabularnewline
\textit{OK3} &
	Produktreife &
	Wie gut hat sich das ORM bis jetzt in der Realität beweisen können? Wie lange existiert es schon? Gibt es eine aktive Community und wird es aktiv weiterentwickelt? \newline \emph{Hohe Bewertung = Hohe Produktreife}&
	\faStar\faStar\faStar\tabularnewline
\textit{OK4} &
	``Ease of use'' &
	Wie einfach ist das initiale Erstellen, die Konfiguration und die Wartbarkeit von Models? Führt das ORM irgendwelchen ``syntactic sugar'' \cite{syntacticsugar} ein um die Arbeit zu erleichtern? \newline \emph{Hohe Bewertung = Hoher ``Ease of use''-Faktor} &
	\faStar \tabularnewline
\textit{OK5} &
	Testbarkeit &
	Wie gut können die mit dem Framework oder der Technologie erstellte Komponenten durch Unit Tests getestet werden? \newline \emph{Hohe Bewertung = Hohe Testbarkeit} &
	\faStar\faStar \tabularnewline
\tableend
\end{tabularx}
\caption{Bewertungskriterien für ORM-Evaluation}
\label{tab:bewertungskriterienORM}
\end{table}


\begin{table}[H]
\newcolumntype{s}{>{\centering\hsize=0.15\hsize}X}
\tablestyle
\tablealtcolored
\begin{tabularx}{\textwidth}{X s s s s s s}
\tableheadcolor
	\tablehead &
	\rotatebox{90}{\bfseries\textit{OK1 Unterstützung DBs} } &
	\rotatebox{90}{\bfseries\textit{OK2 Relationen}} &
	\rotatebox{90}{\bfseries\textit{OK3 Produktreife}} &
	\rotatebox{90}{\bfseries\textit{OK4 ``Ease of use''}} &
	\rotatebox{90}{\bfseries\textit{OK5 Testbarkeit}} &
	\rotatebox{90}{\bfseries\textit{Total}}
	\tabularnewline
\tablebody
	\textit{JugglingDB} &
	\threeStars &
	\oneStar &
	\oneStar &
	\twoStars &
	\twoStars &
	\directlua{
		tex.print(math.round(
			(3 * 1 +
			1 * 3 +
			1 * 3 +
			2 * 1 +
			2 * 2) / 5
		))
	}
	\tabularnewline

	\textit{Node-ORM2} &
	\twoStars &
	\twoStars	&
	\oneStar &
	\threeStars &
	\oneStar &
	\directlua{
		tex.print(math.round(
			(2 * 1 +
			2 * 3 +
			1 * 3 +
			3 * 1 +
			1 * 2) / 5
		))
	}
	\tabularnewline

	\textit{Sequelize} &
	\oneStar &
	\twoStars &
	\twoStars &
	\twoStars &
	\oneStar &
	\directlua{
		tex.print(math.round(
			(1 * 1 +
			2 * 3 +
			2 * 3 +
			2 * 1 +
			1 * 2) / 5
		))
	}
	\tabularnewline
\tableend
\end{tabularx}
\caption{Bewertungsmatrix JavaScript ORMs}
\label{tab:bewertungsmatrixORM}
\end{table}

Alle verglichenen ORMs haben eine ähnliche Gesamtbewertung. Bei ``Sequelize'' stechen jedoch die Produktreife und die Unterstützung für Relationen heraus.

Diese zwei Gründe zusammen mit der aktuellen Roadmap \cite{RoadmapSequelize} haben schliesslich zur Überzeugung geführt, dass Sequelize die richtige Wahl ist.