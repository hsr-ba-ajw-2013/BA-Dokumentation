\section{Entscheidung}

Nach eingängiger Auseinandersetzung mit den drei Technologien Ruby, Java und JavaScript ergeben sich für die finale Technologieentscheidung folgende drei Frameworkkandidaten:


\begin{table}[H]
\newcolumntype{s}{>{\centering\hsize=0.15\hsize}X}
\tablestyle
\tablealtcolored
\begin{tabularx}{\textwidth}{X s s s s s s s}

\tableheadcolor
	\tablehead &
	\rotatebox{90}{\bfseries\textit{TK1 Eigenkonzepte} } &
	\rotatebox{90}{\bfseries\textit{TK2 Eignung}} &
	\rotatebox{90}{\bfseries\textit{TK3 Produktreife}} &
	\rotatebox{90}{\bfseries\textit{TK4 Aktualität}} &
	\rotatebox{90}{\bfseries\textit{TK5 ``Ease of use''}} &
	\rotatebox{90}{\bfseries\textit{TK6 Testbarkeit}}
	\tabularnewline
\tablebody
	\textit{Ruby: Ruby on Rails} &
	\oneStar &
	\oneStar &
	\threeStars &
	\oneStar &
	\threeStars &
	\twoStars
	\tabularnewline

	\textit{Java: Studienarbeit-Zusammenstellung}	&
	\threeStars &
	\threeStars &
		&
		&
	\threeStars &
	\twoStars
	\tabularnewline

	\textit{JavaScript: Sails} &
	\twoStars &
	\twoStars &
	\oneStar &
	\threeStars &
	\twoStars &
	\oneStar
	\tabularnewline

\tableend
\end{tabularx}
\caption{Finale Frameworkkandidaten für Technologieentscheidung}
\end{table}