\subsection{Ruby}

Insbesondere mit dem Framework \emph{Ruby on Rails} \cite{RubyOnRails} wurde Ruby für die Entwicklung von Umfangreichen Webapplikationen seit Veröffentlichung in den 90ern immer beliebter. Mit fast kindlicher Selbstverständlichkeit bringt Ruby viele Konzepte wie \gls{Multiple Inheritance} (in Form von Mixins) oder die funktionale Behandlung von jeglichen Werten/Objekten von Haus aus mit.

Für den Einsteiger etwas verwirrend setzt es zudem auf eine für den Menschen ``leserliche'' Syntax als beispielsweise von Java oder anderen verwandten Sprachen gewohnt. Folgende Codebeispiele bewirken die selbe Ausgabe auf der Kommandozeile, unterscheiden sich aber deutlich in ihrer Formulierung:

\begin{lstlisting}[language=Java, caption=Negierte if-Abfrage in Java]
if(!enabled) {
	System.out.println("Ich bin deaktiviert!");
}
\end{lstlisting}

\begin{lstlisting}[language=Ruby, caption=Negierte if-Abfrage in Ruby]
puts "Ich bin deaktiviert!" unless enabled
\end{lstlisting}

Während der kurzen Technologieevaluationsphase wurde im Bereich Ruby das Hauptaugenmerk auf \emph{Ruby on Rails} gelegt. Insbesondere die \gls{Scaffolding}tools und der von denen generierte Quellcode wurde näher begutachtet. Die Resultate sind wiederum auf die sechs Bewertungskriterien appliziert worden.

\subsubsection*{Bewertung Ruby on Rails}
\begin{table}[H]
\newcolumntype{s}{>{\centering\hsize=0.15\hsize}X}
\tablestyle
\tablealtcolored
\begin{tabularx}{\textwidth}{X s s s s s s s}
\tableheadcolor
	\tablehead &
	\rotatebox{90}{\bfseries\textit{TK1 Eigenkonzepte} } &
	\rotatebox{90}{\bfseries\textit{TK2 Eignung}} &
	\rotatebox{90}{\bfseries\textit{TK3 Produktreife}} &
	\rotatebox{90}{\bfseries\textit{TK4 Aktualität}} &
	\rotatebox{90}{\bfseries\textit{TK5 ``Ease of use''}} &
	\rotatebox{90}{\bfseries\textit{TK6 Testbarkeit}} &
	\rotatebox{90}{\bfseries\textit{Gesamtbewertung}}
	\tabularnewline
\tablebody
	\textit{Ruby on Rails} &
	\oneStar &
	\oneStar &
	\threeStars &
	\oneStar &
	\threeStars &
	\twoStars &
	\directlua{
		tex.print(math.round(
			(1 * 3 +
			1 * 3 +
			3 * 3 +
			1 * 1 +
			3 * 2 +
			2 * 2) / 6
		))
	}
	\tabularnewline
\tableend
\end{tabularx}
\caption{Bewertung Ruby on Rails}
\end{table}


\subsubsection*{Interpretation}
Nach genauerem Befassen mit Ruby on Rails sind die Einschätzung des Projektteams gespalten.

Zum Einen minimiert Ruby on Rails den Aufwand für das Erledigen von Routineaufgaben extrem (\gls{Scaffolding}). Der generierte Code ist sofort verwendbar und, gute Ruby-Kenntnisse vorausgesetzt, gut erweiterbar.

Zum Anderen ist aber gerade die Einfachheit, wie bspw. Controllers oder Models erzeugt und in den Applikationsablauf eingebunden werden, alles Andere als optimal wenn es darum geht, Architekturrichtlinien eindeutig und klar demonstrieren zu können.

Unter dem Vorbehalt, dass Ruby on Rails für die Demonstration der definierten Architekturrichtlinien evtl. nicht die richtige Wahl sein könnte, kann das Projektteam nur eine bedingte Empfehlung für das Ruby Framework abgeben.