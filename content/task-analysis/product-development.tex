\section{Produkteentwicklung}
Die Findung einer passenden Produkteidee gestaltete sich unter den im vorherigen Abschnitt erwähnten Bedingungen nicht unbedingt als einfach:

Zwar soll ein Gros des Arbeitsaufwandes in das Entwickeln einer beispielhaften Architektur fliessen, diese soll aber in einem für Studierende möglichst attraktiven Gewand präsentiert werden.

\subsection{Workshop}
Um die zweitrangige Prozedur der Ideenfindung pragmatisch abhandeln zu können wurde ein Workshop mit Brainstorming und anschliessender Diskussionsrunde durchgeführt. Folgende Tabelle zeigt die Favoriten aus einem Pool generierter Ideen.

Die Spalte \emph{Potential} bewertet jede Idee nach subjektiver Einschätzung des Projektteams unter berücksichtigung folgender Faktoren:
\begin{itemize}
	\item Funktionsumfang
	\item Konzeptionelle und technische Herausforderung
	\item Attraktivität (Für Projektteam)
	\item Attraktivität (Für Studierende des Moduls Internettechnologien)
\end{itemize}

\begin{table}[H]
\tablestyle
\tablealtcolored
\begin{tabularx}{\textwidth}{l X X l}
\tableheadcolor
	\tablehead Idee &
	\tablehead Pro &
	\tablehead Contra &
	\tablehead Potential \tabularnewline
\tablebody
	\textit{\gls{WG}-Aufgabenverwaltung} &
	Viele Studierende identifizieren sich tendenziell da sie selber in einer \gls{WG} wohnen, Faktor \emph{Gamification} sehr interessant &
	&
	hoch\tabularnewline

	\textit{Instant Messanger} &
	Attraktive Features wären realisierbar (Realtime, Websockets etc.) &
	Funktionelle Anforderungen könnten Rahmen sprengen &
	mittel \tabularnewline

	\textit{Aufgabenverwaltung} &
	Evtl. gute Verwendung des Produkts &
	``\emph{Gibt's wie Sand am Meer}'', viele bestehnde Beispielapplikationen\cite{TodoMVC} &
	mittel \tabularnewline

	\textit{Chat} &
	&
	Bereits oft verwendet in bestehender Vorlesung, abgenutzte Thematik &
	tief \tabularnewline
	
	\textit{Forum} &
	&
	``\emph{Gibt's wie Sand am Meer}'' &
	tief \tabularnewline
\tableend
\end{tabularx}
\caption{Produktideenpool}
\end{table}

Am verheissungsvollsten wurde die Idee des \gls{WG} Aufgabenverwaltungstool eingeschätzt und gefiel dem gesamten Team von Beginn an ziemlich gut. Die Thematik \gls{Gamification} in einem konkreten Produkt umsetzen zu können eliminierte schlussendlich die letzten Zweifel.

In einem nächsten Schritt wurde die rohe Produkteidee mit einem Mindmap (siehe Anhang \ref{sec:produkteentwicklung}) weiter ausgebaut und die ersten funktionalen Anforderungen wurden entwickelt. Daneben konnte eine konkrete Kurzbeschreibung sowie das erste Branding für das geplatne Produkt formuliert resp. entworfen werden.


\subsection{Die finale Produkteidee: Roomies}
\emph{Roomies} soll einer \gls{WG} ermöglichen, anfallende Aufgaben leicht unter den verschiedenen Bewohnern zu organisieren. Damit auch langweilige Ämtchen endlich erledigt werden, schafft Roomies durch ein Ranglisten- und Badgesystem (\gls{Gamification}) einen Anreiz, um seine Mitbewohner übertrumpfen zu wollen.

Durch das Aufgreifen einer Thematik aus dem Studentenalltag soll Roomies für lernende aus dem Modul \emph{Internettechnologien} einen leichten Einstieg in die tendenziell trockene Materie der Softwarearchitektur bieten.

\subsection{Branding}
Der namensgebende Audruck \emph{Roomie} stammt aus dem US-amerikanischen und bedeutet soviel wie \emph{Mitbewohner} oder \emph{Zimmernachbar}\cite{Roomie}. Passend dazu soll neben dem Namen auch das restliche Produktbranding an die US-amerikanische Collegewelt angelehnt werden.

Vom Logo über die Farbwahl bis zum späteren User Interface Design sollen folgende Stilelemente als roter Faden verwendet werden:

\begin{enumerate}
	\item \emph{Gedimmte} Farben, keine grellen Akzente
	\item Simple, aber eingängige und klar definierte Formensprache
	\item Serifenbetonte Schriftart als Stilmittel
\end{enumerate}

\begin{figure}[H]
	\centering
	\includegraphics[width=5cm]{content/images/roomies-withshadow.png}
	\caption{Roomies Logo im College Stil}
\end{figure}

\begin{figure}[H]
	\centering
	\includegraphics[width=12cm]{content/images/logo-variants.png}
	\caption{Roomies Logo in verschiedenen Grössen \& Varianten}
\end{figure}