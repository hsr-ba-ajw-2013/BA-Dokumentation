\chapter{LaTeX Demo Beispiele}

\section*{Glossary Example}

\gls{sqlite} <-- Glossar \lipsum*[1]

\section*{Bibliography and Citation Example}

Dies ist ein Zitat aus einem Buch \cite{Matthews201111}.
\lipsum*[2]

\section*{Table Example}
\subsection*{Normal}
\begin{table}[H]
\tablestyle
\begin{tabularx}{\textwidth}{lXXlX}
\tableheadcolor
   \tablehead Tabellenkopf &
   \tablehead Tabellenkopf &
   \tablehead Tabellenkopf &
   \tablehead Tabellenkopf &
   \tablehead Tabellenkopf \tabularnewline
\tablebody
   \textit{Beschreibung} & Inhalt & Inhalt & Inhalt & Inhalt \tabularnewline
   \textit{Beschreibung} & Inhalt & Inhalt & Inhalt & Inhalt \tabularnewline
\tableend
\end{tabularx}
\caption{Tabelle mit tabularx}
\end{table}

\subsection*{Zebra}
\begin{table}[H]
\tablestyle
\tablealtcolored
\begin{tabularx}{\textwidth}{lXXlX}
\tableheadcolor
   \tablehead Tabellenkopf &
   \tablehead Tabellenkopf &
   \tablehead Tabellenkopf &
   \tablehead Tabellenkopf &
   \tablehead Tabellenkopf \tabularnewline
\tablebody
   \textit{Beschreibung} & Inhalt & Inhalt & Inhalt & Inhalt \tabularnewline
   \textit{Beschreibung} & Inhalt & Inhalt & Inhalt & Inhalt \tabularnewline
   \textit{Beschreibung} & Inhalt & Inhalt & Inhalt & Inhalt \tabularnewline
   \textit{Beschreibung} & Inhalt & Inhalt & Inhalt & Inhalt \tabularnewline
\tableend
\end{tabularx}
\caption{Zebra}
\end{table}


\section*{Code Listings}

\begin{lstlisting}[caption=Fancy Java Code Listing]
/**
 * Fancy Listings
 */
public void main() {
   bla("Bla Bla");
}
\end{lstlisting}

\begin{lstlisting}[language=XML, caption=Fancy XML]
<root>
   <node name="bla" />
   <node>
      <subnode>Content</subnode>
   </node>
</root>
\end{lstlisting}

\section*{UML}
Using tikz-uml --- \url{http://www.ensta-paristech.fr/~kielbasi/tikzuml/index.php?lang=en&id=doc}.

\begin{tikzpicture}
\umlemptyclass{A1}
\umlemptyclass[x=5]{A2}
\umlassoc[arg1=arg1, mult1=mult1, arg2=arg2, mult2=mult2]{A1}{A2}
\end{tikzpicture}