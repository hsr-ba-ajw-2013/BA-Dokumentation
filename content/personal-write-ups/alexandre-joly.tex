\section{Alexandre Joly}





\subsection*{Evaluation}
Da wir für die Demonstration der Architekturrichtlinien eine Beispielanwendung brauchten, war eine kleine Evaluation von verschiedenen Technologien nötig.

Für die Evaluation wurden vom Team und vom Betreuer die Technologien node.js JavaEE und Ruby on Rails definiert worden. Die von mir ausgewertete Technologie, Ruby on Rails, stand zum Schluss der Bewertung Kopf an Kopf mit node.js. Die Entscheidung fiel, aufgrund von persönlichen Präferenzen und diversen Erfahrungen mit JavaScript. Ich persönlich konnte bereits meine erste ernsthafte Erfahrungen mit JavaScript bei der Studienarbeit sammeln. Aus diesen Grund kam mir diese Entscheidung entgegen.

Zusätzlich zur Evaluation brauchte es eine Idee für eine Beispielanwendung. Da es als Unterrichtsmaterial gebraucht werden sollte, wollten wir versuchen die Studenten zu erreichen, indem wir eine App schreiben, die sie brauchen können. Da das WG-Leben für fast jeden Student bekannt ist, trifft eine TODO-App für Wohngemeinschaften ganz gut.

\subsection*{Beispielapplikation Roomies}
Mit der Implementation der Beispielanwendung kam für einen Entwickler der interessantere Teil des Projekts, nämlich das Programmieren.
Schnell war klar dass ein Lösung für das Konzept ``Unobstrusive JavaScript'' her musste. Darum wurden zwei Task parallel abgewickelt. Die eine, das Implementieren der Use Case und die Andere, die Entwicklung eines Frameworks, welches Code Sharing ermöglicht. Das Entwickeln und das Integrieren in die Beispielanwendung dieses Frameworks namens ``barefoot'' dauerte ein wenig länger als von uns erwartet, was zu eine kleine Verzögerung von mehreren Meilenstein führte. Trotz allem konnten zwei sehr schöne Produkte geschrieben werden. Die Demo-App ``Roomies'', welches gute Beispiele für die Architekturrichtlinien zu Verfügung stellt. Und ``barefoot'', ein Framework für Code Sharing zwischen Browser und Server, welche einigen Entwickler eine riesige Erleichterung sein wird. Es wundert mich wie schnell, wie erfolgreich dieses Framework tatsächlich wird. Ich werde es bestimmt weiterverfolgen.

\subsection*{Architekturrichtlinien}
Viele der Richtlinien sind Web-Entwickler bewusst oder unbewusst bekannt. Denn manche sind eine Selbstverständlichkeit geworden. Trotzdem werden immer noch Web-Seiten entwickelt die wichtige Prinzipien verletzen.
Meiner Meinung nach sollten diese Prinzipien fester Bestandteil der Projektplanung sein.


\subsection*{Fazit}
Diese Bachelorarbeit hat mich weitergebracht. Ich konnte auf JavaScript-Seite viel lernen, wichtige Prinzipien für Web-Anwendungen und nützliche Automatisierungstools kenne lernen.

Die Zusammenarbeit mit dem BA-Team war sehr bereichernd. Ich konnte aus den verschiedenen Erfahrungen des Teams sehr viel profitieren und lernen. Wie bereits gewohnt aus der Studienarbeit, war die Zusammenarbeit mit Herrn Rudin und Kevin Gaunt durchaus positiv. Die konstruktiven Vorschläge und Tipps halfen dem Produkt und Team sehr.