\section{Alexandre Joly}





\subsection*{Evaluation}
Um die Architekturrichtlinien zu demonstrieren war eine Beispielanwendung und eine entsprechende Evaluation von verschiedenen Technologien nötig.

Für die Evaluation wurden vom Team und Betreuer die Technologien node.js JavaEE und Ruby on Rails definiert. Die von mir ausgewertete Technologie, Ruby on Rails, stand zum Schluss der Bewertung Kopf an Kopf mit node.js. Die Entscheidung fiel aufgrund von persönlichen Präferenzen und diversen Erfahrungen mit JavaScript. Ich persönlich konnte bereits meine ersten ernsthaften Erfahrungen mit JavaScript bei der Studienarbeit sammeln. Aus diesen Grund kam mir diese Entscheidung entgegen.

Zusätzlich zur Technologie brauchte es eine Idee für die Beispielanwendung. Da diese als Unterrichtsmaterial gebraucht werden sollte, wollten wir versuchen die Studenten zu erreichen. Deswegen wollten wir eine Applikation schreiben, die sie brauchen können. Da das WG-Leben für fast jeden Student bekannt ist, ist eine TODO-App für Wohngemeinschaften ganz gut.

\subsection*{Beispielapplikation Roomies}
Mit der Implementation der Beispielanwendung kam für die Entwickler der interessantere Teil des Projekts, nämlich das Programmieren.
Schnell war klar dass eine Lösung für das Konzept ``Unobtrusive JavaScript'' her musste. Darum wurden zwei Tasks parallel abgewickelt. Die Eine, das Implementieren der Use Cases und die Andere, die Entwicklung eines Frameworks welches Code Sharing ermöglicht. Das Entwickeln und das Integrieren des Frameworks ``barefoot'' in die Beispielapplikation dauerte ein wenig länger als von uns erwartet. Dies führte zu einer kleinen Verzögerung von mehreren Meilensteinen. Trotz allem konnten zwei sehr schöne Produkte geschrieben werden. Die Demo-App ``Roomies'', welche gute Beispiele für die Architekturrichtlinien zur Verfügung stellt. Und ``barefoot'', ein Framework für Code Sharing zwischen Browser und Server, welches einigen Entwicklern eine riesige Erleichterung sein wird. Es wundert mich wie schnell wie erfolgreich dieses Framework tatsächlich wird. Ich werde es bestimmt weiterverfolgen.

\subsection*{Architekturrichtlinien}
Viele der Richtlinien sind Web-Entwicklern bewusst oder unbewusst bekannt. Manche davon sind eine Selbstverständlichkeit geworden. Trotzdem werden immer noch Webseiten entwickelt, welche wichtige Prinzipien verletzen.
Meiner Meinung nach sollten diese Prinzipien fester Bestandteil der Projektplanung sein.

\subsection*{Fazit}
Diese Bachelorarbeit hat mich weitergebracht. Ich konnte auf JavaScript-Seite viel lernen, wichtige Prinzipien für Web-Anwendungen und nützliche Automatisierungstools kennenlernen.

Die Zusammenarbeit mit dem BA-Team war sehr bereichernd. Ich konnte aus den verschiedenen Erfahrungen des Teams sehr viel profitieren und lernen. Wie bereits gewohnt aus der Studienarbeit, war die Zusammenarbeit mit Herrn Rudin und Kevin Gaunt durchwegs positiv. Die konstruktiven Vorschläge und Tipps halfen dem Produkt und Team sehr.