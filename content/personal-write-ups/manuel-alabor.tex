\section{Manuel Alabor}

Bei der Auswahl dieser Bachelorarbeit zog mich besonders die offene Aufgabenstellung an. Zwar musste auf diese Weise nach Projektstart zuerst einmal klar vereinbart werden, in welche Richtung die Arbeit streben sollte. Nach der klaren Definition konnte ich mich dafür umso mehr mit der Aufgabe identifizieren und war entsprechend motiviert.

\subsection*{Evaluation}

In der Technologieevaluation durfte ich mich um den Kandidaten ``Java'' kümmern. Da ich mich bereits im Rahmen der Studienarbeit detailliert auf dem Gebiet von RESTful Webservices mit dieser Technologie auseinandergesetzt hatte, konnte ich mich daneben intensiver um die grafische Konzipierung von ``Roomies'' kümmern.

Rückblickend hätten wir bei der Auswahl der Technologien meiner Meinung nach sorgfältiger und strukturierter vorgehen können. Unter dem gegebenen Zeitrahmen halte ich unsere pragmatische Lösung aber für zweckmässig.

Die Auswahl der zu untersuchenden Architekturkonzepte empfand ich als weniger spannend, freute mich aber umso mehr auf die praktische Umsetzung und konkrete Demonstration dieser.


\subsection*{Umsetzung Roomies \& barefoot}

Mit grossem Elan starteten wir nach der Vorstudie in die Implementation unserer Beispielapplikation. Zwar hatten wir bereits damit gerechnet, dass die Umsetzung der Richtlinie \emph{Unobtrusive JavaScript} und \emph{No Duplication} nicht einfach werden würde, doch zeichnete sich nach der ersten Iteration klar ab, dass wir diese immer noch unterschätzt hatten.

Aus diesem Grund durfte ich mich ab Iteration 2 speziell um diesen Aufgabenbereich kümmern, während meine zwei Teamkollegen mit der Umsetzung der Use Cases fortfuhren. Nach einigen Experimenten war für mich einleuchtend, dass es keinen Sinn machen würde, die für das ``Code Sharing'' notwendige Logik fix mit ``Roomies'' zu verdrahten. So entstand mit ``barefoot'' ein eigenständiges Framework.

Die Integration von ``barefoot'' in den bestehenden Quelltext von ``Roomies'' war anspruchsvoll, da mehrfach Fehler im Framework korrigiert werden mussten. Ich denke aber, dass die finalen Quelltext-Produkte jede Minute der investierten Arbeit wert sind: ``Roomies'' bietet anschauliche Beispiele zu abstrakten Konzepten und ``barefoot'' stellt eine elegante Lösung für ein wiederkehrendes Problem dar.


\subsection*{Richtliniendemonstration}

Den zweiten Hauptteil unserer Arbeit widmeten wir der ausführlichen Auseinandersetzung mit den gewählten Architekturrichtlinien. Ich habe die Recherchen zu den mitunter sehr unterschiedlichen Themenfeldern sehr genossen. Dabei hatte ich im Vergleich zu meinen Kommilitonen aber wohl das Glück, keine Konzepte mit thematisch übermässigen Überschneidungen ausgesucht zu haben.

Ich halte die Analyse der Architekturkonzepte für sehr gelungen. Insbesondere die Diskussion zum Ende jeder Untersuchung bietet mit vielen Querverweisen Möglichkeiten zum ``Blick über den Tellerrand''.


\subsection*{Team}

Das bereits aus der Studienarbeit bewährte Betreuer- und Projektteam hat auch während dieser Bachelorarbeit harmonisch und lösungsorientiert funktioniert. Besonders geschätzt habe ich einmal mehr die offene Diskussionskultur und die Hilfsbereitschaft aller Personen.


\subsection*{Fazit}

Zusammenfassend waren die letzten 16 Wochen allesamt sehr interessant und lehrreich für mich. In allen Bereichen konnte ich neue Dinge lernen oder bestehendes Wissen vertiefen.

In Erinnerung behalten werde ich insbesondere die praktischen Erfahrungen mit LaTeX welches zur Erstellung der Projektdokumentation genutzt wurde, aber auch die vielen Herausforderungen welche es im Bereich von ``barefoot'' zu bestehen gab.
