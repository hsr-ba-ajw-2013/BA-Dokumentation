\section{Michael Weibel}

Die letzten 16 Wochen Bachelorarbeit waren vollgepackt mit einem interessanten Thema, viel Arbeit und interessanten Diskussionen. Hier ein Überblick über meine persönliche Erfahrung.

\subsection*{Evaluation}
Nach der initialen Themenfindung war schnell klar, dass wir eine Beispielapplikation benötigen um die Architekturrichtlinien zu zeigen.

Ich widmete mich dem Thema JavaScript, welches schlussendlich aufgrund der Kriterien und der persönlichen Interessen des Projektteams und des Betreuers Hans Rudin ausgewählt wurde. Dies hat mich gefreut, schliesslich befasse ich mich schon länger intensiv mit JavaScript und Node.js.

Die Evaluationphase und deren Kriterien für die Auswahl der Sprache und des Frameworks war unter anderem aus Zeitgründen mit persönlichen Vorlieben geprägt. In einem anderen Projekt müssten wohl die Kriterien spezifischer gewählt werden und eine grössere Bandbreite an Technologien mit einem Proof of Concept testen.

\subsection*{Beispielapplikation Roomies}
Als nächstes kam ein grosser Teil Implementation. Bald wurde klar, dass sich das Projektteam in zwei Teile aufteilen wird, um einerseits die Use Cases zu implementieren und andererseits ein Code Sharing Framework zu entwickeln.

Ich war im Team zusammen mit Alexandre Joly um die Use Cases zu implementieren. Wir kamen relativ gut voran, jedoch haben wir zuwenig auf eine entkoppelte API geachtet. Wenn wir das gemacht hätten, wäre der spätere Umstieg auf ``barefoot'' wohl einfacher und schneller vonstatten gegangen.

Insgesamt war die Implementation aber gut und wir konnten eine Menge lernen. Das entstandene Framework ``barefoot'' ist sehr interessant und ich könnte mir vorstellen, dies auch bei anderen Projekten einzusetzen.

\subsection*{Architekturrichtlinien}
Die Architekturrichtlinien sind meiner Meinung nach definitiv hilfreich und ein grosser Teil wird wohl an vielen Orten unbewusst umgesetzt.

Die Richtlinien an sich waren für mich nicht neu. Viele davon sind mir aufgrund meiner Erfahrung klar. Die intensive Beschäftigung damit gibt jedoch eine neue Sicht darauf und ein gutes Vokabular.

Es gibt einen Punkt, welchen ich an den Richtlinien bemängle: Viele davon sagen eigentlich dasselbe aus oder sind zumindest sehr eng verwandt. Natürlich kann argumentiert werden, dass diese eng verwandten Richtlinien entsprechend viele verschiedene Sichten auf sehr ähnliche Probleme geben. Dennoch denke ich, dass die Richtlinien insgesamt gekürzt werden könnten.

\subsection*{Fazit}
Insgesamt war die Bachelorarbeit lehrreich für mich. Dies insbesondere auch bezüglich der Serverseitigen Verwendung von JavaScript.

Die Zusammenarbeit im Team war wie schon in der Studienarbeit sehr gut. Die Zusammenarbeit mit den Betreuern war ebenfalls sehr angenehm und produktiv.